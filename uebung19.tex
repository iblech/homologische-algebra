\documentclass{uebblatt}
\haiitrue

\begin{document}

\maketitle{19}{}

\begin{aufgabe}{Nullmorphismen}
\begin{enumerate}
\item Zeige: Ein Morphismus~$f$ in~$D(\A)$ ist genau dann Null, wenn ein
Quasiisomorphismus~$s$ existiert, sodass~$sf$ nullhomotop ist.
\item Zeige: Ein Komplex ist genau dann azyklisch, wenn sein
Identitätsmorphismus in~$D(\A)$ Null ist.
\item Finde einen Morphismus, der nicht in~$\Kom(\A)$, aber in~$\K(A)$ Null
ist; einen, der nicht in~$\K(\A)$, aber in~$D(\A)$ Null ist; einen, der nicht
in~$D(\A)$ Null ist, aber in Kohomologie den Nullmorphismus induziert.
\item Finde einen nichttrivialen Morphismus zwischen den folgenden Komplexen
in~$D(\mathrm{Ab})$.
\[ \xymatrix{
  \cdots \ar[r] & 0 \ar[r] & 0 \ar[r] & \ZZ/(2) \ar[r] & 0 \ar[r] & \cdots \\
  \cdots \ar[r] & 0 \ar[r] & \ZZ \ar[r] & 0 \ar[r] & 0 \ar[r] & \cdots
} \]
\end{enumerate}
\end{aufgabe}

\begin{aufgabe}{Komplexe mit vorgegebener Kohomologie}
Sei~$\B$ eine Serresche Unterkategorie einer abelschen Kategorie~$\A$.
Sei~$\Kom_\B(\A)$ die volle Unterkategorie derjenigen Objekte~$K^\bullet$
von~$\Kom(\A)$, deren Kohomologien~$H^n(K^\bullet)$ alle in~$\B$ liegen.
Dann definieren wir~$D_\B(\A) \defeq \Kom_\B(\A)[\text{qis}^{-1}]$.
\begin{enumerate}
\item Zeige, dass~$D_\B(\A)$ auf kanonische Weise eine volle
Unterkategorie von~$D(\A)$ ist.
\item Sei jedes Objekt aus~$\B$ ein Unterobjekt eines Objekts aus~$\B$, welches
als Objekt von~$\A$ injektiv ist. Zeige, dass der kanonische Funktor~$D^+(\B)
\to D^+_\B(\A)$ eine Kategorienäquivalenz ist.
% Beweis in Huybrechts, Fourier--Mukai transforms in algebraic geometry, Seite 42.
\end{enumerate}
\end{aufgabe}

\begin{aufgabe}{Zerfallende kurze exakte Sequenzen}
\begin{enumerate}
\item
Zeige, dass folgende Bedingungen an eine kurze exakte Sequenz~$0 \to X \xra{i} Y
\xra{p} Z \to 0$ in einer abelschen Kategorie äquivalent sind:
\begin{enumerate}
\item[1.] Die Sequenz zerfällt.
\item[2.] Es gibt einen Morphismus~$s : Z \to Y$ mit~$ps = \id$.
\item[3.] Es gibt einen Morphismus~$t : Y \to X$ mit~$ti = \id$.
\end{enumerate}
\item Folgere, dass wenn~$X$ injektiv oder~$Z$ projektiv ist, die Sequenz
zerfällt.
\item Folgere, dass additive Funktoren stets zerfallende kurze exakte Sequenzen
bewahren.
\end{enumerate}
\end{aufgabe}

\begin{center}\emph{-- Bitte wenden. --}\end{center}

\newpage
\begin{aufgabe}{Klassifikation kurzer exakter Sequenzen}
Für Objekte~$X$ und~$Y$ in einer abelschen Kategorie~$\A$ ist~$\Ext^1(X,Y)$ die
Klasse aller kurzen exakten Sequenzen der Form~$0 \to Y \to {?} \to X \to 0$
modulo der Äquivalenzrelation "`$\text{obere Zeile} \sim \text{untere Zeile}$"'
für jedes kommutative Diagramm der Form
\[ \xymatrix{
  0 \ar[r] & Y \ar[r] \ar@{=}[d] & {?} \ar[r]\ar[d] & X \ar[r]\ar@{=}[d] & 0 \\
  0 \ar[r] & Y \ar[r] & {?}' \ar[r] & X \ar[r] & 0. \\
} \]
\begin{enumerate}
\item Ist~$\varphi : Y \to Y'$ ein Morphismus und~$E : 0 \to Y \to Z \to X
\to 0$ eine kurze exakte Sequenz, so können wir das Diagramm
\[ \xymatrix{
  {}\phantom{\varphi}E{:} & 0 \ar[r] & Y \ar[r]^i \ar[d]_{\varphi} & Z \ar[r]\ar@{-->}[d] & X \ar[r]^p\ar@{=}[d] & 0 \\
  \varphi E{:} & 0 \ar[r] & Y' \ar@{-->}[r] & Z \amalg_Y Y' \ar@{-->}[r] & X \ar[r] & 0
} \]
konstruieren. Zeige, dass die untere Zeile~$\varphi E$ wieder exakt ist, und dass die
Zuordnung~$E \mapsto \varphi E$ eine wohldefinierte Abbildung~$\Ext^1(X,Y) \to
\Ext^1(X,Y')$ induziert.

{\scriptsize\emph{Tipp:} Der auftretende Kolimes kann auch als~$(Z \oplus Y') /
\{ (\varphi(y), -i(y)) \,|\ y\?Y \}$ geschrieben werden. Beweise die Behauptung
mit Diagrammjagden.\par}
\end{enumerate}
Analog kann man auch für Morphismen~$\psi : X' \to X$ eine wohldefinierte
Zuordnung~$E \in \Ext^1(X,Y) \mapsto E \psi \in \Ext^1(X',Y)$ konstruieren.
Nimm zur Kenntnis: $(\varphi_1 \circ \varphi_2) E =
\varphi_1(\varphi_2 E)$, $E (\psi_1 \circ \psi_2) = (E \psi_1) \psi_2$,
$(\varphi E) \psi = \varphi (E \psi)$.

Sind~$E,E' \in \Ext^1(X,Y)$, so definieren wir ihre \emph{Baersumme}
als~$\nabla_Y (E \oplus E') \Delta_X \in \Ext^1(X,Y)$. Dabei sind~$\Delta_X : X
\to X \oplus X$ und~$\nabla_Y : Y \oplus Y \to Y$ die kanonischen Morphismen
und~$E \oplus E'$ die sich durch direkte Summenbildung ergebende Sequenz
in~$\Ext^1(X \oplus X, Y \oplus Y)$.

\begin{enumerate}
\addtocounter{enumi}{1}
\item Zeige, dass~$\Ext^1(X,Y)$ mit der Baersumme zu einer abelschen Gruppe mit
Nullelement~$[0 \to Y \to X \oplus Y \to X \to 0]$ wird.
\end{enumerate}

\end{aufgabe}

\begin{aufgabe}{Die kanonische Filtrierung eines Komplexes}
Die \emph{gute Abschneidung} eines Komplexes~$K^\bullet$ über einer abelschen
Kategorie~$\A$ ist der Komplex
\[ (\tau_{\leq n} K^\bullet)^i \defeq \begin{cases}
  K^\bullet, & \text{für $i < n$}, \\
  \ker(K^n \to K^{n+1}), & \text{für $i = n$}, \\
  0, & \text{für $i > n$}.
\end{cases} \]

\begin{enumerate}
\item Es gibt auch die \emph{dumme Abschneidung}. Die gute Abschneidung hat ihr
gegenüber den Vorteil, dass~$H^i(\tau_{\leq n} K^\bullet)$ noch für~$i \leq n$
mit~$H^i(K^\bullet)$ übereinstimmt. Beweise diesen Sachverhalt.
\item Bestimme den Kokern der kanonischen Inklusion~$\tau_{\leq n-1}K^\bullet
\hookrightarrow \tau_{\leq n}K^\bullet$.
\item Finde einen Quasiisomorphismus vom Kokern in den im Grad~$n$
konzentrierten Komplex~$H^n(K^\bullet)[-n]$.
\item Folgere: In~$K(D(\Kom^b(\A)))$ gilt die
Rechnung~$K^\bullet = \sum_n (-1)^n\, H^n(K^\bullet)$.

{\scriptsize
Die abgeleitete Kategorie~$D(\Kom^b(\A))$ ist im Allgemeinen nicht abelsch.
Ihre K-Theorie ist daher anders zu definieren: als die von den Objekten
von~$D(\Kom^b(\A))$ erzeugte abelsche Gruppe modulo den Relationen~$X = X' +
X''$ für jedes \emph{ausgezeichnete Dreieck}~$X' \to X \to X'' \to$. Das muss dich
jetzt aber noch nicht kümmern. Bestätige die Rechnung einfach
in~$K(\Kom^b(\A))$, verwende aber die zusätzlichen Rechenregeln, dass
quasiisomorphe Komplexe dieselbe Klasse in der K-Theorie haben
und~$L^\bullet[1] = -L^\bullet$ gilt.
\par}
\end{enumerate}
\end{aufgabe}

\end{document}

\begin{aufgabe}{Bewahrung von Injektiven}
Beweise, dass additive Funktoren, die einen linksexakten Linksadjungierten
besitzen, injektive Objekte bewahren.
\end{aufgabe}

Ext^2
