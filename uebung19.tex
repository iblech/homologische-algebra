\documentclass{uebblatt}
\haiitrue

\begin{document}

\maketitle{19}{}

\begin{aufgabe}{Die kanonische Filtrierung eines Komplexes}
Die \emph{gute Abschneidung} eines Komplexes~$K^\bullet$ über einer abelschen
Kategorie~$\A$ ist der Komplex
\[ (\tau_{\leq n} K^\bullet)^i := \begin{cases}
  K^\bullet, & \text{für $i < n$}, \\
  \ker(K^n \to K^{n+1}), & \text{für $i = n$}, \\
  0, & \text{für $i > n$}.
\end{cases} \]

\begin{enumerate}
\item Es gibt auch die \emph{dumme Abschneidung}. Die gute Abschneidung hat ihr
gegenüber den Vorteil, dass~$H^i(\tau_{\leq n} K^\bullet)$ noch für~$i \leq n$
mit~$H^i(K^\bullet)$ übereinstimmt. Beweise diesen Sachverhalt.
\item Bestimme den Kokern der kanonischen Inklusion~$\tau_{\leq n-1}K^\bullet
\hookrightarrow \tau_{\leq n}K^\bullet$.
\item Finde einen Quasiisomorphismus vom Kokern in den im Grad~$n$
konzentrierten Komplex~$H^n(K^\bullet)[-n]$.
\item Folgere: In~$K(D(\Kom^b(\A)))$ gilt die
Rechnung~$K^\bullet = \sum_n (-1)^n\, H^n(K^\bullet)$.

{\scriptsize
Die abgeleitete Kategorie~$D(\Kom^b(\A))$ ist im Allgemeinen nicht abelsch.
Ihre K-Theorie ist daher anders zu definieren: als die von den Objekten
von~$D(\Kom^b(\A))$ erzeugte abelsche Gruppe modulo den Relationen~$X = X' +
X''$ für jedes \emph{ausgezeichnete Dreieck}~$X' \to X \to X'' \to$. Das muss dich
jetzt aber noch nicht kümmern. Bestätige die Rechnung einfach
in~$K(\Kom^b(\A))$, verwende aber die zusätzlichen Rechenregeln, dass
quasiisomorphe Komplexe dieselbe Klasse in der K-Theorie haben
und~$L^\bullet[1] = -L^\bullet$ gilt.
\par}
\end{enumerate}
\end{aufgabe}

\begin{aufgabe}{Komplexe mit vorgegebener Kohomologie}
Sei~$\B$ eine Serresche Unterkategorie einer abelschen Kategorie~$\A$.
Sei~$\Kom_\B(\A)$ die volle Unterkategorie derjenigen Objekte~$K^\bullet$
von~$\Kom(\A)$, deren Kohomologien~$H^n(K^\bullet)$ alle in~$\B$ liegen.
Dann definieren wir~$D_\B(\A) := \Kom_\B(\A)[\text{qis}^{-1}]$.
\begin{enumerate}
\item Zeige, dass~$D_\B(\A)$ auf kanonische Art und Weise eine volle
Unterkategorie von~$D(\A)$ ist.
\item Sei jedes Objekt aus~$\B$ ein Unterobjekt eines Objekts aus~$\B$, welches
als Objekt von~$\A$ injektiv ist. Zeige, dass der kanonische Funktor~$D^+(\B)
\to D^+_\B(\A)$ eine Kategorienäquivalenz ist.
\end{enumerate}
\end{aufgabe}

\begin{aufgabe}{Nullheit von Morphismen}
\begin{enumerate}
\item Zeige: Ein Morphismus~$f$ in~$D(\A)$ ist genau dann Null, wenn ein
Quasiisomorphismus~$s$ existiert, sodass~$sf$ nullhomotop ist.
\item Zeige, dass die Implikationen
\[ \text{$f = 0$ in $\Kom(\A)$} \Longrightarrow
   \text{$f = 0$ in $\K(\A)$} \Longrightarrow
   \text{$f = 0$ in $\D(\A)$} \Longrightarrow
   \text{$H^n(f) = 0$ in $\A$ für alle~$n$}
\]
im Allgemeinen nicht umkehrbar sind.

\emph{Tipp:} Folgt noch.
\end{enumerate}
\end{aufgabe}

Aufgabe zu Ext-Gruppen auf Seite 184.

\end{document}
