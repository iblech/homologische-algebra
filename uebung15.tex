\documentclass{uebblatt}
\haiitrue

\begin{document}

\maketitle{15}{}

Bei Diagrammjagden mit abelschen Gruppen oder Moduln hantiert man mit
\emph{Elementen}: "`Für jedes Element"', "`wegen Exaktheit existiert ein
Urbild"', \ldots{} Bei Objekten beliebiger Kategorien kann man nicht mehr von
Elementen im wörtlichen Sinn sprechen, ein geeigneter Ersatz sind aber
\emph{globale Elemente} -- das sind Morphismen~$1 \to X$ in der Kategorie,
wobei~$1$ das terminale Objekt bezeichnet. Im Fall der Kategorie der Mengen
entsprechen die globalen Elemente gerade den gewöhnlichen (wieso?).

In abelschen Kategorien sind globale Elemente aber stets langweilig: Das
terminale Objekt ist in diesem Fall das Nullobjekt, also besitzt jedes Objekt
genau ein globales Element (wieso?). Für eine sinnvolle Theorie muss man daher
nicht nur die globalen Elemente, sondern \emph{Elemente beliebiger Stufe}
betrachten: Morphismen~$I \to X$, wobei~$I$ ein beliebiges Objekt ist. Solche
Morphismen stellen wir uns als~$I$-indizierte Familien von hypothetischen
gewöhnlichen Elementen von~$X$ vor.

Es gibt aber noch eine weitere Schwierigkeit: Im Allgemeinen lassen
sich~$I$-Elemente nicht längs Epimorphismen~$\pi : Y \twoheadrightarrow X$
liften -- das heißt, dass zu einem~$I$-Element~$x : I \to X$
nicht notwendigerweise ein~$I$-Element~$y : I \to Y$ mit~$x = \pi \circ y$
existiert. (Geht dies ausnahmsweise doch, und zwar für alle Epimorphismen, so
heißt~$I$ \emph{projektiv}.)

Da das bei Diagrammjagden aber ein wichtiger Schritt ist, müssen wir erlauben,
bei Bedarf den Parameterbereich~$I$ zu \emph{verfeinern}, das heißt zu einer
Überdeckung~$J \twoheadrightarrow I$ überzugehen. Dann sind Lifts immer
möglich: Wenn wir~$x$ längs~$\pi$ zurückziehen, also das
Faserproduktdiagramm
\[ \xymatrix{
  J \ar@{->>}[r]^p \ar[d]_y & I \ar[d]^x \\
  Y \ar@{->>}[r]_\pi & X
} \]
betrachten, so ist das resultierende~$J$-Element~$y$ ein Lift
des~$I$-Elements~$x$.

Der tiefere Hintergrund ist folgender: Um ohne Verfeinerungen
Lifts produzieren zu können, müssten wir \emph{auf lineare Art und Weise}
Urbilder auswählen können. Das ist im Allgemeinen nicht möglich (auch nicht
unter Verwendung des Auswahlaxioms). Wenn wir aber Verfeinerungen erlauben,
sammelt der neue Parameterbereich~$J$ alle möglichen Werte für die Wahlen auf,
sodass keine Wahlen mehr getroffen werden müssen. Die Epimorphie von~$p$ drückt
aus, dass die Wahlen jeweils einzeln möglich wären (dass wir also nicht aus
leeren Mengen auswählen müssten).

Die Buchführung über nötige Verfeinerungen kann man in einer neuen Sprache
verbergen und so gewöhnliche elementbasierte Diagrammjagden in beliebigen
abelschen Kategorien interpretieren. Das ist das Ziel der ersten beiden
Aufgaben. Was Aussagen der neuen Sprache tatsächlich bedeuten, ist in Tafel~XXX
definiert.

\newpage

\begin{aufgabe}{Beispiele für die interne Sprache}
\begin{enumerate}
\item Zeige: Ein Morphismus~$f : X \to Y$ in~$\C$ ist genau dann \ldots

\renewcommand{\arraystretch}{1.2}
\begin{tabular}{ll}
  1. der Nullmorphismus, & wenn
  $I \models \forall x\?X\_ f(x) = 0$, \\
  2. ein Monomorphismus, & wenn
  $I \models \forall x\?X\_ f(x) = 0 \Rightarrow x = 0$, \\
  3. ein Epimorphismus, & wenn
  $I \models \forall y\?Y\_ \exists x\?X\_ f(x) = y$.
\end{tabular}

\item Zeige: Ein Sequenz~$X \xra{f} Y \xra{g} Z$ ist genau dann bei~$Y$ exakt,
wenn
\[ I \models \forall x\?X\_ g(f(x)) = 0
  \qquad\text{und}\qquad
  I \models \forall y\?Y\_ g(y) = 0 \Rightarrow \exists x\?X\_ y = f(x). \]

\item Seien~$X$ und~$Y$ Objekte in~$\C$. Sei~$\varphi(x,y)$ eine Formel, in der
zwei Variablen der Typen~$X$ und~$Y$ vorkommen. Gelte
$0 \models \forall x\?X\_ \exists!y\?Y\_ \varphi(x,y)$, ausgeschrieben also
\begin{multline*}0 \models \forall x\?X\_ \exists y\?Y\_ \varphi(x,y)
  \qquad\text{und} \\
  0 \models \forall x\?X\_ \forall y,y'\?Y\_ \varphi(x,y) \wedge \varphi(x,y')
  \Rightarrow y = y'. {\qquad\qquad\qquad} \end{multline*}
Zeige, dass dann genau ein Morphismus~$f : X \to Y$ mit~$0 \models \forall
x\?X\_ \varphi(x,f(x))$ existiert.
\end{enumerate}
\end{aufgabe}

\begin{aufgabe}{Eigenschaften der internen Sprache}
\begin{enumerate}
\item Sei~$\varphi$ eine \emph{Formel über~$I$}, das heißt, dass
in~$\varphi$~$I$-Elemente vorkommen dürfen. Sei~$p : J \to I$ ein beliebiger
Morphismus. Sei~$p^*\varphi$ die Formel über~$J$, die man aus~$\varphi$ erhält,
wenn man alle in~$\varphi$ vorkommenden~$I$-Elemente mit~$p$ vorkomponiert und
so zu~$J$-Elementen macht. Zeige:
\[ I \models \varphi \qquad\Longrightarrow\qquad
  J \models p^*\varphi. \]
\item Sei~$\varphi$ eine \emph{Formel über~$I$}. Sei~$p : J \twoheadrightarrow
I$ ein Epimorphismus. Zeige:
\[ I \models \varphi \qquad\Longleftarrow\qquad
  J \models p^*\varphi. \]
\item Sei~$\varphi$ eine Formel über~$I$. Gelte, dass man mit den in Tafel~XXX
angegebenen Schlussregeln aus~$\varphi$ eine weitere Formel~$\psi$ folgern
kann. Zeige:
\[ I \models \varphi \quad\Longrightarrow\qquad I \models \psi. \]

\emph{Tipp:} \ldots
\end{enumerate}
\end{aufgabe}

\begin{aufgabe}{Lemmas}
Viererlemma, Schlangenlemma, \ldots
\end{aufgabe}

\end{document}
