\documentclass{uebblatt}
\haiitrue

\newcommand{\dd}{{\vee\!\!\!\vee}}

\begin{document}

\maketitle{23}{}

\begin{aufgabe}{Abgeleitetes Dualisieren}
Sei~$A$ ein kommutativer Ring.
\begin{enumerate}
\item Zeige: $D^b(\Mod(A)_\mathrm{coh}) \simeq D^b(\Mod(A))_\mathrm{coh}$.
\item Sei~$(\smallplaceholder)^\dd : D^-(\Mod(A))^\op \to D^+(\Mod(A))$ die
Rechtsableitung des Dualisierungsfunktors. Sei~$M$ ein kohärenter~$A$-Modul.
Zeige: $(M^\dd)^\dd \cong M$.
\end{enumerate}

{\tiny\emph{Hinweis:} Die Kategorie~$\Mod(A)_\mathrm{coh}$ ist die volle
Unterkategorie aller kohärenten Moduln. Ist~$A$ noethersch (was du gerne
voraussetzen darfst), sind die kohärenten Moduln gerade die endlich erzeugten.
Die Kategorie~$D^b(\Mod(A))_\mathrm{coh}$ ist die volle Unterkategorie
derjenigen Komplexe, deren Kohomologiemoduln alle kohärent sind.\par}
\end{aufgabe}

\begin{itemize}
\item Nachrechnen: TR1, TR2, TR3 für $K^\star$ und $D^\star$
\item Auflösungen durch~$F$-azyklische Objekte
\item Kleine Lemmas, die fürs Nachrechnen der Aussagen im Kapitel über
abgeleitete Funktoren wichtig sind
\end{itemize}

\end{document}
