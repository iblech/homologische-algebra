\documentclass{uebblatt}
\haiitrue

\newcommand{\dd}{{\vee\!\!\!\vee}}

\begin{document}

\maketitle{23}{}

\begin{aufgabe}{Abgeleitetes Dualisieren}
Sei~$A$ ein kommutativer Ring, der als Modul über sich selbst kohärent ist.
\begin{enumerate}
\item Zeige: $D^b(\Mod(A)_\mathrm{coh}) \simeq D^b(\Mod(A))_\mathrm{coh}$.
\item Sei~$(\smallplaceholder)^\dd : D^-(\Mod(A))^\op \to D^+(\Mod(A))$ die
Rechtsableitung des Dualisierungsfunktors. Sei~$M$ ein kohärenter~$A$-Modul.
Zeige: $(M^\dd)^\dd \cong M$.
\end{enumerate}

{\tiny\emph{Hinweis:} Die Kategorie~$\Mod(A)_\mathrm{coh}$ ist die volle
Unterkategorie der kohärenten~$A$-Moduln. Unter der Voraussetzung, dass~$A$ als
Modul über sich selbst kohärent ist, ist das die kleinste volle
abelsche Unterkategorie von~$\Mod(A)$, welche alle freien Moduln endlichen
Rangs enthält. Ist~$A$ noethersch (was du gerne
voraussetzen darfst), sind die kohärenten Moduln gerade die endlich erzeugten.
Die Kategorie~$D^b(\Mod(A))_\mathrm{coh}$ ist die volle Unterkategorie
derjenigen Komplexe, deren Kohomologiemoduln alle kohärent sind.\par}
\end{aufgabe}

\begin{aufgabe}{Auflösungen durch azyklische Objekte}
Sei~$F : \A \to \B$ ein linksexakter Funktor. Existiere seine
Rechtsableitung~$RF : D^+(\A) \to D^+(\B)$. Ein Objekt~$U$ heißt genau dann
\emph{$F$-azyklisch}, wenn~$R^{\geq 1}F(U) = 0$.
\begin{enumerate}
\item Sei~$0 \to X \to Y \to Z \to 0$ eine kurze exakte Sequenz. Zeige, dass
wenn~$X$ und~$Y$ oder~$X$ und~$Z$ jeweils~$F$-azyklisch sind, dann auch das
dritte Objekt~$F$-azyklisch ist.
\item Sei~$X^\bullet \in K^+(\A)$ ein azyklischer Komplex aus~$F$-azyklischen
Objekten. Zeige, dass~$F(X^\bullet)$ ebenfalls azyklisch ist.
\item Beweise Lerays Azyklizitätslemma: Ist~$X^\bullet \in D^+(\A)$ ein Komplex, der nur
aus~$F$-azyklischen Objekten besteht, so ist der kanonische
Morphismus~$F(X^\bullet) \to RF(X^\bullet)$ ein Isomorphismus.
\item Folgere: Ist~$0 \to M \to X^\bullet$ eine Auflösung durch~$F$-azyklische
Objekte, so ist~$R^nF(M)$ kanonisch zu~$H^n(F(X^\bullet))$ isomorph.
\end{enumerate}

{\tiny\emph{Tipp:} Teilaufgabe~c) baut nicht direkt auf den ersten beiden
Teilaufgaben auf. Wenn du magst, dann beschränke dich in Teilaufgabe~c) auf den Fall,
dass~$X^\bullet$ in beide Richtungen beschränkt ist. Führe einen
Induktionsbeweis und verwende das ausgezeichnete Dreieck~$\hat\tau_{\geq a+1}
X^\bullet \to X^\bullet \to \hat\tau_{\leq a} X^\bullet \to$, das zwischen
dummen Abschneidungen vermittelt. Teilaufgabe~d) kann schnell aus~c) gefolgert
werden. Es gibt aber auch elementare Beweise, die~c) nicht verwenden.\par}
\end{aufgabe}

\begin{aufgabe}{Triangulierte Kategorien}
Eine \emph{triangulierte Kategorie} ist eine additive Kategorie~$\C$ zusammen
mit einer Autoäquivalenz~$T : \C \to \C$ und einer Klasse \emph{ausgezeichneter
Dreiecke}, die die unten stehenden Axiome erfüllen (wobei wir~"`$X[1]$"' statt~"`$T(X)$"'
schreiben). Sei~$\A$ eine abelsche Kategorie. Beweise, dass~$K^*(\A)$ und~$D^*(\A)$
mit dem Verschiebungsfunktor und den gwöhnlichen ausgezeichneten Dreiecken
triangulierte Kategorien sind. Auf den Nachweis von~TR4 kannst du
verzichten.\par
\begin{scriptsize}
\begin{itemize}
\item[TR1] Für jedes Objekt~$X$ ist $X \xra{\id} X \to 0 \to$
ausgezeichnet.\\[-1.5em]

Für jeden Morphismus~$X \xra{f} Y$ existiert ein Objekt~$Z$ und ein
ausgezeichnetes Dreieck~$X \xra{f} Y \to Z \to$.\\[-1.5em]

Jedes zu einem ausgezeichneten Dreieck isomorphe Dreieck ist selbst
ausgezeichnet.

\item[TR2] Ist~$X \xra{f} Y \xra{g} Z \xra{h}$ ein ausgezeichnetes Dreieck, so
ist auch $Y \xra{g} Z \xra{h} X[1] \xra{-f[1]}$ ausgezeichnet.
% und $Z[-1] \xra{-h[-1]} X \xra{f} Y \xra{g}$, aber das ist redundant.

\item[TR3] Jedes kommutative Diagramm der abgebildeten Form, in dem die beiden Zeilen
ausgezeichnete Dreiecke sind, kann vermöge eines (nicht notwendigerweise
eindeutigen!) gestrichelt eingezeichneten Morphismus zu einem Morphismus von
Dreiecken ergänzt werden.

\item[TR4] Das zu Unrecht gefürchtete Oktaederaxiom.
\marginpar{\scriptsize\vspace*{-0.5cm}\hspace*{-3cm}$
\xymatrix{
  X \ar[r] \ar[d] & Y \ar[r] \ar[d] & Z \ar[r] \ar@{-->}[d] & \\
  \widetilde X \ar[r] & \widetilde Y \ar[r] & \widetilde Z \ar[r] &
}$}
\end{itemize}
\end{scriptsize}

\begin{minipage}{0.8\textwidth}{\tiny\emph{Hinweis:} Die Uneindeutigkeit des Morphismus in~TR3 -- die fehlende
Funktorialität der Kegel -- führt zu gravierenden Problemen. Zum Glück sind die
meisten in der Natur vorkommenden triangulierten Kategorien nur
die~1-kategoriellen Schatten von stabilen~$(\infty,1)$-Kategorien, die diese
Probleme nicht haben.\par}\end{minipage}
\end{aufgabe}

\end{document}
