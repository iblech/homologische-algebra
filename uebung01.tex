\documentclass{uebblatt}

\begin{document}

\maketitle{1}{-- \emph{Motto} --}

\begin{aufgabe}{Kosimpliziale Identitäten}
Die \emph{Korandabbildung}~$\partial^i_n : [n-1] \to
[n]$ ist diejenige eindeutig bestimmte monotone Injektion, die nicht den
Wert~$i$ annimmt. Die \emph{Koentartungsabbildung}~$\sigma^i_n : [n+1] \to [n]$
ist diejenige eindeutig bestimmte monotone Surjektion, für die der Wert~$i$
zwei verschiedene Urbilder besitzt. Beide Definitionen sind für~$n \geq 0$
und~$0 \leq i \leq n$ sinnvoll. Der untere Index wird in der Notation oftmals
unterdrückt.
\begin{enumerate}
\item Zeige:

Jede monotone Injektion~\tabto{4.55cm}$[n] \to [m]$ ist Verkettung
von Korandabbildungen. \\
Jede monotone Surjektion~\tabto{4.55cm}$[n] \to [m]$ ist
Verkettung von Koentartungsabbildungen. \\
Jede monotone Abbildung~\tabto{4.55cm}$[n] \to [m]$
ist Verkettung von Korand- und Koentartungsabbildungen.

\item Verifiziere die folgenden \emph{kosimplizialen Identitäten}.
\begin{align*}
  \partial^j \partial^i &= \partial^i \partial^{j-1} & \text{($i < j$)} \\
  \sigma^j \sigma^i &= \sigma^i \sigma^{j+1} & \text{($i \leq j$)} \\
  \sigma^j \partial^i &= \partial^i \sigma^{j-1} & \text{($i < j$)} \\
  \sigma^j \partial^i &= \id & \text{($i = j$, $i = j+1$)} \\
  \sigma^j \partial^i &= \partial^{i-1} \sigma^j & \text{($i > j + 1$)} 
\end{align*}
Welche anschauliche Bedeutung haben die Identitäten? Kann man
sich Indexschlachten sparen?

\item Seien zwei formale Verkettungsausdrücke von Korand- und
Koentartungsabbildungen gegeben. Gelte, dass beide dieselbe monotone
Abbildung~$[n] \to [m]$ beschreiben. Zeige: Allein unter Verwendung der
kosimplizialen Identitäten kann man die beiden Ausdrücke ineineinander
überführen.

\emph{Tipp:} Zeige, dass jede monotone Abbildung~$f : [n] \to [m]$ auf
\emph{eindeutige} Art und Weise als Verkettung
\[ f = \partial^{i_1} \cdots \partial^{i_s} \sigma^{j_t} \cdots \sigma^{j_1} \]
geschrieben werden kann, wobei in der Notation von links nach rechts die oberen
Indizes der Korandabbildungen streng monoton abnehmen und die der
Koentartungsabbildungen streng monoton zunehmen.
\end{enumerate}

Ähnlich wie Moduln können auch Kategorien \emph{endlich präsentiert} sein.
Diese Aufgabe zeigt, dass die simpliziale Buchhaltungskategorie~$\Delta$ von
den Korand- und Koentartungsabbildungen und nur den kosimplizialen Identitäten als
Relationen erzeugt wird.
\end{aufgabe}

\newpage

\begin{center}-- \emph{Aufgabenvorschläge zusätzlich/statt den Skelettaufgaben}
--\end{center}

\begin{itemize}
\item Kann man diejenige Abbildung, die ein Dreieck auf eine seiner Kanten
projiziert, durch eine Abbildung semisimplizialer Mengen realisieren?
\item Welche semisimpliziale Menge~$S$ hat die Eigenschaft, dass~$X \times S
\cong X$ für alle~$X$?
\item Fixiere deinen Lieblingsraum (zum Beispiel die~$S^1$) und eine lokal
endliche Über\-dec\-kung, sodass deren nichtleere endliche Durchschnitte
zusammenziehbar sind. Zeichne die semisimpliziale Menge zu dem Nerv dieser
Überdeckung. Was passiert, wenn die Überdeckung nicht diese gute Eigenschaft
hat?
\end{itemize}

\end{document}
