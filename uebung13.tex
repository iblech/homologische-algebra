\documentclass{uebblatt}

\newcommand{\dualnumbers}{\ensuremath{\RR[\varepsilon]/(\varepsilon^2)}}

\begin{document}

\maketitle{13}{}

\begin{aufgabe}{Der wandelnde Tangentialvektor}
Sei~$\Delta$ der folgende lokal geringte Raum: Der zugrundeliegende
topologische Raum ist der einpunktige Raum, und der eindeutige Halm der
Strukturgarbe~$\O_\Delta$ ist der Ring~\dualnumbers.
Ferner wird~$\RR^0$ mit der Strukturgarbe~$\O_{\RR^0}$, deren eindeutiger
Halm~$\RR$ ist, zu einem lokal geringten Raum. Es gibt einen kanonischen
Morphismus~$\Delta \to \RR^0$ lokal geringter Räume (welcher?).

\begin{enumerate}
\item Welche Elemente sind in~\dualnumbers{} invertierbar?
\item Sei~$X$ eine reelle glatte Mannigfaltigkeit. Sei~$x_0 \in X$. Zeige:
Lokale~$\RR$-Al\-geb\-ren\-ho\-mo\-mor\-phis\-men~$\O_{X,x_0} \to \dualnumbers$ stehen in
kanonischer Eins-zu-Eins-Korrespondenz mit~$\RR$-linearen Derivationen~$\O_{X,x_0} \to \RR$.

\emph{Hinweis:} Die Ringgarbe~$\O_X$ ist die Garbe der glatten reellwertigen
Funktionen auf~$X$. Ein Ringhomomorphismus ist genau dann lokal, wenn er
Invertierbarkeit reflektiert. Ist~$\varphi : \O_{X,x_0} \to \dualnumbers$ ein
lokaler Ringhomomorphismus, so gilt~$\varphi(f) = f(x_0) + \varepsilon \cdot
D(f)$ für eine von~$f \in \O_{X,x_0}$ abhängige Zahl~$D(f)$ -- wieso?
\item Sei~$X$ eine reelle glatte Mannigfaltigkeit. Dann gibt es einen
kanonischen Morphismus lokal geringer Räume~$X \to \RR^0$. Zeige: Morphismus
lokal geringter Räume~$\Delta \to X$, welche mit den Strukturmorphismen
nach~$\RR^0$ verträglich sind, stehen in kanonischer Eins-zu-Eins-Korrespondenz
zu Tangentialvektoren von~$X$.
\end{enumerate}

Teilaufgabe~c) erklärt, wieso~$\Delta$ (oder besser~$\id : \Delta \to \Delta$)
auch \emph{der wandelnde Tangentialvektor} genannt wird. Der
Ring~$\dualnumbers$ heißt auch \emph{Ring der dualen Zahlen} und ist in der
Numerik bei der Technik der automatischen Differentiation wichtig. Um davon
einen ersten Eindruck zu erhalten, musst du nur für ein Polynom~$p \in \RR[X]$
den Term~$p(x_0 + \varepsilon \cdot v) \in \dualnumbers$ auswerten.
\end{aufgabe}

\end{document}
