\documentclass{uebblatt}

\newcommand{\dualnumbers}{\ensuremath{\RR[\varepsilon]/(\varepsilon^2)}}

\begin{document}

\maketitle{13}{}

\begin{aufgabe}{Der wandelnde Tangentialvektor}
Sei~$\Delta$ der folgende lokal geringte Raum: Der zugrundeliegende
topologische Raum ist der einpunktige Raum, und der eindeutige Halm der
Strukturgarbe~$\O_\Delta$ ist der Ring~\dualnumbers.
Ferner wird~$\RR^0$ mit der Strukturgarbe~$\O_{\RR^0}$, deren eindeutiger
Halm~$\RR$ ist, zu einem lokal geringten Raum. Es gibt einen kanonischen
Morphismus~$\Delta \to \RR^0$ lokal geringter Räume (welchen?).

\begin{enumerate}
\item Welche Elemente in~\dualnumbers{} sind invertierbar?
\item Sei~$X$ eine reelle glatte Mannigfaltigkeit. Sei~$x_0 \in X$. Zeige:
Lokale~$\RR$-Al\-geb\-ren\-ho\-mo\-mor\-phis\-men~$\O_{X,x_0} \to \dualnumbers$ stehen in
kanonischer Eins-zu-Eins-Korrespondenz mit~$\RR$-linearen Derivationen~$\O_{X,x_0} \to \RR$.
(Solche müssen folgende Leibnizregel erfüllen: $D(fg) = f(x_0) D(g) + g(x_0)
D(f)$.)

\emph{Hinweis:} Die Ringgarbe~$\O_X$ ist die Garbe der glatten reellwertigen
Funktionen auf~$X$. Ein Ringhomomorphismus ist genau dann lokal, wenn er
Invertierbarkeit reflektiert. Ist~$\varphi : \O_{X,x_0} \to \dualnumbers$ ein
lokaler Ringhomomorphismus, so gilt~$\varphi(f) = f(x_0) + \varepsilon \cdot
D(f)$ für eine von~$f \in \O_{X,x_0}$ abhängige Zahl~$D(f)$ -- wieso?

\item Sei~$X$ eine reelle glatte Mannigfaltigkeit. Dann gibt es einen
kanonischen Morphismus lokal geringter Räume~$X \to \RR^0$. Zeige: Morphismen
lokal geringter Räume~$\Delta \to X$, welche mit den Strukturmorphismen
nach~$\RR^0$ verträglich sind, stehen in kanonischer Eins-zu-Eins-Korrespondenz
zu Tangentialvektoren von~$X$.
\end{enumerate}

Teilaufgabe~c) erklärt, wieso~$\Delta$ (oder besser~$\id : \Delta \to \Delta$)
auch \emph{der wandelnde Tangentialvektor} genannt wird. Der
Ring~$\dualnumbers$ heißt auch \emph{Ring der dualen Zahlen} und ist in der
Numerik bei der Technik der automatischen Differentiation wichtig. Um davon
einen ersten Eindruck zu erhalten, musst du nur für ein Polynom~$p \in \RR[X]$
den Term~$p(x_0 + \varepsilon \cdot v) \in \dualnumbers$ auswerten.
\end{aufgabe}

\begin{aufgabe}{Funktionsinterpretation}
Ist~$f \in \O_{X,x}$ ein Keim der Strukturgarbe~$\O_X$ eines lokal geringten
Raums~$X$, so sieht man~$[f] \in \O_{X,x}/\mmm_x$ als "`Funktionswert von~$f$ an
der Stelle~$x$"' an. Der Körper, in dem~$f$ Funktionswerte in diesem Sinn
annimmt, kann also anders als bei gewöhnlichen reellwertigen Funktionen von
Punkt zu Punkt variieren.
\begin{enumerate}
\item Sei~$X$ eine glatte reelle Mannigfaltigkeit. Beweise, dass die
Faktorringe~$\O_{X,x}/\mmm_x$ kanonisch isomorph zu~$\RR$ sind und dass die
angegebene Vorstellung von Funktionswerten mit der bekannten und hier
anwendbaren Definition übereinstimmt.
\item Sei~$X = \Spec \ZZ$. Wo hat die "`Funktion"' $45 \in \Gamma(X,\O_X)$
Nullstellen? Wo liegen vermutlich doppelte Nullstellen?

\emph{Tipp:} Zur Definition von~$\Spec\ZZ$ siehe Übungsblatt 5. In
Übungsblatt~6 wurden die Halme von~$\O_{\Spec\ZZ}$ berechnet.
\item Finde einen lokal geringten Raum~$X$ und einen globalen Schnitt~$f \in
\Gamma(X,\O_X)$, der nicht Null ist, aber dessen Funktionswerte an jedem Punkt
verschwinden.
\end{enumerate}
\end{aufgabe}

\begin{aufgabe}{Der terminale lokal geringte Raum}
Zeige, dass der (überhaupt nicht einpunktige!) Raum~$\Spec \ZZ$ der terminale
lokal geringte Raum ist.

\emph{Hinweis:} Bekanntlich ist der Ring~$\ZZ$ initial in der Kategorie der
Ringe. Damit folgt sofort, dass~$\Spec\ZZ$ terminal in der Kategorie der
affinen Schemata ist -- denn diese ist äquivalent zur dualen Kategorie von der
der Ringe. Hier geht es aber darum, Terminalität in der größeren Kategorie
aller lokal geringten Räume nachzuweisen. Wissen über Schemata ist hierfür
nicht nötig.

\emph{Bemerkung:} Die étale Kohomologie von~$\Spec\ZZ$ stellt eine Verbindung
zwischen Primzahlen und der Verschlingungszahl von Knoten her.
\url{http://math.ucr.edu/home/baez/week257.html}
\end{aufgabe}

\begin{aufgabe}{Schnitte als Faserprodukte}
\begin{enumerate}
\item Seien~$U$ und~$V$ Teilmengen einer Menge~$X$. Was ist das Faserprodukt~$U
\times_X V$ in der Kategorie der Mengen?
\item Seien~$U$ und~$V$ offene Teilmengen eines topologischen Raums~$X$. Was
ist das Faserprodukt~$U \times_X V$ in der Kategorie der offenen Mengen
von~$X$?
\end{enumerate}
\end{aufgabe}

\begin{aufgabe}{Kategorielle Vollständigkeit}
\begin{enumerate}
\item Zeige, dass die Kategorie der Mengen \emph{vollständig} ist: Jedes Diagramm
mit kleiner Indexkategorie besitzt einen Limes.
\item Errate die duale Aufgabe und löse sie.
\end{enumerate}
\end{aufgabe}

\begin{aufgabe}{Freie Körper?}
\begin{enumerate}
\item Zeige: Die Kategorie der Körper besitzt kein initiales Objekt.
\item Folgere: Der Vergissfunktor von der Kategorie der Körper in die Kategorie
der Mengen besitzt keinen Linksadjungierten. Freie Körper gibt es also nicht.
\end{enumerate}
\end{aufgabe}

\end{document}
