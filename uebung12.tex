\documentclass{uebblatt}
\usepackage{multicol}

\begin{document}

\maketitle{12}{}

\begin{aufgabe}{Freie Konstruktionen}
Sei~$\C$ eine Kategorie algebraischer Strukturen (etwa~$\C = \mathrm{Ring}$
oder~$\C = \mathrm{Mod}(R)$). Anschaulich stellt man sich das von den Elementen
einer gewissen Menge~$M$ \emph{frei erzeugte Objekt}~$L(M)$ in~$\C$ wie folgt
vor: Man beginnt mit den Elementen aus~$M$ und fügt all solche Ausdrücke hinzu,
dass~$L(M)$ zu einem Objekt von~$\C$ wird (etwa Summen und Produkte im Fall~$\C
= \mathrm{Ring}$). Dabei nimmt man nur solche Identifikationen vor, die von den
Axiomen gefordert werden (etwa das Assoziativgesetz). Die Zuordnung~$M \mapsto
L(M)$ definiert dann einen Funktor~$\Set \to \C$, welcher \emph{linksadjungiert
zum Vergissfunktor} $V : \C \to \Set$ ist.

\begin{enumerate}
\item Erkläre, inwieweit die Adjunktionsbeziehung~$L \dashv V$ die anschauliche
Vorstellung kodiert. (Diese Frage hat eine präzise Antwort.)
\item Bestimme für die folgenden Vergissfunktoren Linksadjungierte.
\begin{multicols}{2}
\renewcommand{\theenumii}{\arabic{enumii}}
\begin{enumerate}
\item $\mathrm{Mod}(R) \to \Set$
\item $\mathrm{Mon} \to \Set$
\item $\mathrm{Grp} \to \Set$
\item $\mathrm{Ring} \to \Set$
\item $\mathrm{Top} \to \Set$
\item $\mathrm{sSet} \to \Set$
\item $\mathrm{Ab} \to \mathrm{Grp}$
\item $\mathrm{Mod}(R) \to \mathrm{Ab}$
\item $\mathrm{Alg}(R) \to \mathrm{Mod}(R)$
\item $\mathrm{Met}_\mathrm{complete} \to \mathrm{Met}$
\item $\mathrm{Alg}(k) \to \mathrm{LieAlg}(k)$
\item $\mathrm{sSet} \to \text{semi-sSet}$
\end{enumerate}
\end{multicols}
Dabei ist~$\mathrm{Mon}$ die Kategorie der Monoide,~$\mathrm{Alg}(R)$ die Kategorie
der~$R$-Algebren,~$\mathrm{Met}_\mathrm{complete}$ die Kategorie der
vollständigen metrischen
Räume und gleichmäßig stetigen Abbildungen und~$\text{semi-sSet}$ die
Kategorie der Verklebedaten. Findest du zum Vergissfunktor~$\mathrm{Top} \to
\Set$ auch einen \emph{Rechts}adjungierten?
\end{enumerate}
\end{aufgabe}

Noch zu \TeX{}en:

\begin{verbatim}Berechnung vom Cartier-Dual der affinen Gruppe über k der n-ten
Einheitswurzeln: \Spec k[x]/(x^n - 1), wobei k ein beliebiger (kommutativer)
Grundring ist.

II.3.20: Beweise die universelle Eigenschaft des im Beweis konstruierten
Objektes X.

II.3.24: Zeige (II.16), nämlich, daß F -> FGF -> F und G -> GFG -> G
Identitäten sind.

Adjunktionsverhältnisse zwischen Aufrundung und Abrundung

Ansonsten noch Aufgabe 2., 3. und 8.
Aufgabe 10. ist außerdem ein nettes konkretes Beispiel.
\end{verbatim}

\end{document}
