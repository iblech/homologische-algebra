\documentclass{uebblatt}
\usepackage{multicol}

\begin{document}

\maketitle{12}{}

\begin{aufgabe}{Limiten zyklischer Gruppen}
Sei~$\NN$ mit der Teilbarkeitsordnung versehen. Zeige, dass
der Kolimes~$\colim_{n \in \NN} \ZZ/(n)$ die Faktorgruppe~$\QQ/\ZZ$ ist. Die
Diagrammabbildungen~$\ZZ/(n) \to \ZZ/(m)$ sind dabei durch (in~$\ZZ$ zu
berechnende) Multiplikation mit~$m/n$ gegeben.
\end{aufgabe}

\begin{aufgabe}{Existenzkriterium für endliche Limiten}
Sei~$\C$ eine Kategorie, in der es folgende spezielle endliche Limiten gibt:
ein terminales Objekt, Produkte von je zwei Objekten und Differenzkerne
(Equalizer) von je zwei parallelen Morphismen. Zeige, dass es in~$\C$
dann schon alle endlichen Limiten gibt.

\emph{Hinweis:} Es ist nur noch die universelle Eigenschaft des im Beweis der
Vorlesung konstruierten Objekts~$X$ nachzuweisen.
\end{aufgabe}

\begin{aufgabe}{Freie Konstruktionen}
Sei~$\C$ eine Kategorie algebraischer Strukturen (etwa~$\C = \mathrm{Ring}$
oder~$\C = \mathrm{Mod}(R)$). Anschaulich stellt man sich das von den Elementen
einer gewissen Menge~$M$ \emph{frei erzeugte Objekt}~$L(M)$ in~$\C$ wie folgt
vor: Man beginnt mit den Elementen aus~$M$ und fügt all solche Ausdrücke hinzu,
damit~$L(M)$ zu einem Objekt von~$\C$ wird (etwa Summen und Produkte im Fall~$\C
= \mathrm{Ring}$). Dabei nimmt man nur solche Identifikationen vor, die von den
Axiomen gefordert werden (etwa dem Assoziativgesetz). Die Zuordnung~$M \mapsto
L(M)$ definiert dann einen Funktor~$\Set \to \C$, welcher \emph{linksadjungiert
zum Vergissfunktor} $V : \C \to \Set$ ist.

\begin{enumerate}
\item Erkläre, inwieweit die Adjunktionsbeziehung~$L \dashv V$ obige anschauliche
Vorstellung kodiert. (Diese Frage hat eine präzise Antwort.)
\item Bestimme für die folgenden Vergissfunktoren Linksadjungierte.
\begin{multicols}{2}
\renewcommand{\theenumii}{\arabic{enumii}}
\begin{enumerate}
\item $\mathrm{Mod}(R) \to \Set$
\item $\mathrm{Mon} \to \Set$
\item $\mathrm{Grp} \to \Set$
\item $\mathrm{Ring} \to \Set$
\item $\mathrm{Top} \to \Set$
\item $\mathrm{sSet} \to \Set$
\item $\mathrm{Ab} \to \mathrm{Grp}$
\item $\mathrm{Mod}(R) \to \mathrm{Ab}$
\item $\mathrm{Alg}(R) \to \mathrm{Mod}(R)$
\item $\mathrm{Met}_\mathrm{complete} \to \mathrm{Met}$
\item $\mathrm{Alg}(k) \to \mathrm{LieAlg}(k)$
\item $\mathrm{sSet} \to \text{semi-sSet}$
\end{enumerate}
\end{multicols}
Dabei ist~$\mathrm{Mon}$ die Kategorie der Monoide,~$\mathrm{Alg}(R)$ die Kategorie
der~$R$-Algebren, $\mathrm{Met}_\mathrm{complete}$ die Kategorie der
vollständigen metrischen
Räume und gleichmäßig stetigen Abbildungen und~$\text{semi-sSet}$ die
Kategorie der Verklebedaten. Findest du zum Vergissfunktor~$\mathrm{Top} \to
\Set$ auch einen \emph{Rechts}adjungierten?
\end{enumerate}
\end{aufgabe}

\newpage
\begin{aufgabe}{Globale Charakterisierung von Limes und Kolimes}
Sei~$I$ eine Indexkategorie. Sei~$\C$ eine Kategorie, in der alle~$\I$-förmigen
Limiten existieren. Der \emph{Diagonalfunktor}~$\Delta : \C \to
[\I,\C]$ schickt ein Objekt~$X$ auf den konstanten Funktor bei~$X$ (dieser
schickt jedes Objekt auf~$X$ und jeden Morphismus auf~$\id_X$). Dabei
ist~$[\I,\C]$ die Kategorie der Funktoren~$\I \to \C$.
\begin{enumerate}
\item Sei für jedes Diagramm~$F \in [\I,\C]$ ein Limes~$\lim F \in \C$ gewählt.
Definiere damit einen Funktor~$\lim : [\I,\C] \to \C$, der jedem Diagramm
seinen Limes zuordnet. Wie wirkt er auf Morphismen? Wieso sind die
Funktoraxiome erfüllt?
\item Zeige: $\Delta \dashv \lim$.
\item Formuliere und beweise mit wenig Aufwand die duale Behauptung
über Kolimiten.
\end{enumerate}
\end{aufgabe}

\begin{aufgabe}{Aufrundung und Abrundung}
Wir betrachten die drei monotonen Abbildungen
\[ \begin{array}{@{}rrclrcl@{}}
  \lceil\smallplaceholder\rceil: & \QQ &\longrightarrow& \ZZ, &
  x &\longmapsto& \text{Aufrundung von~$x$ := (kleinste ganze Zahl~$\geq x$)} \\
  i: & \ZZ &\longrightarrow& \QQ, &
  z &\longmapsto& z \\
  \lfloor\smallplaceholder\rfloor: & \QQ &\longrightarrow& \ZZ, &
  x &\longmapsto& \text{Abrundung von~$x$ := (größte ganze Zahl~$\leq x$)}
\end{array} \]
und die induzierten Funktoren zwischen~$B\QQ$ und~$B\ZZ$.
Zeige:
$B\lceil\smallplaceholder\rceil \dashv Bi \dashv B\lfloor\smallplaceholder\rfloor$.
\end{aufgabe}

\begin{aufgabe}{Allquantifikation, Rückzug und Existenzquantifikation}
Eine Abbildung~$f : X \to Y$ zwischen Mengen induziert in kanonischer Art und
Weise drei monotone Abbildungen zwischen den Potenzmengen.
Zeige: $B\exists_f \dashv Bf^{-1} \dashv B\forall_f$.
\[ \begin{array}{@{}rr@{\ }c@{\ }lr@{\ }c@{\ }l@{}}
  \exists_f : & \P(X) &\longrightarrow& \P(Y), &
    U &\longmapsto& \exists_f(U) := f[U] = \{ y \in Y \ |\ \exists x \in
    X{:}\ y = f(x) \wedge x \in U \} \\[0.5em]
  f^{-1} : & \P(Y) &\longrightarrow& \P(X), &
    V &\longmapsto& f^{-1}[V] \\[0.5em]
  \forall_f : & \P(X) &\longrightarrow& \P(Y), &
    W &\longmapsto& \forall_f(W) := \{ y \in Y \ |\ \forall x \in X{:}\
      y = f(x) \Rightarrow x \in W \}
\end{array} \]
\vspace{-1em}
\end{aufgabe}

\begin{aufgabe}{Lineares Currying}
Seien~$R$ und~$S$ Ringe (mit Eins).
Sei~$M$ ein~$R$-$S$-Bimodul. Zeige:
\[ \left(\begin{array}{@{}rcl@{}}
  \lMod{S} &\longrightarrow& \lMod{R} \\
  V &\longmapsto& M \otimes_S V
\end{array}\right)\ \text{\huge$\dashv$}\ \left(\begin{array}{@{}rcl@{}}
  \lMod{R} &\longrightarrow& \lMod{S} \\
  W &\longmapsto& \Hom_R(M,W)
\end{array}\right) \]
\vspace{-1em}
\end{aufgabe}

\begin{aufgabe}{Die Dreiecksidentitäten von Adjunktionen}
Sei~$F \dashv G$ ein Paar adjungierter Funktoren. Zeige, dass zwischen
der \emph{Eins} $\eta : \Id \to G \circ F$ und der \emph{Koeins} $\varepsilon :
F \circ G \to \Id$ folgende Beziehungen bestehen:
\[
  \xymatrix{
    G \ar[rd]_{\eta_{G(\smallplaceholder)}} \ar[rr]^{\id} &&
    G \\
    & GFG \ar[ru]_{G \varepsilon}
  }
  \qquad
  \xymatrix{
    F \ar[rd]_{F \eta} \ar[rr]^{\id} &&
    F \\
    & FGF \ar[ru]_{\varepsilon_{F(\smallplaceholder)}}
  }
\]
\vspace{-1em}
\end{aufgabe}

\begin{aufgabe}{Ein Beispiel für Cartier-Dualität}
Sei~$k$ ein kommutativer Grundring und~$A = k[X]/(X^n - 1)$. Anschaulich
ist~$A$ die Algebra der Funktionen auf dem Unterschema der~$n$-ten
Einheitswurzeln in~$\AA^1_k$ (unabhängig davon, ob diese in~$k$ tatsächlich
existieren oder nicht).
\begin{enumerate}
\item Zeige, dass~$A$ mit der Komultiplikation~$[X] \mapsto [X] \otimes [X]$ zu
einer Bialgebra wird.
\item Welche Bialgebra ist das Cartier-Duale~$A^\vee = \Hom_k(A,k)$?
\end{enumerate}
\end{aufgabe}

\end{document}
