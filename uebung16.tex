\documentclass{uebblatt}
\haiitrue

\begin{document}

\maketitle{16}{}

\begin{aufgabe}{Universelle Eigenschaft der Garbifizierung}
Seien~$\F$ und~$\G$ Prägarben auf einem topologischen Raum~$X$ (oder einer
Örtlichkeit). Sei~$\alpha : \F \to \G$ ein Morphismus von Prägarben. Sei~$\G$
sogar eine Garbe. Sei~$\F \xra{\iota} s(\F)$ die Garbifizierung von~$\F$.
Konstruiere einen Garbenmorphismus~$\overline{\alpha} : s(\F) \to \G$
mit~$\overline{\alpha} \circ \iota = \alpha$ und weise insbesondere seine
Wohldefiniertheit nach.
\end{aufgabe}

\begin{aufgabe}{Halme des Pushforwards}
\begin{enumerate}
\item Sei~$X$ ein topologischer Raum. Sei~$f : Y \hookrightarrow X$ die
Inklusion eines abgeschlossenen Teilraums. Sei~$\E$ eine Garbe auf~$Y$. Zeige:
\[ (f_* \E)_x \cong \begin{cases}
  \E_x, & \text{falls $x \in Y$,} \\
  \{0\}, & \text{falls $x \not\in Y$.}
\end{cases} \]
\item Mache dir anhand eines Beispiels klar, dass die analoge Aussage für
Inklusionen offener Teilräume im Allgemeinen nicht gilt.
\item Folgere, dass der Pushforward-Funktor~$f_* : \mathrm{AbShv}(Y) \to
\mathrm{AbShv}(X)$ in der Situation von Teilaufgabe~a) exakt ist.
\item Sei~$f : Y \to X$ eine abgeschlossene stetige Abbildung. Sei~$\E$ eine
Garbe auf~$Y$. Sei~$x \in X$. Zeige: $(f_* \E)_x \cong \Gamma(f^{-1}[x], \E)$.

\emph{Hinweis:} Beachte, dass die Faser~$f^{-1}[x]$ im Allgemeinen nicht offen
sein wird. Die rechte Seite ist daher als Kolimes über die~$\E(U)$, wobei~$U
\subseteq Y$ alle offenen Mengen mit~$f^{-1}[x] \subseteq U$ durchläuft,
definiert.

\emph{Tipp:} Eine stetige Abbildung~$f : Y \to X$ ist genau dann abgeschlossen,
wenn für alle~$x \in X$ und alle offenen Umgebungen~$U$ von~$f^{-1}[x]$ in~$Y$
eine offene Umgebung~$V$ von~$x$ mit~$f^{-1}[V] \subseteq U$ existiert.
\end{enumerate}
\end{aufgabe}

\end{document}

Für später: j^! usw.
