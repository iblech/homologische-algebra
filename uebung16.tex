\documentclass{uebblatt}
\haiitrue

\begin{document}

\maketitle{16}{}

\begin{aufgabe}{Universelle Eigenschaft der Garbifizierung}
Seien~$\F$ und~$\G$ Prägarben auf einem topologischen Raum~$X$ (oder einer
Örtlichkeit). Sei~$\alpha : \F \to \G$ ein Morphismus von Prägarben. Sei~$\G$
sogar eine Garbe. Sei~$\F \xra{\iota} s(\F)$ die Garbifizierung von~$\F$.
Konstruiere einen Garbenmorphismus~$\overline{\alpha} : s(\F) \to \G$
mit~$\overline{\alpha} \circ \iota = \alpha$ und weise insbesondere seine
Wohldefiniertheit nach.
\end{aufgabe}

\begin{aufgabe}{Halme des Pushforwards}
\begin{enumerate}
\item Sei~$X$ ein topologischer Raum. Sei~$f : Y \hookrightarrow X$ die
Inklusion eines abgeschlossenen Teilraums. Sei~$\E$ eine Garbe auf~$Y$. Zeige:
\[ (f_* \E)_x \cong \begin{cases}
  \E_x, & \text{falls $x \in Y$,} \\
  \{0\}, & \text{falls $x \not\in Y$.}
\end{cases} \]
\item Mache dir anhand eines Beispiels klar, dass die analoge Aussage für
Inklusionen offener Teilräume im Allgemeinen nicht gilt.
\item Folgere, dass der Pushforward-Funktor~$f_* : \mathrm{AbShv}(Y) \to
\mathrm{AbShv}(X)$ in der Situation von Teilaufgabe~a) exakt ist.
\item Sei~$f : Y \to X$ eine abgeschlossene stetige Abbildung. Sei~$\E$ eine
Garbe auf~$Y$. Sei~$x \in X$. Zeige: $(f_* \E)_x \cong \Gamma(f^{-1}[x], \E)$.

\emph{Hinweis:} Beachte, dass die Faser~$f^{-1}[x]$ im Allgemeinen nicht offen
sein wird. Die rechte Seite ist daher als Kolimes über die~$\E(U)$, wobei~$U
\subseteq Y$ alle offenen Mengen mit~$f^{-1}[x] \subseteq U$ durchläuft,
definiert.

\emph{Tipp:} Eine stetige Abbildung~$f : Y \to X$ ist genau dann abgeschlossen,
wenn für alle~$x \in X$ und alle offenen Umgebungen~$U$ von~$f^{-1}[x]$ in~$Y$
eine offene Umgebung~$V$ von~$x$ mit~$f^{-1}[V] \subseteq U$ existiert.
(Siehe zum Beispiel
\href{https://www2.math.uni-paderborn.de/fileadmin/Mathematik/People/wedhorn/Lehre/SkriptMannigfaltigkeiten.pdf#page=86}{Torsten
Wedhorn, \emph{Manifolds, sheaves, and cohomology}, Seite 86}.)
\end{enumerate}
\end{aufgabe}

\begin{aufgabe}{Beispiele für flache Moduln}
\begin{enumerate}
\item Zeige: Jeder freie Modul ist flach.
\item Zeige: Direkte Summanden freier Moduln sind flach.
\item Zeige: Filtrierte Kolimiten flacher Moduln (also Moduln der
Form~$\colim_{i \in I} M_i$, wobei~$I$ eine filtrierte Kategorie ist) sind flach.

\emph{Erinnerung:} Eine Kategorie heißt genau dann \emph{filtriert}, wenn in
ihr jedes endliche Diagram einen Kokegel besitzt (welcher nicht unbedingt eine
universelle Eigenschaft erfüllen muss). Wenn du magst, kannst du der
Einfachheit halber gerne annehmen, dass~$I$ die von einer gerichteten Menge
induzierte Kategorie ist.
\end{enumerate}
\emph{Hinweis:} In der Übung werden wir diskutieren, wie man sich Flachheit von
Moduln geometrisch vorstellen kann.
\end{aufgabe}

\newpage

\begin{aufgabe}{Serresche Quotientenkategorien}
Sei~$\A$ eine abelsche Kategorie. Sei~$\B$ eine \emph{Serresche Unterkategorie}
von~$\A$, das ist eine volle Unterkategorie, welche das Nullobjekt von~$\A$
enthält und für die für jede kurze exakte Sequenz~$0 \to X' \to X \to X'' \to 0$
in~$\A$ folgendes gilt: $X$ liegt genau dann in~$\B$, wenn~$X'$ und~$X''$
in~$\B$ liegen.
\begin{enumerate}
\item Zeige: Ist~$X$ ein Objekt von~$\B$, so liegt auch jedes in~$\A$ zu~$X$ isomorphe
Objekt in~$\B$.
\item Mache dir kurz klar: Die Kategorie der endlich-dimensionalen
Vektorräume ist eine Serresche Unterkategorie der Kategorie aller Vektorräume.
\item Die \emph{Serresche Quotientenkategorie}~$\A/\B$ hat als Objekte
dieselben wie~$\A$. Die Morphismen definiert man über den (gerichteten) Kolimes
\[ \Hom_{\A/\B}(X,Y) \defeq \colim_{X',Y'} \Hom_\A(X', Y/Y'), \]
wobei~$X'$ über alle Unterobjekte von~$X$ mit~$X/X' \in \B$ und~$Y'$ über alle
Unterobjekte von~$Y$ mit~$Y' \in \B$ läuft. Wie ist die Morphismenverkettung
in~$\A/\B$ zu definieren? Wieso ist der Kolimes gerichtet? Wie wird~$\A/\B$ zu einer abelschen Kategorie?
Welchen Funktor~$F : \A \to \A/\B$ kann man kanonisch angeben? Wieso ist dieser
exakt? Wieso gilt genau dann~$F(X) = 0$, wenn~$X \in \B$? Wieso ist~$F : \A \to
\A/\B$ unter allen exakten Funktoren~$G : \A \to \C$ mit~$G(X) = 0$ für~$X \in \B$
initial? Kläre so viele dieser Fragen, wie du möchtest.
\end{enumerate}
\emph{Bemerkung:} Serresche Quotientenkategorien sind zur algorithmischen
Implementierung von Kategorien kohärenter Modulgarben auf gewissen Schemata
nützlich (siehe Artikel von Mohamed Barakat und anderen).
\end{aufgabe}

\begin{aufgabe}{Der Satz von Jordan--Hölder}
Ein Objekt~$X$ einer abelschen Kategorie heißt genau dann \emph{einfach}, wenn
es \emph{genau zwei} Unterobjekte besitzt. (Ein \emph{Unterobjekt} ist ein
Monomorphismus~$U \xra{i} X$. Unterobjekte~$U \xra{i} X$, $U' \xra{i'} X$
werden genau dann als gleich angesehen, wenn es einen Isomorphismus~$q : U \to
U'$ mit~$i' \circ q = i$ gibt.) Das Nullobjekt zählt also nicht als einfach.

Eine \emph{Jordan--Hölder-Reihe} für ein Objekt~$X$ ist eine Filtrierung
$0 = X_0 \hookrightarrow X_1 \hookrightarrow \cdots \hookrightarrow X_{n-1}
\hookrightarrow X_n = X$, sodass die Quotienten~$X_i/X_{i-1}$ jeweils einfache
Objekte sind.
\begin{enumerate}
\item Zeige: Je zwei Jordan--Hölder-Reihen eines Objekts~$X$ haben dieselbe
Länge und bis auf Isomorphie treten dieselben Quotienten auf.

\emph{Tipp:} Lasse dich vom klassischen Beweis des Satzes über
Schreier--Zassenhaus, zum Beispiel für Gruppen oder Moduln, inspirieren.
\item Sei~$A$ eine~$(n \times n)$-Matrix über einem algebraisch abgeschlossenen
Körper~$K$. Der Vektorraum~$K^n$ wird durch die Setzung~$f(X) \cdot v \defeq f(A)v$
für~$f \in K[X]$ und~$v \in K^n$ zu einem~$K[X]$-Modul. Hängen die
Jordanform von~$A$ und Jordan--Hölder-Reihen von~$K^n$ als~$K[X]$-Modul
miteinander zusammen?
\end{enumerate}
\end{aufgabe}

\end{document}

Für später: j^! usw.
