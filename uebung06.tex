\documentclass{uebblatt}

\begin{document}

\maketitle{6}{-- \emph{Garben oganisieren lokale Daten.} --}

\begin{aufgabe}{Beispiele für Garben}
Welche der folgenden Prägarben auf einem topologischen Raum~$X$ sind Garben?
\begin{enumerate}
\item $\C : U \mapsto \{ f : U \to \RR \,|\, \text{$f$ stetig} \}$
\item $\C_\text{const.} : U \mapsto \{ f : U \to \RR \,|\, \text{$f$ konstant} \}$
\item $\C_\text{l.c.} : U \mapsto \{ f : U \to \RR \,|\, \text{$f$ lokal konstant} \}$
\item $\C_\text{bounded} : U \mapsto \{ f : U \to \RR \,|\, \text{$f$ beschränkt} \}$
\end{enumerate}
\end{aufgabe}

\begin{aufgabe}{Mono- und Epimorphismen zwischen Garben}
Ein Morphismus~$f$ einer gewissen Klasse von Morphismen heißt genau dann
\emph{Monomorphismus}, wenn er bezüglich der Verkettung von Morphismen
linkskürzbar ist, d.\,h. wenn für beliebige parallele Morphismen~$p$ und~$q$
aus~$f \circ p = f \circ q$ schon~$p = q$ folgt. Dual ist
ein \emph{Epimorphismus} ein rechtskürzbarer Morphismus. Zeige:
\begin{enumerate}
\item Unter allen Abbildungen von Mengen sind die Monomorphismen gerade die
injektiven Abbildungen.
\item Unter allen Abbildungen von Mengen sind die Epimorphismen gerade die
surjektiven Abbildungen.
\item Unter allen Morphismen von Garben auf einem festen topologischen Raum~$X$
sind die Monomorphismen gerade
diejenigen Garbenmorphismen~$\alpha : \F \to \G$, deren
Komponentenabbildungen~$\alpha_U$ für alle offenen Teilmengen~$U \subseteq X$
injektiv sind. (Das ist genau dann der Fall, wenn die Abbildungen~$\alpha_x :
\F_x \to \G_x$ zwischen den Halmen alle injektiv sind.)
\item Unter allen Morphismen von Garben auf einem festen topologischen Raum~$X$
sind die Epimorphismen gerade
diejenigen Garbenmorphismen~$\alpha : \F \to \G$, deren
Halmabbildungen~$\alpha_x : \F_x \to \G_x$ alle surjektiv sind. (Obacht: Das
ist echt schwächer als zu sagen, dass alle Komponentenabbildungen~$\alpha_U$
surjektiv sind.)
\end{enumerate}
\end{aufgabe}

\end{document}
