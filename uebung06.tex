\documentclass{uebblatt}
\usepackage{calc}

\begin{document}

\maketitle{6}{-- \emph{Garben organisieren lokale Daten.} --}

\begin{aufgabe}{Beispiele für Garben}
Sei~$X$ ein topologischer Raum. Das fundamentale Beispiel für eine Garbe
auf~$X$ ist durch die Zuordnung
\begin{enumerate}
\item $\E : U \longmapsto \{ s : U \to E \,|\, \text{$s$ stetig mit $\pi \circ f = \id|_U$} \}$
\end{enumerate}
und den gewöhnlichen Einschränkungsabbildungen gegeben. Dabei ist~$\pi : E \to
X$ eine feste stetige Abbildung. Zeige, dass~$\E$ tatsächlich eine Garbe ist.

Welche der folgenden Setzungen mit den gewöhnlichen Einschränkungsabbildungen
führen zu Prägarben? Welche sogar zu Garben?
\newlength{\sheafname}
\settowidth{\sheafname}{$\C_\text{bounded}$}
\begin{enumerate}
\addtocounter{enumi}{1}
\item \parbox{\sheafname}{$\C$} $: U \longmapsto \{ f : U \to \RR \ |\  \text{$f$ stetig} \}$
\item \parbox{\sheafname}{$\C_\text{const.}$} $ : U \longmapsto \{ f : U \to \RR \ |\  \text{$f$ konstant} \}$
\item \parbox{\sheafname}{$\C_\text{l.c.}$} $ : U \longmapsto \{ f : U \to \RR \ |\  \text{$f$ lokal konstant} \}$
\item \parbox{\sheafname}{$\C_\text{0}$} $ : U \longmapsto \{ f : U \to \RR \ |\
\text{$f$ hat kompakten Träger in~$U$} \}$
\item \parbox{\sheafname}{$\C_\text{bounded}$} $ : U \longmapsto \{ f : U \to \RR \ |\  \text{$f$ beschränkt} \}$
\end{enumerate}
\end{aufgabe}

\begin{aufgabe}{Exakte Sequenzen von Garben}
Eine Sequenz~$\F \xra{\alpha} \G \xra{\beta} \H$ von Garben abelscher Gruppen
auf einem topologischen Raum~$X$ heißt genau dann \emph{exakt} (bei~$\G$), wenn für jede
offene Teilmenge~$U \subseteq X$ folgende Bedingungen erfüllt sind:
\begin{itemize}
\item "`$\Image \subseteq \Kernel$"'. Für jeden Schnitt~$s \in \Gamma(U,\F)$
gilt~$\beta_U(\alpha_U(s)) = 0 \in \Gamma(U,\H)$.
\item "`$\Image \supseteq \Kernel$"'. Für jeden Schnitt~$t \in \Gamma(U,\G)$
mit~$\beta_U(t) = 0 \in \Gamma(U,\H)$ existieren eine offene Überdeckung~$U =
\bigcup_{i \in I} U_i$ und Schnitte~$s_i \in \Gamma(U_i,\F)$
mit~$\alpha_{U_i}(s_i) = t|_{U_i}$ für alle~$i \in I$.
\end{itemize}

Der Begriff der Exaktheit einer Sequenz von \emph{Prä}garben
abelscher Gruppen wurde anders definiert. Beide Definitionen fallen nicht vom
Himmel, in ihrem jeweiligen Kontext (Garben bzw. Prägarben) sind sie jeweils
genau die richtigen. Das werden wir noch verstehen.

\begin{enumerate}
\item Zeige, dass eine Sequenz von Garben abelscher Gruppen genau dann exakt
ist, wenn sie \emph{halmweise exakt} ist, wenn also die induzierten
Sequenzen~$\F_x \to \G_x \to \H_x$ von abelschen Gruppen für alle~$x \in X$ exakt sind.

\item Sei~$\F \to \G \to \H$ eine exakte Sequenz von \emph{Prä}garben.
Seien~$\F$, $\G$ und~$\H$ aber trotzdem sogar Garben. Zeige, dass die Sequenz
dann auch als Sequenz von Garben exakt ist.

\item Sei~$0 \to \F \to \G \to \H \to 0$ eine kurze exakte Sequenz von
\emph{Prä}garben auf einem topologischen Raum. Seien~$\F$ und~$\H$ sogar
Garben. Zeige, dass~$\G$ ebenfalls eine Garbe ist.

\item \emph{Schnitte nehmen ist linksexakt.}
Sei~$0 \to \F \to \G \to \H \to 0$ eine kurze exakte Sequenz von Garben
abelscher Gruppen. Sei~$U \subseteq X$ eine offene Teilmenge. Zeige, dass die
induzierte Sequenz
\[ 0 \lra \Gamma(U,\F) \lra \Gamma(U,\G) \lra \Gamma(U,\H) \mathrel{\hcancel{$\lra
0$}{0pt}{-0pt}{0pt}{-2pt}} \]
noch exakt ist. Wieso geht die Surjektivität hinten verloren?
(Wenn dem nicht so wäre, gäbe es übrigens das gesamte Teilgebiet der
\emph{Garbenkohomologie} nicht.)
\end{enumerate}
\end{aufgabe}

\begin{aufgabe}{Inneres Hom}
Seien~$\F$ und~$\G$ Prägarben auf einem topologischen Raum~$X$. Dann
definieren wir eine weitere Prägarbe durch die Setzung
\[ \Gamma(U, \HOM(\F,\G)) := \{ \alpha : \F|_U \to \G|_U \text{ Morphismus von Prägarben auf~$U$} \} \]
und die offensichtlichen Einschränkungsabbildungen (welche?).

Zeige: Ist~$\G$ sogar eine Garbe, so ist~$\HOM(\F,\G)$ ebenfalls eine Garbe.
\end{aufgabe}

\begin{aufgabe}{Welke Garben}
Eine Garbe~$\F$ auf einem topologischen Raum~$X$ heißt genau dann \emph{welk}
(engl. \emph{flabby}, franz. \emph{flasque}),
wenn die Einschränkungsabbildungen~$\Gamma(X,\F) \to \Gamma(U,\F)$ für alle
offenen Teilmengen~$U \subseteq \F$ surjektiv sind.

\begin{enumerate}
\item Zeige durch ein explizites Beispiel, dass die Garbe~$\C$ der stetigen
Funktionen auf~$\RR$ aus Aufgabe~1b) nicht welk ist.
\item Sei~$0 \to \F \to \G \to \H \to 0$ eine kurze exakte Sequenz von Garben
abelscher Gruppen auf einem topologischen Raum~$X$. Sei~$\F$ welk. Sei~$U
\subseteq X$ eine offene Teilmenge. Zeige, dass
dann auch die induzierte Sequenz
\[ 0 \lra \Gamma(U,\F) \lra \Gamma(U,\G) \lra \Gamma(U,\H) \lra 0 \]
exakt ist. Wegen dieser besonderen Eigenschaft sind welke Garben für die
homologische Algebra wichtig.

\emph{Tipp:} Opfere eine Katze, um geeignete maximale Fortsetzungen
zu konstruieren.
\item Sei~$0 \to \F \to \G \to \H \to 0$ eine kurze exakte Sequenz von Garben
abelscher Gruppen auf einem topologischen Raum~$X$. Seien~$\F$ und~$\G$ welk.
Zeige, dass dann auch~$\H$ welk ist.
\item Sei~$\pi : E \to X$ eine stetige Surjektion. Zeige, dass die
Garbe~$\widetilde\E$ \emph{aller} Schnitte von~$E \xra{\pi} X$, also die Garbe
mit~$\Gamma(U,\widetilde\E) = \{ s : U \to E \,|\, \pi
\circ s = \id|_U \}$ und den gewöhnlichen Einschränkungsabbildungen, welk ist.
\item Zeige, dass man jede Garbe auf einem topologischen Raum in eine geeignete
welke Garbe einbetten kann.
\end{enumerate}
\end{aufgabe}

\begin{aufgabe}{Weiche Garben}
Eine Garbe~$\F$ auf einem topologischen Raum~$X$ heißt genau dann \emph{weich}
(engl. \emph{soft}, franz. \emph{mou}), wenn die Abbildungen~$\Gamma(X,\F) \to
\Gamma(A,\F)$ für alle \emph{abgeschlossenen} Teilmengen~$A \subseteq X$
surjektiv sind. (Zur Erinnerung:~$\Gamma(A,\F) = \colim\limits_{{U \subseteq
X\,\text{offen},\,A \subseteq U}} \Gamma(U,\F).$)

Weiche Garben werden vor allem auf parakompakten Hausdorffräumen studiert. Ein
topologischer Raum ist genau dann \emph{parakompakt}, wenn jede offene
Überdeckung~$X = \bigcup_i U_i$ eine \emph{lokal endliche Verfeinerung}~$X =
\bigcup_j V_j$ besitzt: Jede der offenen Teilmengen~$V_j$ soll in einer der
Mengen~$U_i$ enthalten sein, und jeder Punkt von~$X$ soll in nur endlich
vielen Mengen~$V_j$ liegen. Parakompakte Hausdorffräume haben ferner folgende
besondere Eigenschaft: Zu jeder offenen Überdeckung~$X = \bigcup_i U_i$
existieren offene Mengen~$V_i$ mit~$\overline{V_i} \subseteq U_i$, welche immer
noch~$X$ überdecken.

\begin{enumerate}
\item Zeige, dass die Garbe~$\C$ der stetigen Funktionen auf~$\RR$ aus
Aufgabe~1b) weich ist.
\item Zeige, dass welke Garben stets weich sind.
\item Sei~$X$ ein parakompakter Hausdorffraum. Zeige die analoge Behauptung wie
bei Aufgabe~4b), nur mit~"`$\F$ weich"' statt~"`$\F$ welk"' und mit~"`$A
\subseteq X$ abgeschlossen"' statt~"`$U \subseteq X$ offen"'.
\item Sei~$X$ ein parakompakter Hausdorffraum. Zeige die analoge Behauptung wie
bei Aufgabe~4c), nur mit~"`weich"' statt~"`welk"'.
\end{enumerate}
\end{aufgabe}

\begin{aufgabe}{Feine Garben}
Eine Garbe~$\F$ abelscher Gruppen auf einem topologischen Raum~$X$ heißt genau dann \emph{fein}
(engl. \emph{fine}, franz. \emph{fin}), wenn für je zwei disjunkte abgeschlosse
Mengen~$A_1, A_2 \subseteq X$ ein Morphismus~$\alpha : \F \to \F$ von Garben
abelscher Gruppen existiert, sodass~$\alpha$ auf einer offenen Umgebung
von~$A_1$ Null und auf einer offenen Umgebung von~$A_2$ die Identität ist.
(Das bedeutet, dass es offene Mengen~$U_1 \supseteq A_1$
und~$U_2 \supseteq A_2$ gibt, sodass~$\alpha_V$ für alle offenen Teilmengen~$V
\subseteq U_1$ die Nullabbildung und sodass~$\alpha_V$ für alle offenen
Teilmengen~$V \subseteq U_2$ die Identitätsabbildung ist.)

\begin{enumerate}
\item Zeige, dass feine Garben auf parakompakten Hausdorffräumen stets weich
sind.
\item Zeige, dass eine Garbe~$\F$ abelscher Gruppen auf einem parakompakten
Hausdorffraum genau dann fein ist, wenn die Hom-Garbe~$\HOM(\F,\F)$ welk ist.

\emph{Tipp:} Parakompakte Hausdorffräume sind \emph{normal}, das heißt
je zwei disjunkte abgeschlossene Teilmengen besitzen offene disjunkte
Umgebungen.
\end{enumerate}
\end{aufgabe}

\begin{center}-- \emph{Noch zu \TeX{}en} --\end{center}
\begin{itemize}
\item Ausgelagerter Beweis: lokaler Homöomorphismus \ldots
\item Affine Schemata
\end{itemize}

\begin{center}-- \emph{Weitere mögliche Aufgaben} --\end{center}
\begin{itemize}
\item Halme von~$\O_X$ für~$X = \CC$ berechnen.
\item Exaktheit der Sequenz~$0 \to \underline{\ZZ} \to \O_X \to \O_X^\times \to 0$
nachrechnen.
\item Verträglichkeit der beiden Definition für~$\Gamma(A,\F)$ nachrechnen,
falls~$A$ sowohl offen als auch abgeschlossen ist.
\end{itemize}


\end{document}

\begin{aufgabe}{Mono- und Epimorphismen zwischen Garben}
Ein Morphismus~$f$ einer gewissen Klasse von Morphismen heißt genau dann
\emph{Monomorphismus}, wenn er bezüglich der Verkettung von Morphismen
linkskürzbar ist, d.\,h. wenn für beliebige parallele Morphismen~$p$ und~$q$
aus~$f \circ p = f \circ q$ schon~$p = q$ folgt. Dual ist
ein \emph{Epimorphismus} ein rechtskürzbarer Morphismus. Zeige:
\begin{enumerate}
\item Unter allen Abbildungen von Mengen sind die Monomorphismen gerade die
injektiven Abbildungen.
\item Unter allen Abbildungen von Mengen sind die Epimorphismen gerade die
surjektiven Abbildungen.
\item Unter allen Morphismen von Garben auf einem festen topologischen Raum~$X$
sind die Monomorphismen gerade
diejenigen Garbenmorphismen~$\alpha : \F \to \G$, deren
Komponentenabbildungen~$\alpha_U$ für alle offenen Teilmengen~$U \subseteq X$
injektiv sind. (Das ist genau dann der Fall, wenn die Abbildungen~$\alpha_x :
\F_x \to \G_x$ zwischen den Halmen alle injektiv sind.)
\item Unter allen Morphismen von Garben auf einem festen topologischen Raum~$X$
sind die Epimorphismen gerade
diejenigen Garbenmorphismen~$\alpha : \F \to \G$, deren
Halmabbildungen~$\alpha_x : \F_x \to \G_x$ alle surjektiv sind. (Obacht: Das
ist echt schwächer als zu sagen, dass alle Komponentenabbildungen~$\alpha_U$
surjektiv sind.)
\end{enumerate}
\end{aufgabe}

\begin{aufgabe}{Mono- und Epimorphismen von Garben}
Ein Morphismus~$\alpha : \F \to \G$ von Garben auf einem topologischen Raum~$X$
heißt genau dann \emph{Epimorphismus}, wenn zu jeder offenen Menge~$U \subseteq X$ und
jedem lokalen Schnitt~$s \in \Gamma(U,\G)$ eine offene Überdeckung~$U =
\bigcup_{i \in I} U_i$ und Schnitte~$t_i \in \Gamma(U_i,\F)$ existieren,
sodass~$\alpha_{U_i}(t_i) = s|_{U_i}$ für alle~$i \in I$.
\end{aufgabe}
