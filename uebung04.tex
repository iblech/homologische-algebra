\documentclass{uebblatt}

\begin{document}

\maketitle{4}{-- \emph{Motto} --}

\begin{aufgabe}{Äquivalenzrelationen I}
Sei~$X$ eine Menge und~$R \subseteq X \times X$ eine Relation auf~$X$
(also lediglich eine Teilmenge, nicht unbedingt eine Äquivalenzrelation).
Sei~$(\sim_R)$ der Schnitt über alle Äquivalenzrelation~$S$ auf~$X$, welche~$R$
umfassen.
\begin{enumerate}
\item Zeige: Der Schnitt~$(\sim_R)$ ist wieder eine Äquivalenzrelation auf~$X$
-- und zwar die feinste, die~$R$ umfasst. (Was bedeutet das? Für jede weitere
Äquivalenzrelation \ldots)

\item Zeige, dass diese auch explizit \emph{(prädikativ)} wie folgt beschrieben
werden kann:
\[ x \sim_R y \quad\Longleftrightarrow\quad
  \exists n \geq 0{:}\
  \exists x_1,\ldots,x_n \in X\_
  x R x_1 \wedge x_1 R x_2 \wedge \cdots \wedge x_{n-1} R x_n \wedge x_n R y.
  \]

\item Sei~$f : X \to Y$ eine Abbildung. Gelte~$f(x) = f(y)$ für alle~$x,y \in
X$ mit~$xRy$. Zeige: Die Setzung~$\bar f : X/{\sim_R} \to Y,\ [x] \mapsto f(x)$
ist wohldefiniert.

\emph{Hinweis:} Spannender ist es, wenn man diese Teilaufgabe direkt mit~a) und
ohne Verwendung von~b) löst.
\end{enumerate}
\end{aufgabe}

\begin{aufgabe}{Äquivalenzrelationen II}
Seien~$Z$ eine Menge und~$R_1$ und~$R_2$ Äquivalenzrelationen auf~$Z$.
Sei~$\sim$ folgende Relation auf~$Z/R_1$:
\[ K \sim L \quad:\Longleftrightarrow\quad
  \exists x \in K, y \in L{:}\ x R_2 y. \]
Sei ferner~$R$ die feinste Äquivalenzrelation auf~$Z$,
welche~$R_1 \cup R_2$ umfasst.

\begin{enumerate}
\item Wieso ist~$\sim$ im Allgemeinen keine Äquivalenzrelation? (Bemühe dich
nicht, ein konkretes Gegenbeispiel aufzustellen.)
\item Sei~$\approx$ die feinste Äquivalenzrelation auf~$Z/R_1$, welche~$\sim$
umfasst. Gib eine kanonische Abbildung~$Z/R \to (Z/R_1)/{\approx}$ an und zeige,
dass sie eine wohldefinierte Bijektion ist.
\item Sei~$Z$ sogar ein topologischer Raum. Zeige dann, dass die Bijektion aus
Teilaufgabe~b) sogar ein Homöomorphismus ist. Die diversen Faktormengen
sollen dabei die Quotiententopologie tragen.
\end{enumerate}
\end{aufgabe}

\end{document}
