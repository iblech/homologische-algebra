\documentclass{uebblatt}

\begin{document}

\maketitle{4}{-- \emph{Induktive Konstruktionen} --}

\begin{aufgabe}{Äquivalenzrelationen I}
Sei~$X$ eine Menge und~$R \subseteq X \times X$ eine Relation auf~$X$
(also lediglich eine Teilmenge, nicht unbedingt eine Äquivalenzrelation).
Sei~$\sim_R$ der Schnitt über alle Äquivalenzrelationen auf~$X$, welche~$R$
umfassen.
\begin{enumerate}
\item Zeige: Der Schnitt~$\sim_R$ ist wieder eine Äquivalenzrelation auf~$X$
-- und zwar die feinste, die~$R$ umfasst. (Was bedeutet das? Für jede weitere
Äquivalenzrelation \ldots)

\item Zeige, dass diese auch explizit wie folgt beschrieben
werden kann:
\begin{multline*}
  x \sim_R y \quad\Longleftrightarrow\quad
  \exists n \geq 0{:}\
  \exists x_0,x_1,\ldots,x_n \in X{:} \\
  x_0 = x \wedge x_n = y \wedge
  \forall i \in \{ 0 , \ldots, n-1 \}{:}\
  x_i R x_{i+1} \vee x_{i+1} R x_i.
\end{multline*}

\item Sei~$f : X \to Y$ eine Abbildung. Gelte~$f(x) = f(y)$ für alle~$x,y \in
X$ mit~$xRy$. Zeige: Die Setzung~$\bar f : X/{\sim_R} \to Y,\ [x] \mapsto f(x)$
ist wohldefiniert.

\emph{Hinweis:} Spannender ist es, wenn man diese Teilaufgabe direkt mit~a) und
ohne Verwendung von~b) löst.

\item Finde Beispiele aus möglichst vielen verschiedenen Teilgebieten der
Mathematik, wo man ebenfalls geeignete große Schnitte zur Konstruktion von Mengen
mit gewissen guten Eigenschaften verwendet.
\end{enumerate}

In \emph{prädikativer Mathematik} gestattet man sich nicht, das
\emph{Potenzmengenaxiom} zu verwenden, demnach jede Menge eine Potenzmenge
besitzt (statt nur einer Potenzklasse). Als Konsequenz kann man feinste
Äquivalenzrelationen nicht mehr als geeigneten Schnitt konstruieren, sondern
nur noch über die explizite Variante aus Teilaufgabe~b). Man sagt auch,
prädikative Mathematik sei konkret -- aber nicht notwendigerweise elegant.
Prädikative Mathematik kann man sowohl klassisch als auch intuitionistisch
betreiben.
\end{aufgabe}

\begin{aufgabe}{Äquivalenzrelationen II}
Seien~$Z$ eine Menge und~$R_1$ und~$R_2$ Äquivalenzrelationen auf~$Z$.
Sei~$R$ die feinste Äquivalenzrelation auf~$Z$, welche~$R_1 \cup R_2$ umfasst.
Sei ferner~$\sim$ folgende Relation auf~$Z/R_1$:
\[ K \sim L \quad:\Longleftrightarrow\quad
  \exists x \in K, y \in L{:}\ x R_2 y \qquad\qquad\text{für alle~$K,L \in
  Z/R_1$.} \]

\begin{enumerate}
\item Wieso ist~$\sim$ im Allgemeinen keine Äquivalenzrelation? (Bemühe dich
nicht, ein konkretes Gegenbeispiel aufzustellen.)
\item Sei~$\approx$ die feinste Äquivalenzrelation auf~$Z/R_1$, welche~$\sim$
umfasst. Gib eine kanonische Abbildung~$Z/R \to (Z/R_1)/{\approx}$ an und zeige,
dass sie eine wohldefinierte Bijektion ist.
\item Sei~$Z$ sogar ein topologischer Raum. Zeige dann, dass die Bijektion aus
Teilaufgabe~b) sogar ein Homöomorphismus ist. Die diversen Faktormengen
sollen dabei die Quotiententopologie tragen.
\end{enumerate}
\end{aufgabe}

%\begin{center}\emph{-- Bitte wenden. --}\end{center}

\begin{aufgabe}{Triangulationen von Prismen}
Bestimme alle nichtdegenerierten Simplizes der simplizialen Mengen~$D[1,2]$,
$D[1,n]$ und~$D[2,2]$.
\end{aufgabe}

\begin{aufgabe}{Homotopien simplizialer Abbildungen}
Bezeichne allgemein~$X \times Y$ das kartesische Produkt simplizialer
Mengen~$X$ und~$Y$; es gilt also~$(X \times Y)_n = X_n \times Y_n$ für alle~$n
\geq 0$.

\begin{enumerate}
\item Zeige, dass simpliziale Abbildungen~$I \to X \times Y$ in kanonischer
1:1--Korrespondenz zu Paaren von simplizialen Abbildungen~$I \to X$,~$I \to Y$
stehen.
\item Definiere zwei sinnvolle simpliziale Abbildungen~$p_0, p_1 : X \to
\Delta[1] \times X$ -- in Analogie zu den stetigen Abbildungen~$x \mapsto
(0,x)$ bzw.~$x \mapsto (1,x)$, die zwischen einem topologischen Raum und seinem
Produkt mit dem Einheitsintervall verlaufen.
\end{enumerate}
Simpliziale Abbildungen~$f,g : X \to Y$ heißen genau dann \emph{einfach
homotop}, wenn es eine simpliziale Abbildung~$h : \Delta[1] \times X \to Y$
gibt sodass~$f = h \circ p_0$ und~$g = h \circ p_1$. Das definiert keine
Äquivalenzrelation auf der Menge der simplizialen Abbildungen von~$X$ nach~$Y$;
die feinste solche Äquivalenzrelation, die einfach homotope Abbildungen
identifiziert, heißt \emph{Homotopie}.
\begin{enumerate}
\addtocounter{enumi}{2}
\item Sei für~$0 \leq i \leq n$ die Abbildung~$\pr_i : \Delta[n] \to \Delta[n]$
diejenige, die ``alles auf die~$i$-te Ecke projiziert''. Konkret gelte
also~$(\pr_i)_n(f) = u_k$ für alle~$n,k \geq 0$ und $f : [k] \to [n]$. Dabei
bezeichne~$u_k$ die konstante Abbildung~$[k] \to [n]$ mit Wert~$i$.

Zeige, dass die Abbildung~$\pr_n$ zur Identitätsabbildung homotop ist.

\item Zeige: Sind~$f$ und~$g$ homotop, so
auch~$q \circ f \circ p$ und~$q \circ g \circ p$.
\[ \xymatrix{
  X' \ar[r]^p & X \ar@/^/[r]^f \ar@/_/[r]_g & Y \ar[r]^q & Y'
} \]

\item \emph{Schwierige und schwammige Bonusaufgabe zum Grübeln.} Inwieweit
impliziert schwache Homotopie von simplizialen Abbildungen die gewöhnliche
topologische Homotopie der zugehörigen geometrischen Realisierungen? Wie sehen
gegebenenfalls solche Homotopien aus?
\end{enumerate}
\end{aufgabe}

\begin{aufgabe}{Induktive Konstruktion des Skeletts}
Anschaulich ergibt sich das~$n$-Skelett einer simplizialen Menge aus
ihrem~$(n-1)$-Skelett durch Einkleben der nichtdegenerierten~$n$-Simplizes.
Allgemein ergibt sich eine simpliziale Menge durch geeignete Verklebung
ihrer nichtdegenerierten Simplizes. Das wollen wir in dieser Aufgabe verstehen.
Sei~$\dot \Delta[n]$ diejenige simpliziale Untermenge von~$\Delta[n]$, der das
eindeutig bestimmte nichtdegenerierte~$n$-Simplex fehlt:
\[ \dot \Delta[n]_m := \{ f : [m] \to [n] \,|\,
  \text{$f$ ist monoton und nicht surjektiv} \} \subseteq \Delta[n]_m. \]
Das ist eine simpliziale Version der~$(n-1)$-Sphäre.
Das \emph{Koprodukt} (disjunkt-gemachte Vereinigung) von simplizialen Mengen
wird einfach levelweise berechnet. Daher gilt
\[ \Bigl(\coprod_{x \in I} \dot\Delta[n]\Bigr)_m =
  \coprod_{x \in I} \dot\Delta[n]_m =
  \{ (x, f) \,|\, x \in I,\ f \in \dot\Delta[n]_m \}. \]
Bei den folgenden Aufgaben fallen viele Detailnachweise an. Kläre so viele, wie
du möchtest. Sei im Folgenden~$X$ eine simpliziale Menge und~$X_{(n)}$ die
Menge ihrer nichtdegenerierten~$n$-Simplizes.

\begin{enumerate}
\item \emph{Yoneda lässt grüßen.} Sei~$x \in X_n$ ein Simplex. Mache dir
klar, dass diese Daten eine simpliziale Abbildung~$\bar x : \Delta[n] \to \sk_n
X,\ f \mapsto X(f)x$ definieren. Was macht diese Abbildung anschaulich?

\item Mache dir klar, dass das Bild von~$\dot\Delta[n]$ unter der Abbildung
aus~a) schon in~$\sk_{n-1} X$ liegt.

\item Gib die kanonischen Abbildungen des oberen linken Teilquadrats des folgenden
Diagramms an. Zeige, dass dieses Quadrat kommutiert. Wie sehen die simplizialen
Mengen anschaulich aus?

\item \emph{Nun kommt das eigentliche Ziel.} Sei~$Y$ eine beliebige simpliziale
Menge und seien simpliziale Abbildungen~$\sk_{n-1}X \to Y$, $\coprod_{x \in
X_{(n)}} \Delta[n] \to Y$ gegeben, die das "`schräge Quadrat"' zum Kommutieren
bringen. Zeige: Es gibt genau eine simpliziale Abbildung~$\sk_n X \to Y$, die
die beiden Teildreiecke kommutieren lässt.

\[ \xymatrixcolsep{4pc}\xymatrixrowsep{4pc}\xymatrix{
  \coprod_{x \in X_{(n)}} \dot\Delta[n] \ar[r] \ar[d] &
  \sk_{n-1} X \ar[d] \ar@/^/[rdd] \\
  \coprod_{x \in X_{(n)}} \Delta[n] \ar[r] \ar@/_/[rrd] &
  \sk_n X \ar@{-->}[rd] \\
  && Y
} \]

Später werden wir lernen, dass diese universelle Eigenschaft ausdrückt, dass
das obere linke Teilquadrat ein \emph{Pushout-Diagramm} ist. Das macht die
eingangs erwähnte Aussage über das~$n$-Skelett präzise.

\emph{Tipps:} Ein Simplex~$x \in (\sk_n X)_m$ lässt sich auf eindeutige Art und
Weise in der Form~$x = X(f)u$ schreiben, wobei~$f : [m] \to [\ell]$ eine
Surjektion,~$u \in X_\ell$ nichtdegeneriert und~$\ell \leq n$ ist. Es genügt
schon, die gesuchte Abbildung auf den nichtdegenerierten Simplizes von~$\sk_n
X$ zu definieren -- die restlichen Werte sind durch das Axiom an simpliziale
Abbildungen schon festgelegt (inwiefern?).
\end{enumerate}
\end{aufgabe}

\end{document}
