\documentclass{uebblatt}
\usepackage{calc}

\begin{document}

\maketitle{7}{}

\begin{aufgabe}{Funktorialität der langen exakten Sequenz}
Sei ein kommutatives Diagramm von Komplexen und Komplexmorphismen gegeben,
dessen Zeilen exakte Sequenzen sind:
\[ \xymatrixcolsep{3.5pc}\xymatrixrowsep{3.5pc}\xymatrix{
  0 \ar[r]
    & A^\bullet \ar[r] \ar[d]_{f^\bullet}
    & B^\bullet \ar[r] \ar[d]_{g^\bullet}
    & C^\bullet \ar[r] \ar[d]_{h^\bullet}
    & 0 \\
  0 \ar[r]
    & \widetilde A^\bullet \ar[r]
    & \widetilde B^\bullet \ar[r]
    & \widetilde C^\bullet \ar[r]
    & 0
} \]
Bekanntlich induzieren die beiden kurzen exakten Sequenzen dann lange exakte
Sequenzen in Kohomologie. Zeige, dass diese folgendes Diagramm kommutieren
lassen:
\[ \xymatrixcolsep{3.5pc}\xymatrixrowsep{3.5pc}\xymatrix{
  \cdots \ar[r]
    & H^n(A^\bullet) \ar[r] \ar[d]_{H^n(f^\bullet)}
    & H^n(B^\bullet) \ar[r] \ar[d]_{H^n(g^\bullet)}
    & H^n(C^\bullet) \ar[r] \ar[d]_{H^n(h^\bullet)}
    & H^{n+1}(A^\bullet) \ar[r] \ar[d]_{H^{n+1}(h^\bullet)}
    & \cdots \\
  \cdots \ar[r]
    & H^n(\widetilde A^\bullet) \ar[r]
    & H^n(\widetilde B^\bullet) \ar[r]
    & H^n(\widetilde C^\bullet) \ar[r]
    & H^{n+1}(\widetilde A^\bullet) \ar[r]
    & \cdots
} \]
Wenn du schon weißt, was ein Funktor ist, dann erkläre den Titel der Aufgabe!
\end{aufgabe}

\begin{aufgabe}{Degenerierte Ketten}
Sei~$A$ eine simpliziale abelsche Gruppe und~$CA_\bullet$ der zugehörige
Komplex abelscher Gruppen mit~$CA_n = A_n$ und Differential~$d = \sum_i (-1)^n
A(\partial^i)$. Sei~$DA_\bullet \hookrightarrow CA_\bullet$ der Unterkomplex
der \emph{degenerierten Ketten}.
\begin{enumerate}
\item Zeige: Das Differential~$d : CA_n \to CA_{n-1}$ bildet die Elemente
aus~$DA_n$ auf Elemente aus~$DA_{n-1}$ ab.
\end{enumerate}
Mit den Einschränkungen von~$d$ wird damit~$DA_\bullet$ zu einem Komplex und die
kanonischen Injektionen~$DA_\bullet \to NA_\bullet$ werden zu einem Komplexmorphismus.
\begin{enumerate}
\addtocounter{enumi}{1}
\item Zeige: $H_n(CA_\bullet) \cong H_n(CA_\bullet/NA_\bullet)$, auf kanonische
Art und Weise.
\item Sei nun~$X$ eine simpliziale Menge, sodass Ränder nichtdegenerierter
Simplizes wieder nichtdegeneriert sind (vgl.~Blatt~2, Aufgabe~2). Erinnere
dich, dass~$A_n := \ZZ\langle X_n \rangle$ (freie abelsche Gruppe auf den
Elementen von~$X_n$) zu einer simplizialen abelschen Gruppe wird.
Sei~$\widetilde A_n := \ZZ\langle X_{(n)} \rangle$ die freie abelsche Gruppe
auf den nichtdegenerierten~$n$-Simplizes.

Überlege zunächst, wie~$\widetilde A$ zu einer simplizialen abelschen Gruppe wird.
Zeige dann: $H_n(C\widetilde A) \cong H_n(CA)$. Damit ist also gerechtfertigt,
dass man sich bei Berechnung von Homologie auf die nichtdegenerierten Simplizes
einschränken darf.
\end{enumerate}

\emph{Auf der Rückseite wird ein ausführlicher Tipp folgen.}
\end{aufgabe}

\[ DA_n := \sum_{i=0}^{n-1} \Image(A(\sigma^i) : A_{n-1} \to A_n) \subseteq
CA_n. \]
Mit dem Summensymbol ist die Summe von Untergruppen gemeint.

\end{document}
