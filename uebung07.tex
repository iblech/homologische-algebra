\documentclass{uebblatt}
\usepackage{calc}

\begin{document}

\maketitle{7}{}

\begin{aufgabe}{Funktorialität der langen exakten Sequenz}
Sei ein kommutatives Diagramm von (Koketten-)Komplexen und Komplexmorphismen gegeben,
dessen Zeilen exakte Sequenzen sind:
\[ \xymatrixcolsep{3.5pc}\xymatrixrowsep{3.5pc}\xymatrix{
  0 \ar[r]
    & A^\bullet \ar[r] \ar[d]_{f^\bullet}
    & B^\bullet \ar[r] \ar[d]_{g^\bullet}
    & C^\bullet \ar[r] \ar[d]_{h^\bullet}
    & 0 \\
  0 \ar[r]
    & \widetilde A^\bullet \ar[r]
    & \widetilde B^\bullet \ar[r]
    & \widetilde C^\bullet \ar[r]
    & 0
} \]
Bekanntlich induzieren die beiden kurzen exakten Sequenzen dann lange exakte
Sequenzen in Kohomologie. Zeige, dass diese folgendes Diagramm kommutieren
lassen:
\[ \xymatrixcolsep{3.5pc}\xymatrixrowsep{3.5pc}\xymatrix{
  \cdots \ar[r]
    & H^n(A^\bullet) \ar[r] \ar[d]_{H^n(f^\bullet)}
    & H^n(B^\bullet) \ar[r] \ar[d]_{H^n(g^\bullet)}
    & H^n(C^\bullet) \ar[r] \ar[d]_{H^n(h^\bullet)}
    & H^{n+1}(A^\bullet) \ar[r] \ar[d]_{H^{n+1}(h^\bullet)}
    & \cdots \\
  \cdots \ar[r]
    & H^n(\widetilde A^\bullet) \ar[r]
    & H^n(\widetilde B^\bullet) \ar[r]
    & H^n(\widetilde C^\bullet) \ar[r]
    & H^{n+1}(\widetilde A^\bullet) \ar[r]
    & \cdots
} \]
Wenn du schon weißt, was ein Funktor ist, dann erkläre den Titel der Aufgabe!
\end{aufgabe}

\begin{aufgabe}{Degenerierte Ketten}
Sei~$A$ eine simpliziale abelsche Gruppe und~$CA_\bullet$ der zugehörige
Komplex abelscher Gruppen mit~$CA_n = A_n$ und Differential~$d = \sum_i (-1)^n
A(\partial^i)$. Sei~$DA_\bullet \hookrightarrow CA_\bullet$ der Unterkomplex
der \emph{degenerierten Ketten} (siehe Rückseite).
\begin{enumerate}
\item Zeige: Das Differential~$d : CA_n \to CA_{n-1}$ bildet die Elemente
aus~$DA_n$ auf Elemente aus~$DA_{n-1}$ ab.
\end{enumerate}
Mit den Einschränkungen von~$d$ wird damit~$DA_\bullet$ zu einem Komplex und die
kanonischen Injektionen~$DA_\bullet \to NA_\bullet$ werden zu einem Komplexmorphismus.
\begin{enumerate}
\addtocounter{enumi}{1}
\item Zeige: $H_n(CA_\bullet) \cong H_n(CA_\bullet/NA_\bullet)$, auf kanonische
Art und Weise.
\item Sei nun~$X$ eine simpliziale Menge, sodass Ränder nichtdegenerierter
Simplizes wieder nichtdegeneriert sind (vgl.~Blatt~2, Aufgabe~2). Erinnere
dich, dass~$A_n := \ZZ\langle X_n \rangle$ (freie abelsche Gruppe auf den
Elementen von~$X_n$) zu einer simplizialen abelschen Gruppe wird.
Sei~$\widetilde A_n := \ZZ\langle X_{(n)} \rangle$ die freie abelsche Gruppe
auf den nichtdegenerierten~$n$-Simplizes.

Überlege zunächst, wie~$\widetilde A$ zu einer simplizialen abelschen Gruppe wird.
Zeige dann: $H_n(C\widetilde A_\bullet) \cong H_n(CA_\bullet)$. Damit ist also
gerechtfertigt, dass man sich bei Berechnung von Homologie auf die
nichtdegenerierten Simplizes einschränken darf.
\end{enumerate}

\emph{Auf der Rückseite gibt es einen ausführlichen Tipp zu Teilaufgabe~b).}
\end{aufgabe}

Der Unterkomplex der degenerierten Ketten ist in Grad~$n$ durch
\[ DA_n := \sum_{i=0}^{n-1} \Image(A(\sigma^i) : A_{n-1} \to A_n) \subseteq
CA_n \]
gegeben. Mit dem Summensymbol ist die Summe von Untergruppen gemeint. Ferner
benötigen wir den \emph{normalisierten Moore-Komplex} $NA_\bullet$ mit
\begin{align*}
  NA_n &:= \bigcap_{i=0}^{n-1} \Kernel(A(\partial^j) : A_{n-1} \to A_n)
\qquad\text{(nicht bis~$n$!)} \\
\intertext{und für alle~$j \geq 0$ die Unterkomplexe~$N^jA_\bullet$ mit}
  N^jA_n &:= \begin{cases}
  \bigcap_{i=0}^j \Kernel(A(\partial^j) : A_{n-1} \to A_n), &
  \text{für~$n \geq j + 2$,} \\
  NA_n, &
  \text{für~$n \leq j + 1$.} \end{cases}
\end{align*}
Konventionsgemäß gilt~$N^{-1}A_n = CA_n$ (leere Schnitte!).
Der Quotientenkomplex~$CA_\bullet/DA_\bullet$ ist in Grad~$n$ durch die
Faktorgruppe~$CA_n/DA_n$ gegeben. Das Differential schickt~$[x] \mapsto [dx]$
und ist wegen Teilaufgabe~a) wohldefiniert. Folgende voneinander unabhängig
bearbeitbaren Schritte führen zum Ziel.

\begin{enumerate}
\item[1.] Der kanonische Morphismus~$NA_\bullet \to CA_\bullet/DA_\bullet$ ist
ein Isomorphismus von Kettenkomplexen.

\emph{Tipp:} Zeige mit Induktion über~$j$, dass für~$0 \leq j \leq n-1$ die
Verkettung~$N^j A_n \hookrightarrow CA_n \twoheadrightarrow CA_n/D^jA_n$ ein
Isomorphismus ist, wobei~$D^j A_n = \sum_{i=0}^j \Image(A(\sigma^i)).$
Die Behauptung folgt dann für~$j = n-1$. Schon der Induktionsanfang~$j = 0$ ist
interessant.

\item[2.] Die~$N^j A_\bullet$ sind für alle~$j \geq 0$ wirklich Komplexe, mit
dem Differential von~$CA_\bullet$.

\item[3.] Sei~$e^j : N^{j+1}A_\bullet \hookrightarrow N^jA_\bullet$ die
Einbettung. Wir definieren eine Möchtegernumkehrung~$f^j : N^jA_\bullet \to
N^{j+1}A_\bullet$ durch
\[ (f^j)_n(x) := \begin{cases}
  x - A(\sigma^{j+1}) A(\partial^{j+1})x, &
  \text{für $n \geq j+2$,} \\
  x, & \text{für $n \leq j+1$}. \end{cases} \]
Dann ist~$f^j$ in dem Sinne wohldefiniert, als dass es wirklich Werte
in~$N^{j+1}A_\bullet$ annimmt, und~$f^j$ ist ein Komplexmorphismus.

\item[4.] Es gilt~$f^j \circ e^j = \id_{N^{j+1}A_\bullet}$.

\item[5.] Es gilt~$\id_{N^jA_\bullet} - e^j f^j = d t^j + t^j d : N^jA_\bullet \to
N^jA_\bullet$. Dabei ist
\[ (t^j)_n(x) := \begin{cases}
  (-1)^j A(\sigma^{j+1}), & \text{für $n \geq j+1$,} \\
  0, & \text{sonst.} \end{cases} \]

\item[6.] Sei~$e : NA_\bullet \to CA_\bullet$ die Einbettung. Dann definiert
die Verkettung
\[ CA_n = N^{-1}A_n \xra{f^{-1}} N^0A_n \xra{f^0} \cdots \xra{f^{n-2}} N^{n-1}A_n =
NA_n \]
einen Komplexmorphismus~$f : CA_\bullet \to NA_\bullet$ mit~$f \circ e = \id :
NA_\bullet \to NA_\bullet$.

\item[7.] Die Morphismen~$T_n : CA_n \to CA_{n+1}$ mit
\[ T_n = e^0 \cdots e^{n-2} t^{n-1} f^{n-2} \cdots f^0 +
  e^0 \cdots e^{n-3} t^{n-2} f^{n-3} \cdots f^0 + \cdots +
  e^0 t^1 f^0 + t^0 \]
bezeugen die Kettenkomotopie~$e \circ f \simeq \id_{CA_\bullet}$. Also ist~$f$
eine Kettenhomotopieäquivalenz und induziert daher in Kohomologie
Isomorphismen.
\end{enumerate}

Diese Art der Beweisführung findet sich etwa in Goerss, Jardine:
\emph{Simplicial Homotopy Theory} (Seiten 145ff.).

\end{document}
