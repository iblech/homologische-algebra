\documentclass{uebblatt}
\usepackage{calc}

\begin{document}

\maketitle{7}{-- \emph{Motto} --}

\begin{aufgabe}{Funktorialität der langen exakten Sequenz}
Sei ein kommutatives Diagramm von Komplexen und Komplexmorphismen gegeben,
dessen Zeilen exakte Sequenzen sind:
\[ \xymatrixcolsep{3.5pc}\xymatrixrowsep{3.5pc}\xymatrix{
  0 \ar[r]
    & A^\bullet \ar[r] \ar[d]_{f^\bullet}
    & B^\bullet \ar[r] \ar[d]_{g^\bullet}
    & C^\bullet \ar[r] \ar[d]_{h^\bullet}
    & 0 \\
  0 \ar[r]
    & \widetilde A^\bullet \ar[r]
    & \widetilde B^\bullet \ar[r]
    & \widetilde C^\bullet \ar[r]
    & 0
} \]
Bekanntlich induzieren die beiden kurzen exakten Sequenzen dann lange exakte
Sequenzen in Kohomologie. Zeige, dass diese folgendes Diagramm kommutieren
lassen:
\[ \xymatrixcolsep{3.5pc}\xymatrixrowsep{3.5pc}\xymatrix{
  \cdots \ar[r]
    & H^n(A^\bullet) \ar[r] \ar[d]_{H^n(f^\bullet)}
    & H^n(B^\bullet) \ar[r] \ar[d]_{H^n(g^\bullet)}
    & H^n(C^\bullet) \ar[r] \ar[d]_{H^n(h^\bullet)}
    & H^{n+1}(A^\bullet) \ar[r] \ar[d]_{H^{n+1}(h^\bullet)}
    & \cdots \\
  \cdots \ar[r]
    & H^n(\widetilde A^\bullet) \ar[r]
    & H^n(\widetilde B^\bullet) \ar[r]
    & H^n(\widetilde C^\bullet) \ar[r]
    & H^{n+1}(\widetilde A^\bullet) \ar[r]
    & \cdots
} \]
Wenn du schon weißt, was ein Funktor ist, dann erkläre den Titel der Aufgabe!
\end{aufgabe}

\begin{aufgabe}{Degenerierte Ketten}
Sei~$C$ eine simpliziale abelsche Gruppe und~$\widetilde C$ der zugehörige
Komplex abelscher Gruppen mit Differential~$d = \sum_i (-1)^n C(\partial^i)$.
\end{aufgabe}

\end{document}
