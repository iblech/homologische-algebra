\documentclass{uebblatt}

\begin{document}

\maketitle{5}{-- \emph{Motto} --}

\begin{aufgabe}{Homologische Charakterisierung von Zusammenhang}
Sei~$X$ eine simpliziale Menge. Sei~$\approx$ die feinste Äquivalenzrelation
auf~$X_0$ mit~$X(\partial^1)u \approx X(\partial^0)u$ für alle~$u \in X_1$;
anschaulich gilt genau dann~$x \approx y$, wenn sich die~0-Simplizes~$x$
und~$y$ durch einen Kantenzug miteinander verbinden lassen.
\begin{enumerate}
\item Seien~$x,y \in X_0$. Zeige, dass die entsprechenden Punkte in der
geometrischen Realisierung~$|X|$ genau dann durch einen stetigen Pfad
miteinander verbunden werden können, wenn~$x \approx y$.

\emph{Tipp:} Eine Richtung ist leichter als die andere. Konstruiere für die
andere eine geeignete simpliziale Abbildung~$X \to \ul{\Omega}$ und betrachte
deren geometrische Realisierung. Dabei bezeichnet~$\Omega \supseteq \{0,1\}$ die
Menge der Wahrheitswerte und~$\ul{\Omega}$ die diskrete simpliziale Menge mit
Eckenmenge~$\Omega$. Verwende, dass das Einheitsintervall zusammenhängend ist.

\item Zeige für beliebige 0-Simplizes~$x,y \in X_0$:
\[ x \approx y
  \quad\Longrightarrow\quad
  x - y \in \Image(d^0 : C_1(X,\ZZ) \to C_0(X,\ZZ)). \]

\item Zeige:~$H_0(X,\ZZ) \cong \ZZ\langle \text{Wegzusammenhangskomponenten
von~$|X|$} \rangle$ (freier~$\ZZ$-Modul).
\end{enumerate}
\end{aufgabe}

\begin{aufgabe}{Homologieberechnungen}
Berechne die Homologie (mit Koeffizienten in~$\ZZ$) von folgenden simplizialen
Mengen:
\begin{enumerate}
\item das Standard-$n$-Simplex~$\Delta[n]$,
\item die~$(n-1)$-dimensionale Sphäre~$\dot\Delta[n]$,
\item der zweidimensionale Torus,
\item die reelle projektive Ebene.
\end{enumerate}
Verwende dazu als Kettengruppen die freien~$\ZZ$-Moduln über den
\emph{nichtdegenerierten} Simplizes; später werden wir verstehen, wieso diese
die volle Homologie berechnen. Inwieweit bestätigt sich das Motto \emph{Homologie misst
(mehrdimensionale) Löcher}?\footnote{Mit Homologie kann man auch Löcher des
umgebenden logischen Rahmens messen, etwa inwieweit das Auswahlaxiom
fehlschlägt: Andreas Blass.
\emph{\href{http://www.ams.org/journals/tran/1983-279-01/S0002-9947-1983-0704615-7/S0002-9947-1983-0704615-7.pdf}{Cohomology detects failures of the axiom of
choice}}. Trans. Amer. Math. Soc. \textbf{279}, S.~257--269.}
\end{aufgabe}

\end{document}
