\documentclass{uebblatt}

\begin{document}

\maketitle{5}{-- \emph{Hier könnte dein Motto stehen.} --}

\begin{aufgabe}{Homologische Charakterisierung von Zusammenhang}
Sei~$X$ eine simpliziale Menge. Sei~$\approx$ die feinste Äquivalenzrelation
auf~$X_0$ mit~$X(\partial^1)u \approx X(\partial^0)u$ für alle~$u \in X_1$;
anschaulich gilt genau dann~$x \approx y$, wenn sich die~0-Simplizes~$x$
und~$y$ durch einen Kantenzug miteinander verbinden lassen.
\begin{enumerate}
\item Seien~$x,y \in X_0$. Zeige, dass die entsprechenden Punkte in der
geometrischen Realisierung~$|X|$ genau dann durch einen stetigen Pfad
miteinander verbunden werden können, wenn~$x \approx y$.

\emph{Tipp:} Eine Richtung ist leichter als die andere. Konstruiere für die
andere eine geeignete simpliziale Abbildung~$X \to \ul{\Omega}$ und betrachte
deren geometrische Realisierung. Dabei bezeichnet~$\Omega \supseteq \{0,1\}$ die
Menge der Wahrheitswerte und~$\ul{\Omega}$ die diskrete simpliziale Menge mit
Eckenmenge~$\Omega$. Verwende, dass das Einheitsintervall zusammenhängend ist.

\item Zeige für beliebige 0-Simplizes~$x,y \in X_0$:
\[ x \approx y
  \quad\Longrightarrow\quad
  x - y \in \Image(d^0 : C_1(X,\ZZ) \to C_0(X,\ZZ)). \]

\item Zeige:~$H_0(X,\ZZ) \cong \ZZ\langle \text{Wegzusammenhangskomponenten
von~$|X|$} \rangle$ (freier~$\ZZ$-Modul).
\end{enumerate}
\end{aufgabe}

\begin{aufgabe}{Homologieberechnungen}
Berechne die Homologie (mit Koeffizienten in~$\ZZ$) von folgenden simplizialen
Mengen:
\begin{enumerate}
\item dem Standard-$n$-Simplex~$\Delta[n]$,
\item der~$(n-1)$-dimensionalen Sphäre~$\dot\Delta[n]$ (siehe Aufgabe~5 von Blatt~4),
\item dem zweidimensionalen Torus,
\item der reellen projektiven Ebene.
\end{enumerate}
Verwende dazu als Kettengruppen die freien~$\ZZ$-Moduln über den
\emph{nichtdegenerierten} Simplizes; später werden wir verstehen, wieso diese
die volle Homologie berechnen. Inwieweit bestätigt sich das Motto \emph{Homologie misst
(mehrdimensionale) Löcher}?\footnote{Mit Homologie kann man auch Löcher des
umgebenden logischen Rahmens messen, etwa inwieweit das Auswahlaxiom
fehlschlägt: Andreas Blass.
\emph{\href{http://www.ams.org/journals/tran/1983-279-01/S0002-9947-1983-0704615-7/S0002-9947-1983-0704615-7.pdf}{Cohomology detects failures of the axiom of
choice}}. Trans. Amer. Math. Soc. \textbf{279}, S.~257--269.}
\end{aufgabe}

\begin{center}-- \emph{Bitte wenden.} --\end{center}

\newpage

\begin{aufgabe}{Affine Schemata I}
Sei~$A$ ein kommutativer Ring (mit Eins). Sei~$\Spec A$ die Menge der Primideale
von~$A$. Eine Teilmenge~$U \subseteq \Spec A$ heißt genau dann \emph{offen},
wenn sie eine (beliebige) Vereinigung von \emph{standardoffenen Mengen}
ist; solche sind Mengen der Form~$D(f) = \{ \ppp \in \Spec A \,|\, f \not\in
\ppp \}$ mit~$f \in A$. Anschaulich stellt man sich ein Ringelement~$f \in A$
als \emph{Funktion} auf~$\Spec A$ und die Menge~$D(f)$ als Menge der Punkte,
wo~$f$ nicht verschwindet, vor.

Ist~$\aaa \subseteq A$ ein Ideal, so ist~$\sqrt{\aaa} \defeq \{ f \in A \,|\,
\exists n \geq 0{:}\ f^n \in \aaa \}$ das zugehörige \emph{Radikalideal}.
Sind~$f_1,\ldots,f_n$ Ringelemente, so ist~$(f_1,\ldots,f_n) \defeq \{ \sum_i a_i
f_i \,|\, a_1,\ldots,a_n \in A \}$ das von diesen Elementen \emph{erzeugte
Ideal}.

Zeige folgende Behauptungen und interpretiere sie anschaulich, für alle
Ringelemente~$f,g,g_1,\ldots,g_n \in A$ und Primideale~$\ppp \in \Spec A$:
\begin{enumerate}
\item $D(f) \subseteq D(g) \Longleftrightarrow \sqrt{(f)} \subseteq
\sqrt{(g)}$.
\item $D(f) \subseteq D(g_1) \cup \cdots \cup D(g_n) \Longleftrightarrow
\sqrt{(f)} \subseteq \sqrt{(g_1,\ldots,g_n)}$.
\item $D(f) \cap D(g) = D(fg)$.
\item Der topologische Abschluss von~$\{\ppp\}$ ist
durch~$\{ \qqq \in \Spec A \,|\, \ppp \subseteq \qqq \}$ gegeben.
Wann ist also die Menge~$\{\ppp\}$ selbst schon abgeschlossen?
\end{enumerate}

\emph{Tipp:} Zeige, dass der Schnitt über alle Primideale~$\ppp$, welche ein
vorgegebenes Ideal~$\aaa$ umfassen, gleich~$\sqrt{\aaa}$ ist. Für eine Richtung
musst du eine Katze opfern und für geeignete Elemente~$f \in A$
folgendes Mengensystem betrachten:
\[ \U \defeq \{ \bbb \subseteq A \,|\, \text{$\bbb$ ist ein Ideal mit $\aaa
\subseteq \bbb$ und~$f^n \not\in \bbb$ für alle~$n \geq 0$} \}. \]

Diese Aufgabe ist eine Hinführung auf \emph{affine Schemata}; es fehlt noch
die Konstruktion einer geeigneten Ringgarbe~$\O_{\Spec A}$ -- erst dann kann
man Geometrie betreiben.
In intuitionistischer Logik ist die Beschreibung über
Primideale offensichtlich nicht angebracht. Da man aber den \emph{Rahmen der
offenen Teilmengen} von~$\Spec A$ explizit beschreiben kann (nämlich wie?),
kann man intuitionistisch~$\Spec A$ immer noch als \emph{Örtlichkeit}
konstruieren. Das hängt eng mit \emph{dynamischen Methoden in der Algebra}
zusammen.
\end{aufgabe}
\enlargethispage{2em}

\begin{aufgabe}{Abgeschnittene simpliziale Mengen}
Eine~\emph{$N$-abgeschnittene simpliziale Menge} ist eine Familie von
Daten~$(X_n,X(f))$ wie bei simplizialen Mengen, nur dass die Simplexmengen
lediglich für~$0 \leq n \leq N$ und die Abbildungen~$X(f)$ für~$f : [k] \to
[\ell]$ mit~$0 \leq k,\ell \leq N$ gegeben sein müssen. In offensichtlicher Weise
(wie genau?) legt jede simpliziale Menge~$X$ und jede~$M$-abgeschnittene
simpliziale Menge~$X$ (mit~$M \geq N$) eine~$N$-abgeschnittene Menge~$\Tr^N X$
fest.

Ist~$Y$ eine~$N$-abgeschnittene simpliziale Menge, so möchten wir durch die
Setzungen
\begin{align*}
  \widehat Y_m &\defeq Y_m, \qquad \text{für $m \leq N$,} \\
  \widehat Y_{N+1} &\defeq \{ (y_0,\ldots,y_{N+1}) \ |\
    y_0,\ldots,y_{N+1} \in Y_N,\
    \text{$Y(\partial^i)y_j = Y(\partial^{j-1})y_i$ für~$i < j$} \},
\end{align*}
sowie~$\widehat Y(\partial^i_{N+1}) = ((y_0,\ldots,y_{N+1}) \mapsto y_i)$
und gesunden Menschenverstand eine~$(N+1)$-abgeschnittene simpliziale Menge
definieren.

\begin{enumerate}
\item Wie kann man sich die Elemente von~$\widehat Y_{n+1}$ als
"`virtuelle"'~$(N+1)$-Simplizes vorstellen? Was sollen die~$y_i$ eines solchen
Simplex sein? Wieso soll die Kompatibilitätsbedingung an die~$y_i$ und die
Randabbildungen erfüllt sein? Inwieweit füllen diese virtuellen Simplizes
vorhandene "`simpliziale Löcher"' in~$Y_N$?

\item Leite eine sinnvolle Definition für~$\widehat Y(\sigma^i_N)$ her, $0 \leq i \leq N$.

\item Zeige, dass~$\widehat Y$ die \emph{universelle}~$(N+1)$-Fortsetzung
von~$Y$ ist; zeige also: Ist~$Z$ eine beliebige~$(N+1)$-abgeschnittene
simpliziale Menge und~$F : \Tr^N Z \to Y$ eine~$N$-abgeschnittene simpliziale
Abbildung, so gibt es genau eine Fortsetzung von~$F$ zu
einer~$(N+1)$-abgeschnittenen simplizialen Abbildung~$Z \to \widehat Y$.
\end{enumerate}
\end{aufgabe}

\end{document}
