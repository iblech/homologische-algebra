\documentclass{uebblatt}
\haiitrue
\weihnachtsbonustrue

\begin{document}

\maketitle{25}{}

\begin{aufgabe}{Čech-Methoden zur Berechnung von Garbenkohomologie}
Für eine Prägarbe~$\F$ abelscher Gruppen auf einem topologischen Raum ist die
Gruppe der~$n$-Čech-Koketten bezüglich einer Überdeckung~$\U = (U_i)_i$ von~$X$
definiert als~$\check C^n(\U, \F) \defeq \prod_{i_0,\ldots,i_n \in I}
\F(U_{i_0 \cdots i_n})$, wobei~$U_{i_0 \cdots i_n} \defeq U_{i_0} \cap \cdots
\cap U_{i_n}$. Die Kohomologie des entstehenden Komplexes wird mit~$\check
H^\bullet(\U, \F)$ bezeichnet. Sei~$\F$ im Folgenden sogar eine Garbe abelscher Gruppen.
\begin{enumerate}
\item Sei~$\iota : \AbSh(X) \to
\AbPSh(X)$ der Vergissfunktor. Sei~$U \subseteq X$ eine offene Menge.
Zeige:~$(R^n \iota\F)(U) \cong H^n(U, \F)$.

{\tiny\emph{Tipp:} Verwende die Komposition~$\Gamma_U
\circ \iota : \AbSh(X) \to \Ab$, wobei~$\Gamma_U : \AbPSh(X) \to \Ab$ die
Gruppe der~$U$-Schnitte bestimmt. Benutze Aufgabe~3 von Blatt~21, um eine
technische Voraussetzung nachzuweisen.\par}
\item Zeige: Die Garbifizierung der Prägarben~$R^n \iota \F$ ist für~$n \geq 1$
Null.

{\tiny\emph{Tipp:} Verwende~$\Id_{\AbSh(X)} \cong s \circ \iota$, wobei~$s$ der Garbifizierungsfunktor
ist.\par}
\item Konstruiere zwei Spektralsequenzen mit
\[ \check H^p(\U, R^q \iota \F) \Longrightarrow H^n(X, \F)
  \quad\text{und}\quad
  \check H^p(U, R^q \iota \F) \Longrightarrow H^n(X, \F). \]
{\tiny\emph{Hinweis:} Es ist $\check H^p(U,\E) \defeq \colim_{\U} \check H^p(\U,\E)$,
wobei~$\U$ über alle offenen Überdeckungen von~$U$ läuft. Man kann zeigen, dass
für verschiedene Verfeinerungen~$\U \to \V$ die induzierten
Morphismen~$\check H^p(\U,\E) \to \check H^p(\V,\E)$ übereinstimmen.\par}
% XXX: Tipp
\item Gelte~$H^q(U_{i_0 \cdots i_p}, \F) = 0$ für alle~$q > 0$ und~$p \geq 0$.
Zeige: $\check H^p(\U, \F) \cong H^p(X, \F)$.
\item Zeige: Die Abbildung~$\check H^n(X,\F) \to H^n(X,\F)$ ist für~$n = 0$
und~$n = 1$ ein Isomorphismus und für~$n = 2$ ein Monomorphismus.
{\tiny\emph{Tipp:} Verwende Aufgabe~2.\par}
\item Zeige: Ist~$X$ parakompakt, ist die Abbildung sogar für alle~$n \geq 0$ ein
Isomorphismus.

{\tiny\emph{Tipp:} Verwende folgendes Lemma: Ist~$\check H^n(X,\E) = 0$ für
alle~$n \geq 0$ und alle Prägarben~$\E$ mit~$s\E = 0$, so ist für alle
Prägarben~$\E$ die kanonische Abbildung~$\check H^n(X,\E) \to H^n(X,s\E)$ in
allen Graden~$n \geq 0$ ein Isomorphismus.\par}
\end{enumerate}
\end{aufgabe}

\begin{aufgabe}{Die exakte Sequenz in niedrigen Graden zu einer
Spektralsequenz}
Sei~$E_2^{pq} \Rightarrow E_\infty^n$ eine im ersten Quadranten konzentrierte
Spektralsequenz. Konstruiere daraus eine exakte Sequenz der Form
\[ 0 \lra E_2^{1,0} \lra E^1_\infty \lra E_2^{0,1} \lra E_2^{2,0} \lra
E^2_\infty. \]
\end{aufgabe}
\vspace{-1.5em}

\begin{aufgabe}{Die Serre--Hochschild-Spektralsequenz}
Sei~$G$ eine Gruppe. Ein \emph{$G$-Modul}~$A$ ist eine abelsche Gruppe~$A$
zusammen mit einer linearen Operation von~$G$.
\begin{enumerate}
\item Zeige: Der Funktor~$(\smallplaceholder)^G : \Mod(G) \to \Ab$, $A \mapsto A^G = \{ x \in A \,|\,
\text{$gx = x$ für alle $g \in G$} \}$ ist linksexakt.
\item Zeige: Die Kategorie der~$G$-Moduln ist äquivalent zur Kategorie der
Moduln über dem Ring~$\ZZ[G]$, und unter dieser Korrespondenz entspricht der
Invariantenfunktor aus~a) dem Funktor~$\Hom_{\ZZ[G]}(\ZZ, \smallplaceholder)$.
\item Sei~$H \subseteq G$ ein Normalteiler. Überlege, wie~$G$ und~$G/H$ auf~$H^n(H,A)
\defeq R^n(\smallplaceholder)^H(A)$ wirken und schreibe die Wirkungen im Kontext
der Definitionen aus der Homologischen Algebra~I explizit hin.
\item Konstruiere eine Spektralsequenz~$H^p(G/H, H^q(H,A))
\Rightarrow H^n(G,A)$.
{\tiny\emph{Tipp:} $A^G = (A^H)^{G/H}$.\par}
\end{enumerate}
\end{aufgabe}

\end{document}

http://math.stanford.edu/~notzeb/sheaf-coh.pdf
https://www.uni-due.de/~bm0065/teaching/ss09/rs5_12.pdf
