\documentclass{uebblatt}
\haiitrue

\begin{document}

\maketitle{25}{}

\begin{aufgabe}{Čech-Methoden zur Berechnung von Garbenkohomologie}
Für eine Prägarbe~$\F$ abelscher Gruppen auf einem topologischen Raum ist die
Gruppe der~$n$-Čech-Koketten bezüglich einer Überdeckung~$\U = (U_i)_i$ von~$X$
definiert als~$\check C^n(\U, \F) \defeq \prod_{i_0,\ldots,i_n \in I}
\F(U_{i_0 \cdots i_n})$, wobei~$U_{i_0 \cdots i_n} \defeq U_{i_0} \cap \cdots
\cap U_{i_n}$. Sei~$\F$ im Folgenden sogar eine Garbe abelscher Gruppen.
\begin{enumerate}
\item Sei~$\iota : \AbSh(X) \to
\AbPSh(X)$ der Vergissfunktor. Sei~$U \subseteq X$ eine offene Menge.
Zeige:~$(R^n \iota\F)(U) \cong H^n(U, \F)$.

{\tiny\emph{Tipp:} Verwende die Grothendieck-Spektralsequenz für~$\Gamma_U
\circ \iota : \AbSh(X) \to \Ab$, wobei~$\Gamma_U : \AbPSh(X) \to \Ab$ die
Gruppe der~$U$-Schnitte bestimmt. Benutze Aufgabe~3 von Blatt~21, um dazu eine
technische Voraussetzung nachzuweisen.\par}
\item Zeige: Die Garbifizierung der Prägarben~$R^n \iota \F$ ist für~$n \geq 1$
Null.

{\tiny\emph{Tipp:} Verwende die Grothendieck-Spektralsequenz
für~$\Id_{\AbSh(X)} \cong s \circ \iota$, wobei~$s$ der Garbifizierungsfunktor
ist.\par}
\item Konstruiere zwei Spektralsequenzen mit
\[ \check H^p(\U, R^q \iota \F) \Longrightarrow H^{p+q}(X, \F)
  \quad\text{und}\quad
  \check H^p(U, R^q \iota \F) \Longrightarrow H^{p+q}(X, \F). \]
{\tiny\emph{Hinweis:} Es ist $\check H^p(U,\E) \defeq \colim_{\U} \check H^p(\U,\E)$,
wobei~$\U$ über alle offenen Überdeckungen von~$U$ läuft. Man kann zeigen, dass
für verschiedene Verfeinerungen~$\U \to \V$ die induzierten
Morphismen~$\check H^p(\U,\E) \to \check H^p(\V,\E)$ übereinstimmen.\par}
% XXX: Tipp
\item Gelte~$H^q(U_{i_0 \cdots i_r}, \F) = 0$ für alle~$q > 0$ und~$r \geq 0$.
Zeige: $\check H^p(\U, \F) \cong H^p(X, \F)$.
\item Zeige: Die Abbildung~$\check H^n(X,\F) \to H^n(X,\F)$ ist für~$n = 0$ und
für~$n = 1$ ein Isomorphismus und für~$n = 2$ ein Monomorphismus.

{\tiny\emph{Tipp:} Verwende Aufgabe~2.\par}
% XXX: Tipp
\item Zeige: Ist~$X$ parakompakt, ist die Abbildung sogar für alle~$n \geq 0$ ein
Isomorphismus.
% XXX: Tipp
\end{enumerate}
\end{aufgabe}

\begin{aufgabe}{Die exakte Sequenz in niedrigen Graden zu einer
Spektralsequenz}
XXX
\end{aufgabe}

\begin{aufgabe}{Gruppenkohomologie}
Seite 214, Aufgabe 1
\end{aufgabe}

\end{document}
