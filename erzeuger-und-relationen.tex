\documentclass{uebblatt}
\usepackage{stmaryrd}
\xyoption{rotate}
\geometry{tmargin=3cm,bmargin=3cm,lmargin=3.1cm,rmargin=3.1cm}

\begin{document}

\section*{Garben über Erzeuger und Relationen}

\subsection*{Erzeuger und Relationen in der Algebra}

Gruppen oder Moduln spezifiert man gelegentlich durch Angabe von
\emph{Erzeugern} und \emph{Relationen} zwischen diesen Erzeugern.

\begin{itemize}
\item Die additive Gruppe der ganzen Zahlen ist die von einem Erzeuger frei
erzeugte Gruppe:~$\ZZ = \langle x \rangle$.
\item Der additive Monoid der natürlichen Zahlen ist der von einem Erzeuger
frei erzeugte Monoid:~$\NN = \langle x \rangle$.
\item Die Gruppe~$\ZZ/(2)$ ist die von einem Erzeuger~$x$, modulo der
Relation~$x \circ x = e$ (neutrales Element), erzeugte Gruppe:~$\ZZ/(2) =
\langle x \,|\, x \circ x = e \rangle$.
\item Die Dieder-Gruppe~$D_n$ ist durch zwei Erzeuger und eine Relation
erzeugt:~$D_n = \langle r,s \,|\, r^n = e, srs = r^{-1} \rangle$.
\item Das Tensorprodukt~$M \otimes_A N$ kann wie folgt präsentiert werden:~$M
\otimes_A N = \langle (x,y) \,|\, x \in M, y \in N, (x,y_1+y_2) =
(x,y_1)+(x,y_2), \ldots \rangle$.
\end{itemize}

Ist ein algebraisches Objekt~$X$ durch Erzeuger und Relationen gegeben, so kann
man Morphismen in ein weiteres Objekt~$Y$ einfach dadurch spezifizieren, indem
man für jeden Erzeuger von~$X$ jeweils ein gewisses Element von~$Y$ als Bild
vorgibt und darauf achtet, dass diese Bilder die gegebenen Relationen erfüllen.

Außerdem kann man ein und dieselbe Präsentation durch Erzeuger und Relationen
in verschiedenen Kontexten interpretieren. Etwa ist~$\langle x,y \rangle$ in
der Kategorie der Gruppen die von zwei Elementen frei erzeugte Gruppe~$\ZZ
\star \ZZ$. Dieselbe Präsentation führt in der Kategorie der abelschen Gruppen
zur abelschen Gruppe~$\ZZ^2$.


\subsection*{Erzeugnisse als Kolimiten}

In jeder Kategorie algebraischer Objekte kann man die Erzeugnisse von
Präsentationen als gewisse Kolimiten charakterisieren. Sei~$\ul{n}$ das von~$n$
Erzeugern~$e_1,\ldots,e_n$ ohne Relationen erzeugte Objekt: in~$\Grp$ also~$\ZZ
\star \cdots \star \ZZ$, in~$\Ab$ ist es~$\ZZ^n$, in~$\mathrm{Mod}(R)$ ist
es~$R^n$.

\begin{itemize}
\item $\ZZ \star \ZZ$ ist in der Kategorie der Gruppen der Kolimes des
folgenden Diagramms:
\[ \ul{1} \qquad \ul{1} \]
\item $\langle x,y,z \,|\, 2x = 3y \rangle$ ist in der Kategorie der abelschen
Gruppen der Kolimes des folgenden Diagramms:
\[ \xymatrix{
  \ul{1} \ar[rd]_{e_1 \mapsto e_1} && \ul{1} \ar[ld]^{e_1 \mapsto e_2} & \ul{1} \\
  & \ul{2} \\
  & \ul{1} \ar@/^/[u]^{e_1 \mapsto 2e_1} \ar@/_/[u]_{e_1 \mapsto 3e_2}
} \]
Kleinere Diagramme sind ebenfalls möglich (etwa~$\ul{1} \to \ul{2} \quad
\ul{1}$), aber weniger systematisch.
\end{itemize}

Die allgemeine Regel lautet also wie folgt:
\begin{itemize}
\item Für jeden Erzeuger~$x$ platziert man eine Kopie von~$\ul{1}$ im
Diagramm:~$\ul{1}_x$.
\item Für jede Relation~$r$ der Form~$t_1(a_1,\ldots,a_n) =
t_2(a_1,\ldots,a_n)$ platziert man eine Kopie von~$\ul{n}$,
notiert~$\ul{n}_r$; eine Kopie von~$\ul{1}$, notiert~$\ul{1}^r$; und
folgende Morphismen:

Für~$i=1,\ldots,n$ einen Morphismus~$\ul{1}_{a_i} \to \ul{n}_r$, der
den Erzeuger von~$\ul{1}_{a_i}$ auf~$e_i$ schickt; einen
Morphismus~$\ul{1}^r \to \ul{n}_r$, der den Erzeuger von~$\ul{1}^r$
auf~$t_1(e_1,\ldots,e_n)$ schickt; einen Morphismus~$\ul{1}^r \to \ul{n}_r$,
der den Erzeuger von~$\ul{1}^r$ auf~$t_2(e_1,\ldots,e_n)$.
\end{itemize}


\subsection*{Präsentationen von Prägarben und Garben}

Prägarben und Garben auf einem Raum~$X$ kann man ebenfalls durch
Erzeuger und Relationen spezifizieren. Erzeuger sind dabei Schnitte auf
bestimmten offenen Mengen, Relationen sind Vorgaben, welche Restriktionen von
Erzeugern auf kleinere offene Mengen gleich sein sollen.

Etwa ist die terminale Garbe gegeben durch einen Erzeuger~$\star$ auf ganz~$X$
und keine Relationen. Die initiale Garbe ist gegeben durch keinerlei Erzeuger
und keine Relationen. Die konstante Prägarbe~$\NN$ ist gegeben durch abzählbar
viele Erzeuger~$x_0,x_1,\ldots$ und keine Relationen.

Das Erzeugnis einer Präsentation enthält im Fall einer Kategorie algebraischer
Objekte mehr Elemente, als nur die Erzeuger selbst. Etwa enthalten Erzeugnisse
in der Kategorie der~$R$-Moduln auch formale~$R$-Linearkombinationen zwischen
den Erzeugern. Erzeugnisse in der Kategorie der Gruppen enthalten formale
Verknüpfungen und formale Inverse der Erzeuger.

Im Fall von Prägarben kommen ebenfalls zu den Erzeugern weitere Schnitte hinzu.
Ist etwa~$s$ ein Erzeuger auf~$U$, so kommen alle Restriktionen~$s|_V$ für~$V
\subsetneq U$ hinzu.

Im Fall von Garben kommen außerdem Verklebungen hinzu: Ist~$s$ ein Erzeuger
auf~$U$,~$t$ ein Erzeuger auf~$V$ und gilt~$U \cap V = \emptyset$, so gibt es
in der erzeugten Garbe auch einen Schnitt für die Verklebung von~$s$ und~$t$.
(Dass~$U \cap V = \emptyset$, ist eine unnötig starke Forderung. Es kommen auch
Verklebungen von solchen Schnitten hinzu, die auf Überlappungen
übereinstimmen.)

Außerdem können sich im Fall von Garben weitere Relationen ergeben. Gilt
etwa~$X = U \cup V$, und sind zwei Erzeuger~$s$ und~$t$ auf~$X$ mit den
Relationen~$s|_U = t|_U$ und~$s|_V = t|_V$ gegeben, so gilt im Erzeugnis automatisch
auch~$s = t$ auf ganz~$X$.


\subsection*{Garbifizierung über Präsentationen}

Die Garbifizierung einer Prägarbe kann man im Präsentationsbild sehr einfach
verstehen. Ist eine Prägarbe über irgendwelche Erzeuger und Relationen gegeben,
so ist ihre Garbifizierung durch dieselben Erzeuger und Relationen gegeben --
nur jetzt in der Kategorie der Garben interpretiert.

Diese Beobachtung war eine der Motivationen für diese Notizen.


\subsection*{Rückzug über Präsentationen}

Ist~$f : Y \to X$ eine stetige Abbildung und~$\E$ eine Garbe auf~$X$, welche
durch Erzeuger und Relationen gegeben ist, so ist die Garbe~$f^{-1}\E$ durch
dieselben Erzeuger und Relationen gegeben, nur, dass diese statt auf offenen
Mengen~$U \subseteq X$ jetzt auf den zugehörigen offenen Mengen~$f^{-1}[U]
\subseteq Y$ interpretiert werden.


\subsection*{Kategorielle Umsetzung}

Seien Erzeuger~$s_i$ auf~$U_i$ sowie Relationen~$s_{a_k}|_{W_k} =
s_{b_k}|_{W_k}$ auf~$W_k \subseteq U_{a_k} \cap U_{b_k}$ gegeben. Die
zugehörige Prägarbe oder Garbe ist in der entsprechenden Kategorie der Kolimes
des wie folgt aufgebauten Diagramms:

\begin{itemize}
\item Für den Erzeuger~$s_i$ auf~$U_i$ eine Kopie~$\A^i$ der
Garbe~$\Hom(\placeholder,U_i)$. Die Menge der Schnitte dieser Garbe auf einer
offenen Menge~$V \subseteq X$ enthält genau ein Element, falls~$V \subseteq
U_i$, und ist leer sonst.
\item Für jede Relation~$s_{a_k}|_{W_k} =
s_{b_k}|_{W_k}$ eine Kopie~$\B^k$ der Garbe~$\Hom(\placeholder,W_k)$ zusammen
mit zwei Morphismen: einem Morphismus~$\B^k \to \A^{a_k}$ und einem
Morphismus~$\B^k \to \A^{b_k}$ (diese sind eindeutig bestimmt).
\end{itemize}


\subsection*{Übungsaufgabe}

Wie kann man Prägarben und Garben abelscher Gruppen präsentieren? Zusätzlich
zur Vorgabe von Restriktionen können dabei Summen von Schnitten vorgegeben
werden. Durch welche Erzeuger und Relationen ist~$f^{-1} \E$ gegeben, wenn
Erzeuger und Relationen für~$\E$ bekannt sind? Wie sieht es mit~$f^* \E$ aus?

\end{document}
