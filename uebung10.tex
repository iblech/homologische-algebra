\documentclass{uebblatt}

\begin{document}

\maketitle{10}{}

\begin{aufgabe}{Ein konkretes Modell für endlich-dimensionale Vektorräume}
Sei~$k$ ein Körper. Sei~$\C$ die Kategorie mit
\begin{align*}
  \Ob \C &:= \NN, \\
  \Hom_\C(n, m) &:= k^{m \times n},
\end{align*}
wobei die Morphismenverkettung durch die Matrixmultiplikation gegeben ist.
\begin{enumerate}
\item Zeige, dass die Kategorie~$\C$ (auf unkanonische Art und Weise) zur Kategorie der
endlich-dimensionalen~$k$-Vektorräume äquivalent ist.

\emph{Tipp:} Wähle für jeden endlich-dim. Vektorraum~$V$ einen Iso
$\eta_V : k^{\dim V} \to V$.
\item Zeige, dass~$\C^\op$ auf kanonische Art und Weise äquivalent zu~$\C$ ist. (Das ist etwas
Besonderes!)
\item Zeige, dass die Kategorie der endlich-dimensionalen~$k$-Vektorräume
\emph{auf kanonische Art und Weise} zu ihrer dualen Kategorie äquivalent ist.
\end{enumerate}
\end{aufgabe}

\begin{aufgabe}{Kategorielle Eigenschaften}
Es gibt folgendes Motto: Sei~$\varphi$ eine mathematische Aussage über Kategorien, die
sich nur unter Verwendung der Konzepte \emph{Objekt}, \emph{Morphismus},
\emph{Verkettung von Morphismen} und \emph{Gleichheit von Morphismen}
formulieren lässt. Sind dann~$\C$ und~$\D$ zueinander äquivalente Kategorien,
so gilt~$\varphi$ genau dann in~$\C$, wenn~$\varphi$ in~$\D$ gilt.
Beispiele für Aussagen dieser Art sind etwa:
\begin{itemize}
\item Die Kategorie besitzt ein initiales Objekt (das ist ein Objekt~$0$,
sodass es zu jedem Objekt~$X$ genau ein Morphismus~$0 \to X$ gibt).
\item Je zwei parallele Morphismen sind gleich.
\item Je zwei Endomorphismen eines Objekts vertauschen miteinander.
\item Jedes initiale Objekt ist auch terminal.
\end{itemize}
Beispiele für Aussagen, die über die Reichweite des Mottos hinausgehen, sind:
\begin{itemize}
\item Die Kategorie besitzt genau ein Objekt.
\item Die Kategorie besitzt genau ein initiales Objekt.
\item Sind zwei Objekte zueinander isomorph, so sind sie schon gleich.
\item Je zwei Morphismen (egal zwischen welchen Objekten) sind gleich.
\end{itemize}

\begin{enumerate}
\item Mache dir klar, wieso das Motto gilt.
\item Anna und ihre Freundin Emma haben die Vermutung, dass die folgenden
Kategorien paarweise nicht zueinander äquivalent sind. Jemand ruft ihnen zu:
\emph{Mit Teilaufgabe~a) ist der Nachweis einfach!}, nickt und fliegt davon. Kannst du ihnen
helfen?
\[ \mathrm{Set} \quad
  \mathrm{Set}^\op \quad
  \mathrm{Vect}(\RR) \quad
  \mathrm{Ring} \quad
  \mathrm{Top} \quad
  \mathrm{Man} \quad
  \mathrm{Sh}(\RR^7) \quad
  B\ZZ \quad
  \mathrm{Hask}
\]
\item Was war passiert?
\end{enumerate}
\end{aufgabe}
\newpage

\begin{aufgabe}{Volltreue Funktoren}
\begin{enumerate}
\item Sei~$f : P \to Q$ eine monotone Abbildung zwischen Quasiordnungen. Wann
ist~$Bf : BP \to BQ$ treu? Wann voll? Wann wesentlich surjektiv?
\item Sei~$F : \C \to \D$ ein volltreuer Funktor. Zeige, dass~$F$ eine Äquivalenz
zwischen~$\C$ und einer gewissen vollen Unterkategorie von~$\D$ (welcher?)
induziert.
\item Sei~$F : \C \to \D$ ein volltreuer und wesentlich surjektiver Funktor.
Wenn wir ein genügend starkes Auswahlprinzip zur Verfügung haben, können wir zu
jedem Objekt~$Y \in \D$ ein Objekt~$X_Y \in \C$ und einen
Iso~$g_Y : F(X_Y) \to Y$ wählen.

Erkläre, wie die Zuordnung~$Y \mapsto X_Y$ einen Funktor definiert. Weise die
Funk\-tor\-ei\-gen\-schaft explizit nach. Zeige ferner, dass es einen natürlichen
Isomorphismus~$G \circ F \to \Id_\C$ gibt.
\end{enumerate}
\end{aufgabe}

\begin{aufgabe}{Quotientenkategorien}
Sei~$\C$ eine Kategorie. Seien für je zwei Objekte~$X,Y \in \C$ eine
Äquivalenzrelation~$\sim_{X,Y}$ auf~$\Hom_\C(X,Y)$ gegeben.
\begin{enumerate}
\item Konstruiere eine Kategorie~$\C/{\sim}$ zusammen mit einem Funktor~$Q : \C
\to \C/{\sim}$ mit folgender universeller Eigenschaft:
\begin{itemize}
\item Wenn~$f \sim_{X,Y} \tilde f$ in~$\C$, dann~$Q(f) = Q(\tilde f)$ in~$\C/{\sim}$.
\item Ist~$F : \C \to \D$ ein Funktor, der wie~$Q$ äquivalente Morphismen auf gleiche
schickt, so gibt es genau einen Funktor~$G : \C/{\sim} \to \D$ mit~$F = G \circ
Q$.
\end{itemize}

\emph{Tipp:} Nimm zunächst an, dass für alle passenden Morphismen~$f,a,b$
aus~$f \sim_{X,Y} \tilde f$ schon~$afb \sim_{X',Y'} a\tilde fb$ folgt.

\item Was ist an Teilaufgabe~a) inhaltlich schlecht formuliert?
\end{enumerate}
\end{aufgabe}

\begin{aufgabe}{Morita-Äquivalenz}
Ein Objekt~$X$ einer Kategorie~$\C$ heißt genau dann \emph{Erzeuger}, wenn der
Funktor~$\Hom_\C(X, \placeholder) : \C \to \Set$ volltreu ist.

\emph{Rest folgt in Kürze.}
\end{aufgabe}

\end{document}
