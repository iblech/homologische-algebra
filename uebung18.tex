\documentclass{uebblatt}
\haiitrue

\begin{document}

\maketitle{18}{}

\begin{aufgabe}{Informationsverlust beim Dualisieren}
\begin{enumerate}
\item Zeige:~$(\ZZ/(2))^\vee = 0$.
Folgere:~$(\ZZ/(2))^{\vee\vee} \not\cong \ZZ/(2)$.

Dabei ist~$M^\vee := \Hom_\ZZ(M,\ZZ)$ für~$\ZZ$-Moduln~$M$.
\item Finde eine projektive Auflösung von~$\ZZ/(2)$ der Form~$0 \to {?} \to {?} \to
0$.
\item Dualisiere den Komplex aus~b).
\item Dualisiere den Komplex aus~c).
\item Zeige: Die Komplexe aus~b) und~d) sind zueinander isomorph.
\item Was ist die Moral der Geschichte?
\end{enumerate}
\end{aufgabe}

\begin{aufgabe}{Die Kategorie der vollständigen metrischen Räume}
Sei~$\Met$ die Kategorie der metrischen Räume und lipschitzstetigen
Abbildungen. Sei~$\Met_\mathrm{compl}$ ihre volle Unterkategorie der
vollständigen metrischen Räume. Für welche Klasse~$S$ von Morphismen
gilt~$\Met_\mathrm{compl} \simeq \Met[S^{-1}]$?

%\rotatebox{180}{\emph{Tipp:} In~$S$ ist etwa die Inklusion~$\RR \setminus \{
%23, 42 \} \hookrightarrow \RR$ enthalten.}
\end{aufgabe}

%\begin{aufgabe}{Trivialität der derivierten Kategorie}
%Sei~$\A$ eine abelsche Kategorie. Sei~$k : \D(\A) \to \Kom_0(\A)$ der
%Kohomologiefunktor.
%\end{aufgabe}

\begin{aufgabe}{Die K-Theorie einer abelschen Kategorie}
Die \emph{K-Theorie}~$K(\A)$ einer abelschen Kategorie~$\A$ wird als abelsche
Gruppe von den Objekten aus~$\A$ und, für jede kurze exakte
Sequenz~$0 \to X' \to X \to X'' \to 0$ in~$\A$, der Relation~$X = X' + X''$ erzeugt.

\begin{enumerate}
\item Gelte~$X \cong Y$ in einer abelschen Kategorie~$\A$. Zeige:~$X = Y \in \K(\A)$.
\item Zeige: Die~$K$-Theorie der Kategorie der endlich-dimensionalen Vektorräume
ist isomorph zu~$\ZZ$.
\item Zeige: Die~$K$-Theorie der Kategorie aller Vektorräume ist Null.
\item Sei~$\Komb(\A)$ die Kategorie der beschränkten Kettenkomplexe über einer
abelschen Kategorie~$\A$. Zeige: Die auf Erzeugern gegebene Zuordnung
\[ K(\Komb(\A)) \longrightarrow K(\A),\ K^\bullet \longmapsto
  \chi(K^\bullet) := \sum_{n \in \ZZ} (-1)^n \, K^n \]
\vspace{-1.5em}

induziert eine wohldefinierte Surjektion.
(Teilt man in~$K(\Komb(A))$ noch die Relationen~$K^\bullet[1] =
-K^\bullet$ heraus, wird die induzierte Abbildung sogar ein Isomorphismus.)

\emph{Tipp:} Zeige~$\chi(K^\bullet) = \chi(H^\bullet(K^\bullet))$ und verwende
die lange exakte Sequenz.
\end{enumerate}
\end{aufgabe}

\enlargethispage{1em}

\begin{aufgabe}{Serresche Quotientenkategorien als Lokalisierungen}
Sei~$\B$ eine Serresche Unterkategorie einer abelschen Kategorie~$\A$. Sei~$S$
die Klasse all derjenigen Morphismen in~$\A$, deren Kerne und Kokerne in~$\B$
liegen. Zeige:~$\A/\B \simeq \A[S^{-1}]$.
\end{aufgabe}
\vspace{-1em}

\begin{aufgabe}{Wohldefiniertheit der Arbeit mit Dächern}
\begin{enumerate}
\item Zeige, dass die Komposition von Dächern wohldefiniert ist.
\item Wie addiert man zwei Dächer?
\end{enumerate}
\end{aufgabe}

\end{document}
