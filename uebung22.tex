\documentclass{uebblatt}
\haiitrue

\begin{document}

\maketitle{22}{}

\begin{aufgabe}{Von einem Erzeuger aufgespannte Unterkategorie}
Sei~$X$ ein Objekt einer abelschen Kategorie~$\A$. Sei~$\langle X \rangle
\subseteq \A$ die volle Unterkategorie aller direkten Summen direkter Summanden
von~$X$. Diese Unterkategorie wird additiv.
\begin{enumerate}
\item Sei~$\Ext_\A^i(X,X) = 0$ für alle~$i > 0$. Zeige, dass der kanonische
Funktor~$K^b(\langle X \rangle) \to D^b(\A)$ volltreu ist.
\item Gelte außerdem, dass jedes Objekt aus~$\A$ eine endliche Auflösung durch Objekte
aus~$\langle X \rangle$ besitzt. Zeige, dass der Funktor aus~a) dann sogar eine
Äquivalenz ist.
\end{enumerate}
\end{aufgabe}

\begin{aufgabe}{Auflösungen unbeschränkter Komplexe}
Eine \emph{projektive Linksauflösung} eines Komplexes~$K^\bullet$ ist ein
Komplex~$P^\bullet$ aus Projektiven zusammen mit einem
Quasiisomorphismus~$P^\bullet \to K^\bullet$. Zeige, dass unbeschränkte
Komplexe auch bis auf Homotopieäquivalenz nicht unbedingt eindeutige projektive
(ihrerseits unbeschränkte) Linksauflösungen besitzen müssen.

{\tiny\emph{Tipp:} Zeige, dass der Komplex~$P^\bullet : \cdots \xra{2} \ZZ/(4)
\xra{2} \cdots$ von~$\ZZ/(4)$-Moduln eine projektive Linksauflösung des
Nullkomplexes ist, aber nicht homotopieäquivalent zum Nullkomplex ist.\par}
\end{aufgabe}

\begin{aufgabe}{Kategorielle Charakterisierung von Endlichkeitseigenschaften}
\begin{enumerate}
\item Zeige, dass ein~$A$-Modul~$M$ genau dann endlich erzeugt ist, wenn der
Funktor~$\Hom(M,\smallplaceholder) : \mathrm{Mod}(A) \to \Set$ mit filtrierten
Kolimiten von Monomorphismen vertauscht, wenn also für jedes filtrierte
Diagramm~$(V_i)_i$, in der die Übergangsabbildungen $V_i \to V_j$ alle injektiv
sind, folgende kanonische Abbildung bijektiv ist.
\[ \colim_i \Hom(M,V_i) \longrightarrow \Hom(M, \colim_i V_i) \]
\vspace{-1.7em}
\item Zeige, dass ein~$A$-Modul~$M$ genau dann endlich präsentiert ist, wenn
der Funktor~$\Hom(M,\smallplaceholder)$ mit beliebigen filtrierten Kolimiten
vertauscht.
\end{enumerate}
\end{aufgabe}

\begin{aufgabe}{Interpretation der zweiten Ext-Gruppen}
\begin{enumerate}
\item Seien Objekte~$X \hookrightarrow Y \hookrightarrow Z$ in einer abelschen
Kategorie gegeben. Wie sieht die Vier-Term-Sequenz zu~$\gamma_1 \gamma_2 \in
\Ext^2(Z/Y,X)$ aus, wenn~$\gamma_1$ und~$\gamma_2$ zu folgenden kanonischen
Sequenzen gehören?
\[ \gamma_1 : 0 \to X \to Y \to Y/X \to 0 \qquad
  \gamma_2 : 0 \to Y/X \to Z/X \to Z/Y \to 0 \]
\vspace{-1.7em}
\item Zeige weiter, dass~$\gamma_1 \gamma_2 = 0 \in \Ext^2(Z/Y,X)$.
\item Zeige die Umkehrung: Gilt für Elemente~$\gamma_1 \in \Ext^1(B,C)$ und
$\gamma_2 \in \Ext^1(A,B)$ dass~$\gamma_1 \gamma_2 = 0 \in \Ext^2(A,C)$, so
gibt es ein Objekt~$Z$ und Unterobjekte~$X \hookrightarrow Y \hookrightarrow
Z$, sodass~$A \cong Z/Y$, $B \cong Y/X$, $C \cong X$ und sodass unter diesen
Isomorphismen~$\gamma_1$ und~$\gamma_2$ von der Form wie in~a) sind.
\end{enumerate}
\end{aufgabe}

\newpage

\begin{aufgabe}{Geometrische Interpretation eines Beweises}
Im Beweis der Vorlesung, dass Quasiisomorphismen, deren Quelle ein nach links
beschränkter Komplex aus Injektiven ist, bis auf Homotopie ein Linksinverses
besitzen (Gelfand--Manin, Seite~180f.), wurden händisch Homotopien konstruiert.
Gib einen alternativen Beweis dieser Behauptung, der die
Charakterisierung von Homotopien aus Blatt~20, Aufgabe~4 verwendet.
\end{aufgabe}

\end{document}
