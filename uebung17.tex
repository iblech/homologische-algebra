\documentclass{uebblatt}
\haiitrue

\begin{document}

\maketitle{17}{}

\begin{aufgabe}{Halmweise Exaktheit}
Sei~$X$ ein topologischer Raum (keine Örtlichkeit!). Sei~$0 \to \E \to \F \to
\G \to 0$ eine Sequenz von Garben abelscher Gruppen auf~$X$. Zeige, dass sie
genau dann exakt ist (im allgemeinen Sinn der Exaktheit in abelschen
Kategorien), wenn für jeden Punkt~$x \in X$ die induzierte Sequenz~$0 \to \E_x
\to \F_x \to \G_x \to 0$ von Halmen exakt ist.
\end{aufgabe}

\begin{aufgabe}{Kategorielle Charakterisierung von partieller Exaktheit}
Zeige, dass ein additiver Funktor zwischen abelschen Kategorien genau dann
linksexakt ist, wenn er endliche Limiten bewahrt.
\end{aufgabe}

\begin{aufgabe}{Beispiele für projektive Moduln}
\begin{enumerate}
\item Zeige, dass~$\ZZ/(2)$ als~$\ZZ/(6)$-Modul projektiv, aber
nicht frei ist.
\item Sei~$R = \ZZ[\sqrt{-5}]$ und~$\mmm = (3, 1+\sqrt{-5})$ das bekannte
Beispiel für ein maximales Ideal, das lokal ein Hauptideal, aber nicht selbst
ein Hauptideal ist. Zeige, dass~$\mmm$ als~$R$-Modul projektiv, aber nicht frei
ist.

\emph{Tipp:} Projektivität von endlich präsentierten Moduln kann man lokal
testen. Wäre~$\mmm$ frei, so wäre~$\mmm$ frei vom Rang 1 (wieso?). Damit
wäre~$\mmm$ ein Hauptideal (wieso?).
\end{enumerate}

{\scriptsize Die wichtigste Bezugsquelle für nicht-freie projektive Moduln sind
nichttriviale Vektorbündel, das besagt der Satz von Serre--Swan. Eine gut
lesbare Darstellung gibt es in Abschnitt~6 von Pete Clarks Notizen zu
kommutativer Algebra, \url{http://www.math.uga.edu/~pete/integral.pdf#page=112}.
In der algebraischen Geometrie lernt man, dass auch die Beispiele dieser
Aufgabe von dieser Form sind.\par}
\end{aufgabe}

\begin{aufgabe}{Die Kategorie der Garben als Lokalisierung}
Sei~$X$ ein topologischer Raum. Sei~$S$ die Klasse all
derjenigen Morphismen von Prägarben auf~$X$, die halmweise Bijektionen sind.
Zeige, dass die Kategorie der Garben auf~$X$ die Lokalisierung der Kategorie
der Prägarben nach~$S$ ist: $\Sh(X) \simeq \PSh(X)[S^{-1}]$. Wie sehen unter
dieser Äquivalenz der Vergissfunktor und der Garbifizierungsfunktor aus?

{\scriptsize Ist~$S$ eine Klasse von Morphismen in einer Kategorie~$\C$, so
sind die Objekte von~$\C[S^{-1}]$ per Definition dieselben wie die von~$\C$.
Morphismen~$X \to Y$ in~$\C[S^{-1}]$ sind aber formale Verknüpfungen von
Morphismen in~$\C$ und formalen Inversen von Morphismen aus~$S$. Zum Beispiel
kann ein Morphismus~$X \to Y$ in~$\C[S^{-1}]$ von der Form~$X \to Z
\dashrightarrow Z' \to Z'' \dashrightarrow Z''' \to Y$ sein -- dabei sind die
durchgezogenen Pfeile Morphismen aus~$\C$ und die gestrichelten Pfeile
formale Inverse von Morphismen~$Z' \to Z$, $Z''' \to Z''$ aus~$S$. Auf
diese Art und Weise erreicht man, dass der kanonische Funktor~$\C \to
\C[S^{-1}]$ die Morphismen aus~$S$ auf Isomorphismen schickt und sogar unter
all solchen Funktoren in einem 2-kategoriellen Sinn initial ist. Formale
Inverse sind keine Erfindung der Kategorientheorie. In der Form des Übergangs
von~$\ZZ$ zu~$\QQ$ sind sie allgemein bekannt.
\par}
\end{aufgabe}

\begin{aufgabe}{Ein Beispiel für den Pushforward von Garben}
Sei~$f : S^1 \to S^1$ die Abbildung~$z \mapsto z^2$ des Einheitskreises in
sich. Sei~$\ul{\ZZ}$ die konstante Garbe mit Halmen~$\ZZ$ auf~$S^1$. Was sind
die Halme von~$f_*\ul{\ZZ}$? Ist~$f_*\ul{\ZZ}$ isomorph zu~$\ul{\ZZ^2}$?
\end{aufgabe}

%\begin{aufgabe}{Serresche Quotientenkategorien als Lokalisierungen}
%Sei~$\B$ eine Serresche Unterkategorie einer abelschen Kategorie~$\A$. Denke
%ein wenig über die Äquivalenz~$\A/\B \simeq \A[S^{-1}]$ nach, wobei~$S$ die
%Klasse all derjenigen Morphismen in~$\A$ ist, deren Kerne und Kokerne in~$\B$
%liegen.
%\end{aufgabe}

\end{document}
