\documentclass{uebblatt}
\haiitrue

\begin{document}

\maketitle{26}{}

\begin{aufgabe}{Ein universelles Koeffiziententheorem}
Sei~$M$ ein~$R$-Modul. Sei~$P^\bullet$ ein in nichtpositiven Graden
konzentrierter Komplex flacher~$R$-Moduln.
\begin{enumerate}
\item Konstruiere eine Spektralsequenz $E_2^{pq} =
\Tor_{-p}^R(H^q(P^\bullet),M) \Rightarrow H^{p+q}(P^\bullet \otimes_R M)$.
\item Sei~$R$ sogar ein Integritätsbereich. Extrahiere für~$n \geq 0$ aus der
Spektralsequenz eine kurze exakte Sequenz der Form
\[ 0 \lra H^{-n}(P^\bullet) \otimes_R M \lra H^{-n}(P^\bullet \otimes_R M)
  \lra \Tor_1^R(H^{-n+1}(P^\bullet), M) \lra 0. \]
{\tiny\emph{Tipp:} Zeige, dass die Spektralsequenz im Bereich~$p \in \{ 0,-1
\}, q \leq 0$ konzentriert ist. Für solche Sequenzen hat die
kanonische exakte Sequenz $0 \to F^0 E_\infty^{-n} \to F^{-1}
E_\infty^{-n} \to F^{-1} E_\infty^{-n} / F^0 E_\infty^{-n} \to 0$ die Form
$0 \to E_2^{0,-n} \to E_\infty^{-n} \to E_2^{-1,-n+1} \to 0$.\par}
\end{enumerate}
\end{aufgabe}

\end{document}
