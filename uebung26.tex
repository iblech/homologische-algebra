\documentclass{uebblatt}
\haiitrue

\begin{document}

\maketitle{26}{}

\begin{aufgabe}{Ein universelles Koeffiziententheorem}
Sei~$M$ ein~$R$-Modul. Sei~$P^\bullet$ ein in nichtpositiven Graden
konzentrierter Komplex flacher~$R$-Moduln.
\begin{enumerate}
\item Konstruiere eine Spektralsequenz $E_2^{pq} =
\Tor_{-p}^R(H^q(P^\bullet),M) \Rightarrow H^{p+q}(P^\bullet \otimes_R M)$.
\item Sei~$R$ sogar ein Integritätsbereich. Extrahiere für~$n \geq 0$ aus der
Spektralsequenz eine kurze exakte Sequenz der Form
\[ 0 \lra H^{-n}(P^\bullet) \otimes_R M \lra H^{-n}(P^\bullet \otimes_R M)
  \lra \Tor_1^R(H^{-n+1}(P^\bullet), M) \lra 0. \]
{\tiny\emph{Tipp:} Zeige, dass die Spektralsequenz im Bereich~$p \in \{ 0,-1
\}, q \leq 0$ konzentriert ist. Für solche Sequenzen hat die
kanonische exakte Sequenz $0 \to F^0 E_\infty^{-n} \to F^{-1}
E_\infty^{-n} \to F^{-1} E_\infty^{-n} / F^0 E_\infty^{-n} \to 0$ die Form
$0 \to E_2^{0,-n} \to E_\infty^{-n} \to E_2^{-1,-n+1} \to 0$.\par}
\end{enumerate}
\end{aufgabe}

\begin{aufgabe}{Čech-Methoden für Einsteiger}
\begin{enumerate}
\item Berechne die Kohomologie von~$S^1$ mit Werten in der konstanten Garbe~$\ul{\ZZ}$.

{\tiny\emph{Hinweis:} Verwende eine Überdeckung durch drei offene Mengen.\par}

\item Sei~$X = \AA^2_k \setminus \{0\}$ die punktierte Ebene.
Berechne die Kohomologie von~$\O_X$.

{\tiny\emph{Hinweis:} Obwohl~$X$ ein Schema ist, muss man kaum etwas von
Schematheorie wissen, um die Kohomologie zu berechnen. Verwende die
Überdeckung~$X = D(x) \cup D(y)$; den Isomorphismus~$D(x) \cong \Spec
k[x,y,1/x]$; und das nichttriviale Resultat, dass die höhere Kohomologie von
quasikohärenten Modulgarben (wie~$\O_X$) auf affinen Schemata
verschwindet (die nullte Kohomologie von~$\O_{\Spec A}$ ist~$A$).\par}
\end{enumerate}
\end{aufgabe}

\begin{aufgabe}{Die Frölicher-Spektralsequenz}
Sei~$X$ eine komplexe Mannigfaltigkeit. Sei~$\Omega_X^p$ die Garbe der
holomorphen~$p$-Formen auf~$X$; also ist~$\Omega_X^0$ die Garbe der holomorphen
Funktionen auf~$X$. Wegen des \emph{holomorphen Poincaré-Lemmas} ist eine
Auflösung der konstanten Garbe~$\ul{\CC}$ auf~$X$ durch
\[ 0 \lra \ul{\CC} \lra \Omega_X^0 \lra \Omega_X^1 \lra \Omega_X^2 \lra \cdots \]
gegeben. Konstruiere eine Spektralsequenz $E_1^{pq} = \mathbb{H}^{p+q}(X,
\Omega_X^\bullet) \Rightarrow H^{p+q}(X,\ul{\CC})$.

{\tiny\emph{Tipp:} Verwende die naive Filtrierung~$F^p \Omega_X^\bullet =
\Omega_X^{\geq p}$. \emph{Hinweis:} Ist~$X$ eine kompakte
Kähler-Mannigfaltigkeit, so degeneriert die Spektralsequenz auf der ersten
Seite (das ist nichttrivial) und weist die \emph{Hodge-Zerlegung} $H^n(X,C) = \oplus_{p+q=n}
H^p(X,\Omega_X^q)$ nach.\par}
\end{aufgabe}

\begin{aufgabe}{Mayer--Vietoris für abgeschlossene Überdeckungen}
Sei~$X = A \cup B$ eine abgeschlossene Überdeckung eines topologischen
Raums~$X$. Sei~$\E$ eine Garbe abelscher Gruppen auf~$X$. Leite eine
Mayer--Vietoris-Sequenz her, die die Kohomologie von~$\E$ mit der Kohomologie
der zurückgezogenen Garben~$\E|_A$,~$\E|_B$ und~$\E|_{A \cap B}$ in Verbindung
setzt.

{\tiny\emph{Tipp:} Aus der gegebenen Überdeckung kann man nicht (etwa durch
Komplementbildung) eine offene erhalten und dann die gewöhnliche
Mayer--Vietoris-Sequenz verwenden. Konstruiere stattdessen eine kurze exakte
Sequenz~$0 \to \E \to i_* i^{-1} \E \oplus j_* j^{-1} \E \to k_* k^{-1} \E \to
0$, wobei~$i$,~$j$ und~$k$ die Inklusionen von~$A$, $B$ und~$A \cap B$ in~$X$
sind. Beachte~$\E|_A \defeq i^{-1} \E$ und~$H^\bullet(A,\F) \cong
H^\bullet(X,i_*\F)$.\par}
\end{aufgabe}

\end{document}

http://www.math.mcgill.ca/goren/SeminarOnCohomology/specseq.pdf

http://www.math.columbia.edu/~thaddeus/seattle/voisin.pdf
http://wiki.epfl.ch/kaehler2013/documents/Notes/Notes%205%20Holomorphic%20de%20Rham%20Complexes%20and%20Spectral%20Sequences.pdf
