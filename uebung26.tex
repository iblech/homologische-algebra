\documentclass{uebblatt}
\haiitrue

\begin{document}

\maketitle{26}{}

\begin{aufgabe}{Ein universelles Koeffiziententheorem}
Sei~$M$ ein~$R$-Modul. Sei~$P^\bullet$ ein in nichtpositiven Graden
konzentrierter Komplex flacher~$R$-Moduln.
\begin{enumerate}
\item Konstruiere eine Spektralsequenz $E_2^{pq} =
\Tor_{-p}^R(H^q(P^\bullet),M) \Rightarrow H^{p+q}(P^\bullet \otimes_R M)$.
\item Sei~$R$ sogar ein Integritätsbereich. Extrahiere für~$n \geq 0$ aus der
Spektralsequenz eine kurze exakte Sequenz der Form
\[ 0 \lra H^{-n}(P^\bullet) \otimes_R M \lra H^{-n}(P^\bullet \otimes_R M)
  \lra \Tor_1^R(H^{-n+1}(P^\bullet), M) \lra 0. \]
{\tiny\emph{Tipp:} Zeige, dass die Spektralsequenz im Bereich~$p \in \{ 0,-1
\}, q \leq 0$ konzentriert ist. Für solche Sequenzen hat die
kanonische exakte Sequenz $0 \to F^0 E_\infty^{-n} \to F^{-1}
E_\infty^{-n} \to F^{-1} E_\infty^{-n} / F^0 E_\infty^{-n} \to 0$ die Form
$0 \to E_2^{0,-n} \to E_\infty^{-n} \to E_2^{-1,-n+1} \to 0$.\par}
\end{enumerate}
\end{aufgabe}

\begin{aufgabe}{Die Frölicher-Spektralsequenz}
Sei~$X$ eine komplexe Mannigfaltigkeit. Sei~$\Omega_X^p$ die Garbe der
holomorphen~$p$-Formen auf~$X$; also ist~$\Omega_X^0$ die Garbe der holomorphen
Funktionen auf~$X$. Wegen des \emph{holomorphen Poincaré-Lemmas} ist eine
Auflösung der konstanten Garbe~$\ul{\CC}$ auf~$X$ durch
\[ 0 \lra \ul{\CC} \lra \Omega_X^0 \lra \Omega_X^1 \lra \cdots \]
gegeben. Konstruiere eine Spektralsequenz $E_1^{pq} = \mathbb{H}^{p+q}(X,
\Omega_X^\bullet) \Rightarrow H^{p+q}(X,\ul{\CC})$.

{\tiny\emph{Tipp:} Verwende die naive Filtrierung~$F^p \Omega_X^\bullet =
\Omega_X^{\geq p}$. \emph{Hinweis:} Ist~$X$ eine kompakte
Kähler-Mannigfaltigkeit, so degeneriert die Spektralsequenz auf der ersten
Seite und weist die \emph{Hodge-Zerlegung} $H^n(X,C) = \oplus_{p+q=n}
H^p(X,\Omega_X^q)$ nach.\par}
\end{aufgabe}

Mayer--Vietoris für Garben

Garbenkohomologie des~$\PP^n$

\end{document}

http://www.math.columbia.edu/~thaddeus/seattle/voisin.pdf
http://wiki.epfl.ch/kaehler2013/documents/Notes/Notes%205%20Holomorphic%20de%20Rham%20Complexes%20and%20Spectral%20Sequences.pdf
