\documentclass{uebblatt}
\usepackage{wrapfig}
\haiitrue

\begin{document}

\maketitle{20}{}

\begin{aufgabe}{Beispiele für Ext-Gruppen}
\begin{enumerate}
\item Seien~$A$ und~$B$ abelsche Gruppen. Sei~$U$ eine Untergruppe
von~$A$. Sei~$f : U \to B$ ein Gruppenhomomorphismus. Formuliere und
verifiziere ein hinreichendes und notwendiges Kriterium dafür, dass sich~$f$ zu
einem Gruppenhomomorphismus~$\overline{f} : A \to B$ fortsetzen lässt, in
dem~$\Ext^1_\ZZ(A/U,B)$ vorkommt.
\item Sei~$0 \to X \to Y \to Z \to 0$ eine kurze exakte Sequenz. Formuliere und
verfiziere ein hinreichendes und notwendiges Kriterium dafür, dass diese
Sequenz zerfällt, in dem~$\Ext^1(Z,X)$ vorkommt.
\item Sei~$A$ eine abelsche Gruppe. Zeige: $\Ext^n_\ZZ(\ZZ, A) = 0$ für alle~$n > 0$.
\item Sei~$A$ eine abelsche Torsionsgruppe. Zeige: $\Ext^1_\ZZ(A,\ZZ) \cong
\Hom_\ZZ(A, \QQ/\ZZ)$.
\item Zeige: $\Ext^1_\ZZ(\ZZ/(m), \ZZ/(n)) \cong \ZZ/(m,n)$.
\end{enumerate}
%{\scriptsize \emph{Tipp:} Verwende die klassische Definition der Ext-Funktoren:
%$\Ext^n_\ZZ(M,N) \cong H^n(\Hom_\ZZ(P^\bullet,N)) \cong
%H^n(\Hom_\ZZ(M,I^\bullet))$ für eine projektive Auflösung~$P^\bullet \to M$
%oder eine injektive Auflösung~$N \to I^\bullet$. Außerdem kannst du die lange
%exakte Sequenz für die Ext-Gruppen verwenden: Ist~$0 \to N' \to N \to N \to 0$
%eine kurze exakte Sequenz, so ist die induzierte Sequenz~$\cdots \to
%\Ext^n(M,N') \to \Ext^n(M,N) \to \Ext^n(M,N'') \to \Ext^{n+1}(M,N) \to \cdots$
%exakt (und analog für Sequenzen im ersten Argument).\par
%}
\end{aufgabe}

\begin{aufgabe}{Kohomologischer Kleber}
Seien $X$ und~$Y$ Objekte einer abelschen Kategorie~$\A$.
\begin{enumerate}
\item Sei~$\eta : Y[0] \to X[2]$ ein Morphismus in~$\D(\A)$ und~$C^\bullet$ ein Kegel
von~$\eta$. \\ Zeige:~$H^{-2}(C^\bullet) \cong X$,~$H^{-1}(C^\bullet)
\cong Y$ und die restliche Kohomologie verschwindet.
\item Sei~$C^\bullet$ ein Komplex mit~$H^{-2}(C^\bullet) \cong
X$,~$H^{-1}(C^\bullet) \cong Y$ und restlicher Kohomologie Null. Zeige, dass~$C^\bullet$
ein Kegel eines Morphismus~$\eta : Y[0] \to X[2]$ ist.
\item Ziehe das Fazit: Komplexe mit Kohomologie wie in Teilaufgabe~b) sind
bis auf Isomorphie eindeutig durch~$H^{-2}$, $H^{-1}$ und \emph{kohomologischen
Kleber} gegeben.
\end{enumerate}
%{\scriptsize \emph{Tipp:} Verwende für den ersten Teil die lange exakte Sequenz in
%Kohomologie: Ist~$A^\bullet \to B^\bullet \to C^\bullet \to$ ein
%ausgezeichnetes Dreieck in~$\D(\A)$, so ist die induzierte Sequenz~$\cdots \to
%H^n(A^\bullet) \to H^n(B^\bullet) \to H^n(C^\bullet) \to H^{n+1}(A^\bullet) \to \cdots$
%exakt. Verwende für den zweiten Teil die kanonische Filtrierung (Blatt 19,
%Aufgabe 5).\par}
\end{aufgabe}

\begin{aufgabe}{Kein kohomologischer Kleber}
Sei~$\A$ eine abelsche Kategorie mit~$\Ext^n(X,Y) = 0$ für alle Objekte~$X$ und~$Y$
und alle~$n \geq 2$. Zeige, dass jeder beschränkte Komplex~$K^\bullet$
in~$\D^b(\A)$ isomorph zu seinem Kohomologiekomplex~$H^\bullet(K^\bullet)$ (mit
Nulldifferentialen) ist.

{\scriptsize \emph{Hinweis:} Verwende ohne Beweis, dass ein ausgezeichnetes
Dreieck der Form~$A^\bullet \to B^\bullet \to C^\bullet \to$, wobei der
Morphismus~$C^\bullet \to A^\bullet[1]$ Null ist, \emph{zerfällt} und daher
insbesondere~$B^\bullet$ isomorph zu~$A^\bullet \oplus C^\bullet$ ist. Das
werden wir in angemessener Allgemeinheit später beweisen.
\emph{Tipp:}
Führe einen Induktionsbeweis über die Amplitude von~$K^\bullet$
(was kann das wohl sein?) und verwende die kanonische Filtrierung (Blatt 19,
Aufgabe 5).\par}
\end{aufgabe}

\begin{aufgabe}{Homotopie und Pfadobjekt}
\begin{wrapfigure}{r}{0.05\textwidth}
$\xymatrix{
  & L \\
  K \ar[ru]^f \ar[rd]_g \ar[r] & L^I \ar[d]^{\mathrm{ev}_1}
  \ar[u]_{\mathrm{ev}_0} \\
  & L
}$
\end{wrapfigure}
Sei~$L$ ein Komplex. Das \emph{Pfadobjekt}~$L^I$ ist durch~$(L^I)^n =
L^n \oplus L^{n-1} \oplus L^n$ und~$d(u,p,v) = (du, -dp-u-v, dv)$ definiert.
\begin{enumerate}
\item Definiere zwei kanonische Morphismen $\mathrm{ev}_0, \mathrm{ev}_1 : L^I \to L$.
\item Zeige, dass Homotopien~$h : f \simeq g$ von Komplexmorphismen~$f, g : K \to L$
in kanonischer Eins-zu-Eins-Korrespondenz zu kommutativen Diagrammen wie
abgebildet stehen. Was bedeutet das anschaulich?
\end{enumerate}
\end{aufgabe}

\end{document}

\begin{aufgabe}{Bewahrung von Injektiven}
Beweise, dass additive Funktoren, die einen linksexakten Linksadjungierten
besitzen, injektive Objekte bewahren.
\end{aufgabe}
