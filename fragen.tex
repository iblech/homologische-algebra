\documentclass{uebblatt}

\newcommand{\tipp}[1]{\footnote{\rotatebox{180}{Tipp: #1}}}

\renewcommand*\theenumi{\arabic{enumi}}
\renewcommand{\labelenumi}{\theenumi.}

\begin{document}

\maketitle{$\boldsymbol{\omega}$}{}

\begin{center}
  \emph{-- Themen, die hier noch fehlen --}

  Simpliziale Mengen \textbullet{} Kettenkomplexe \textbullet{} Garben
  \textbullet{} \ldots
\end{center}

\begin{aufgabeE}{Kategorientheorie: Grundlagen}
\item Was ist eine Kategorie, was ein Funktor, was eine natürliche
Transformation?
\item Was sind -- aus drei verschiedenen Teilgebieten der Mathematik --
Beispiele für Kategorien, Funktoren und natürliche Transformationen?
\item Inwieweit verallgemeinern Funktoren Abbildungen zwischen
Mengen?\tipp{Diskrete Kategorien.}
\item Inwieweit verallgemeinern Funktoren monotone Abbildungen zwischen
Quasiordnungen?\tipp{Quasiordnungen induzieren Kategorien.}
\item Inwieweit verallgemeinern Funktoren Gruppen- oder Monoidhomomorphismen?
Was sind in diesem Bild natürliche Transformationen?
\item Unter welchem Trivialnamen sind Kategorien mit nur einem Objekt auch
bekannt?
\item Inwieweit kodieren natürliche Transformationen gleichmäßig definierte
Abbildungsvorschriften?
\item Welche natürlichen Transformationen~$\Id_\Set \to \Id_\Set$ gibt es?
\item Was ist ein Gruppoid?
\item Was sind drei Beispiele für Gruppoide?
\end{aufgabeE}

\begin{aufgabeE}{Kategorientheorie: Verbote}
\item Wieso sollte man Objekte nicht auf Gleichheit testen?
\item Wieso sollte man Funktoren nicht auf Gleichheit testen?
\item Wieso kann man natürliche Transformationen auf Gleichheit testen?
\end{aufgabeE}

\begin{aufgabeE}{Kategorientheorie: Äquivalenzen}
\item Wie kann man von zwei Kategorien feststellen, dass sie nicht zueinander
äquivalent sind?
\item Zu welcher sehr konkreten Kategorie ist die Kategorie der
endlich-dimensionalen~$K$-Vektorräume äquivalent? Wie sieht in diesem Bild der
Dualisierungsfunktor~$\Vect(K) \to \Vect(K)^\op$ aus?
\item Welche Kategorien sind zu diskreten Kategorien äquivalent? (Eine diskrete
Kategorie ist eine, in der jeder Morphismus ein Identitätsmorphismus ist. Was
ist schlecht an dem Konzept einer diskreten Kategorie?)
\item Wieso sind~$\Set$ und~$\Vect(\RR)$ nicht zueinander äquivalent?
\item Wieso sind $\Set$ und~$\Set^\op$ nicht zueinander äquivalent?
\item Wozu ist die Kategorie der (kommutativen)
C\textsuperscript{*}\kern-.1ex-Algebren (mit Eins) äquivalent?
\item Wie kann man die Galoistheorie als Kategorienäquivalenz ausdrücken?
\item Über welche Kategorienäquivalenz ist die Darstellungstheorie von
Fundamentalgruppen (oder besser Fundamentalgruppoiden) eng mit der
Überlagerungstheorie verknüpft?
\item Was ist Pontrjagin-Dualität?
\end{aufgabeE}

\begin{aufgabeE}{Kategorientheorie: Limiten}
\item Was sind Limiten und Kolimiten?
\item Inwieweit sind Produkte Spezialfälle von Limiten?
\item Inwieweit ist der Vektorraum~$K[X]$ aller Polynome ein Kolimes?
\item Wie kann man einer Kategorie ansehen, ob sie alle Limiten besitzt?
\item Inwieweit sind Limiten stets Unterobjekte von Produkten und Kolimiten
stets Quotientenobjekte von Koprodukten?
\item Bei wem muss man sich melden, wenn man nächstes Jahr Zirkelleiter sein
möchte?
\item Welche Limiten existieren in der Kategorie der Mengen? \ldots{} in der
Kategorie der Gruppen? Wieso?
\item Was ist ein Beispiel für eine natürlich auftretende Kategorie, in der
nicht alle Kolimiten existieren?
\item Was haben Limiten mit Darstellbarkeit von Funktoren zu tun?
\end{aufgabeE}

\begin{aufgabeE}{Kategorientheorie: Adjunktionen}
\item Was sind adjungierte Funktorpaare?
\item Was sind Beispiele für Adjunktionen aus drei verschiedenen Teilgebieten
der Mathematik?
\item Wieso bewahren Rechtsadjungierte stets Limiten? (Vielen Dank an Timo
Schürg: RAPL!)
\item Wieso sind Funktoren, die freie Konstruktionen berechnen, stets
Linksadjungierte und nicht Rechtsadjungierte?
\item Wieso gibt es keine freien Körper?
\end{aufgabeE}

\begin{aufgabeE}{Kategorientheorie: Yoneda}
\item Was besagt das Yoneda-Lemma in seiner allgemeinen Formulierung?
\item Wie kann man sich einen Funktor~$\C^\op \to \Set$ anschaulich vorstellen?
Wie sieht unter diesem Bild die Yoneda-Einbettung aus?
\item Was sind drei Beispiele für Anwendungen des Yoneda-Lemmas?
\item Wieso ist der Funktor~$\Hom_\C(X,\placeholder)$ stetig (limesbewahrend)?
\end{aufgabeE}

\begin{aufgabeE}{Geometrie}
\item Was ist ein lokal geringter Raum? Worauf bezieht sich das Adjektiv
"`lokal"'?
\item Was sind drei substanziell verschiedene Beispiele für lokal geringte Räume?
\item Welche Aspekte verallgemeinern lokal geringte Räume im Vergleich zu
glatten Mannigfaltigkeiten?
\item Was ist der wandelnde Tangentialvektor? Wieso heißt er so?
\item Was ist Supergeometrie?
\end{aufgabeE}

\begin{aufgabeE}{Abelsche Kategorien}
\item Welche Möglichkeiten gibt, in~$\Ab$-angereicherten Kategorien das Konzept
des Kerns eines Morphismus zu definieren?
\item Inwieweit ist die abelsche Gruppenstruktur auf den Hom-Mengen einer
abelschen Kategorie kein weiteres Datum? Inwieweit also lässt sie sich
eindeutig aus einer gewissen Eigenschaft rekonstruieren?
\item Welche der folgenden Aussagen über Morphismen in abelschen Kategorien ist
trivial? Monomorphismen sind unter Rückzug stabil. Epimorphismen sind unter
Rückzug stabil.
\item Wie kann man in abelschen Kategorien Diagrammjagden wie in
Modulkategorien führen?
\item Wann heißt eine kurze exakte Sequenz zerfallend? Welche speziellen
Objekte führen automatisch dazu, dass eine Sequenz zerfällt?
\item Wie kann man~$\Ext^1$ über kurze exakte Sequenzen verstehen? Was ist das
Nullelement? Wie sieht die Gruppenstruktur aus?
\item Wie kann man in einer bestimmten Ext-Gruppe entscheiden, ob ein gegebener
Morphismus fortsetzbar ist auf ein Oberobjekt?
\item Was ist eine Serresche Unterkategorie und wie verwendet man sie, um
Serresche Quotientenkategorien zu konstruieren?
\end{aufgabeE}

\begin{aufgabeE}{Garbentheorie}
\item Was ist die universelle Eigenschaft der Garbifizierung?
\item Wie konstruiert man sie?
\item Wie definiert man den Rückzug von Garben?
\item Was weiß man über die Halme zurückgezogener Garben? Wie beweist man das?
\item Wie definiert man den Pushforward von Garben?
\item Unter welchen Voraussetzungen kann man etwas über die Halme vorgedrückter
Garben sagen?
\item Wann ist eine Sequenz von Garben abelscher Gruppen exakt?
\item Ist Vordrücken von Garben abelscher Gruppen ein exakter Funktor? Wie
steht es um den Rückzug?
\item Wie kann die Kategorie der Garben als Lokalisierung der Kategorie der
Prägarben verstanden werden?
\end{aufgabeE}

\begin{aufgabeE}{$K$-Theorie}
\item Was ist die~$K$-Theorie einer abelschen Kategorie?
\item Was ist die~$K$-Theorie der Kategorie der endlich-dimensionalen
Vektorräume über einem Körper?
\item Und was ist die der Kategorie der abelschen Gruppen?
\end{aufgabeE}

\begin{aufgabeE}{Abgeleitete Kategorien}
\item Was ist eine Voraussetzung an eine abelsche Kategorie~$\A$, die
garantiert, dass man~$D^+(\A)$ im selben mengentheoretischen Universum wie~$\A$
konstruieren kann?
\item Wie lautet die universelle Eigenschaft der abgeleiteten Kategorie
genau?
\item Und wie lautet die eines abgeleiteten Funktors?
\item Welche wichtige Tatsache über injektive Objekte geht in den Beweis der
Äquivalenz~$D^+(\A) \simeq K^+(\I)$ ein?
\item Inwieweit wird das Motto, Auflösungen eines Objekts seien genauso gut wie
das aufgelöste Objekt selbst, in der abgeleiteten Kategorie gelöst?
\item Bis auf was sind injektive Auflösungen eines Objekts gleich?
\item Wann ist die abgeleitete Kategorie wieder abelsch?
\item Unter welchen Voraussetzungen ist jeder Komplex in der abgeleiteten
Kategorie isomorph zu seinem Kohomologiekomplex?
\item Wieso invertiert man, angesichts der vorherigen Frage, in der Definition
der abgeleiteten Kategorie nicht nur die Homotopieäquivalenzen?
\item Was ist die dumme Abschneidung eines Komplexes? Was die gute?
\item Welche drei Definitionen des Ext-Funktoren gibt es?
\item Welche wichtigen ausgezeichneten Dreiecke erhält man über die beiden
Abschneidungen?
\item In welchen Fällen kann man die Kohomologie des Kegels eines Morphismus
einfach angeben?
\item Was ist ein Beispiel für eine Kategorie, die nicht genügend viele
Projektive besitzt?
\item Wie konstruiert man abgeleitete Funktoren?
\end{aufgabeE}

\end{document}

Eigenschaften vom Zylinder & Co.
Ausgz. dreiecke

XXX: Titel
