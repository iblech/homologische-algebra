\documentclass{uebblatt}

\UseAllTwocells

\begin{document}

\maketitle{9}{}

\begin{aufgabe}{Funktoren als verallgemeinerte monotone Abbildungen}
Seien~$P$ und~$Q$ Quasiordnungen (oder Partialordnungen). Seien~$BP$ und~$BQ$
die zugehörigen dünnen Kategorien, deren Objekte genau die Elemente von~$P$ bzw.~$Q$
sind und in denen zwischen zwei Objekten genau dann ein Morphismus verläuft,
wenn die Quelle kleinergleich dem Ziel ist.

\begin{enumerate}
\item Zeige, dass Funktoren~$BP \to BQ$ auf kanonische Art und Weise mit
schwach monoton steigenden Abbildungen~$P \to Q$ korrespondieren.
\item Zeige, dass zwischen zwei Funktoren~$Bf, Bg : BP \to BQ$ höchstens eine
natürliche Transformation verlaufen kann; und dass es genau dann eine solche
gibt, wenn für die zugehörigen monotonen Abbildungen~$f, g : P \to Q$ gilt:
$f(x) \preceq g(x)$ für alle~$x \in P$.
\end{enumerate}
\end{aufgabe}

\begin{aufgabe}{Beispiele für natürliche Transformationen}
Sei~$\Id_\Set : \Set \to \Set$ der Identitätsfunktor auf~$\Set$, $P : \Set \to
\Set$ der (kovariante) Potenzmengenfunktor und~$K : \Set
\to \Set$ der Funktor
\[ \begin{array}{@{}rrcl@{}}
  & X &\longmapsto& X \times X \\
  & f &\longmapsto& f \times f := ((a,b) \mapsto (f(a),f(b))).
\end{array} \]
\begin{enumerate}
\item
Zeige: Es gibt nur eine einzige natürliche Transformation $\eta : \Id_\Set
\to \Id_\Set$, nämlich
\[ \eta_X : X \to X,\ x \mapsto x. \]
\item
Zeige: Es gibt nur eine einzige natürliche Transformation $\omega : \Id_\Set
\to K$, nämlich
\[ \omega_X : X \to X \times X,\ x \mapsto (x,x). \]
\emph{Tipp für a) und b):} Betrachte geeignete Abbildungen $1 \to X$, $\star
\mapsto x$. Dabei ist~$1 = \{\star\}$ eine einelementige Menge.
\item
Zeige: Es gibt keine natürliche Transformation $P \to \Id_\Set$, wohl
aber eine in die andere Richtung.
\item Wir nehmen an, dass wir für jede nichtleere Menge~$X$ ein bestimmtes
Element~$a_X \in X$ gegeben haben. Zeige:
Die Setzung
$\tau_X : X \to X,\ x \mapsto a_X$
definiert \emph{nicht} eine natürliche Transformation~$\Id_\C \to \Id_\C$,
wobei~$\C$ die Kategorie der nichtleeren Mengen und beliebigen Abbildungen
bezeichnet.
\item
Welche natürlichen Transformationen $\Id_\C \to \Id_\C$ gibt es,
wenn~$\C$ die Kategorie der reellen Vektorräume bezeichnet?
\end{enumerate}
\end{aufgabe}

\newpage

\begin{aufgabe}{Die 2-Kategorie der Kategorien}
\begin{enumerate}
\item Seien~$\eta : F \to G$ und~$\varepsilon : G \to H$ natürliche
Transformationen zwischen Funktoren~$F,G,H : \C \to \D$. Definiere auf
geeignete Art und Weise die \emph{vertikale Komposition}~$\varepsilon \circ
\eta : F \to H$. Weise nach, dass deine Definition wirklich zu einer
natürlichen Transformation führt.
\item Sei~$\eta : F \to G$ eine natürliche Transformation zwischen
Funktoren~$F,G : \C \to \D$ und~$\varepsilon : J \to K$ eine natürliche
Transformation zwischen Funktoren~$J,K : \D \to \E$. Definiere auf geeignete
Art und Weise die \emph{horizontale Komposition}~$\eta \star \varepsilon : J
\circ F \to K \circ G$. Musst du dazu Wahlen treffen?
\item Verifiziere für passende natürliche Transformationen folgendes
Vertauschungsgesetz:
\[ (\beta' \circ \beta) \star (\alpha' \circ \alpha) =
  (\beta' \star \alpha') \circ (\beta \star \alpha) \]
\end{enumerate}
\[
  \xymatrixcolsep{3pc}
  \xymatrixrowsep{4pc}
  \xymatrix@+=5pc{
    \C
    \ruppertwocell<9>^{\stackrel{F}{}}{<-2.5>_{\mbox{   } \eta}} \ar[r]|G
    \rlowertwocell<-9>_{\stackrel{}{H}}{<2.5>^{\mbox{  }\varepsilon}} & \D &
    \C \rtwocell<6>^F_G{\;\;\eta} &
    \D \rtwocell<6>^J_K{\;\;\varepsilon} &
    \E
  }
\]
\end{aufgabe}

\begin{aufgabe}{Überraschende Kommutativität}
Zeige, dass die zweite Homotopiegruppe stets kommutativ ist.

\emph{Diese Aufgabe wird noch ausgebaut, sodass sie etwas mit 2-Kategorien zu
tun hat.}
\end{aufgabe}

\end{document}
