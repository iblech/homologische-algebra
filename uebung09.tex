\documentclass{uebblatt}

\UseAllTwocells

\begin{document}

\maketitle{9}{}

\begin{aufgabe}{Funktoren als verallgemeinerte monotone Abbildungen}
Seien~$P$ und~$Q$ Quasiordnungen (oder Partialordnungen). Seien~$BP$ und~$BQ$
die zugehörigen dünnen Kategorien, deren Objekte genau die Elemente von~$P$ bzw.~$Q$
sind und in denen zwischen zwei Objekten genau dann ein Morphismus verläuft,
wenn die Quelle kleinergleich dem Ziel ist.

\begin{enumerate}
\item Zeige, dass Funktoren~$BP \to BQ$ auf kanonische Art und Weise mit
schwach monoton steigenden Abbildungen~$P \to Q$ korrespondieren.
\item Zeige, dass zwischen zwei Funktoren~$Bf, Bg : BP \to BQ$ höchstens eine
natürliche Transformation verlaufen kann; und dass es genau dann eine solche
gibt, wenn für die zugehörigen monotonen Abbildungen~$f, g : P \to Q$ gilt:
$f(x) \preceq g(x)$ für alle~$x \in P$.
\end{enumerate}
\end{aufgabe}

\begin{aufgabe}{Beispiele für natürliche Transformationen}
Sei~$\Id_\Set : \Set \to \Set$ der Identitätsfunktor auf~$\Set$, $P : \Set \to
\Set$ der (kovariante) Potenzmengenfunktor und~$K : \Set
\to \Set$ der Funktor
\[ \begin{array}{@{}rrcl@{}}
  & X &\longmapsto& X \times X \\
  & f &\longmapsto& f \times f := ((a,b) \mapsto (f(a),f(b))).
\end{array} \]
\begin{enumerate}
\item
Zeige: Es gibt nur eine einzige natürliche Transformation $\eta : \Id_\Set
\to \Id_\Set$, nämlich
\[ \eta_X : X \to X,\ x \mapsto x. \]
\item
Zeige: Es gibt nur eine einzige natürliche Transformation $\omega : \Id_\Set
\to K$, nämlich
\[ \omega_X : X \to X \times X,\ x \mapsto (x,x). \]
\emph{Tipp für a) und b):} Betrachte geeignete Abbildungen $1 \to X$, $\star
\mapsto x$. Dabei ist~$1 = \{\star\}$ eine einelementige Menge.
\item
Zeige: Es gibt keine natürliche Transformation $P \to \Id_\Set$, wohl
aber eine in die andere Richtung.
\item Wir nehmen an, dass wir für jede nichtleere Menge~$X$ ein bestimmtes
Element~$a_X \in X$ gegeben haben. Zeige:
Die Setzung
$\tau_X : X \to X,\ x \mapsto a_X$
definiert \emph{nicht} eine natürliche Transformation~$\Id_\C \to \Id_\C$,
wobei~$\C$ die Kategorie der nichtleeren Mengen und beliebigen Abbildungen
bezeichnet.
\item
Welche natürlichen Transformationen $\Id_\C \to \Id_\C$ gibt es,
wenn~$\C$ die Kategorie der reellen Vektorräume bezeichnet?
\end{enumerate}
\end{aufgabe}

\newpage

\begin{aufgabe}{Die 2-Kategorie der Kategorien}
\begin{enumerate}
\item Seien~$\eta : F \to G$ und~$\varepsilon : G \to H$ natürliche
Transformationen zwischen Funktoren~$F,G,H : \C \to \D$. Definiere auf
geeignete Art und Weise die \emph{vertikale Komposition}~$\varepsilon \circ
\eta : F \to H$. Weise nach, dass deine Definition wirklich zu einer
natürlichen Transformation führt.
\item Sei~$\eta : F \to G$ eine natürliche Transformation zwischen
Funktoren~$F,G : \C \to \D$ und~$\varepsilon : J \to K$ eine natürliche
Transformation zwischen Funktoren~$J,K : \D \to \E$. Definiere auf geeignete
Art und Weise die \emph{horizontale Komposition}~$\eta \star \varepsilon : J
\circ F \to K \circ G$. Musst du dazu Wahlen treffen?
\item Verifiziere für passende natürliche Transformationen folgendes
Vertauschungsgesetz:
\[ (\beta' \circ \beta) \star (\alpha' \circ \alpha) =
  (\beta' \star \alpha') \circ (\beta \star \alpha) \]
\end{enumerate}
Eine gewöhnliche Kategorie heißt auch~\emph{1-Kategorie}; eine
\emph{strikte 2-Kategorie} ist eine Kategorie, in der die
Hom-Mengen~$\Hom(X,Y)$ nicht nur Mengen, sondern ihrerseits (gewöhnliche
1-)Kategorien sind, und in der die Verknüpfungsoperationen~$\Hom(Y,Z) \times
\Hom(X,Y) \to \Hom(X,Z)$ sogar Funktoren sind. Zur besseren Abgrenzung heißen
die Objekte~$f \in \Hom(X,Y)$ dann~$1$-Morphismen (zwischen~$X$ und~$Y$) und die
Morphismen~$\alpha : f \to g$ in~$\Hom(X,Y)$ dann~$2$-Morphismen (zwischen~$f$
und~$g$).
\begin{enumerate}
\addtocounter{enumi}{3}
\item
Zeige, dass sich 1-Kategorien,
Funktoren und natürliche Transformationen zu einer 2-Kategorie organisieren.
\end{enumerate}
\[
  \xymatrixcolsep{3pc}
  \xymatrixrowsep{4pc}
  \xymatrix@+=5pc{
    \C
    \ruppertwocell<9>^{\stackrel{F}{}}{<-2.5>_{\mbox{   } \eta}} \ar[r]|G
    \rlowertwocell<-9>_{\stackrel{}{H}}{<2.5>^{\mbox{  }\varepsilon}} & \D &
    \C \rtwocell<6>^F_G{\;\;\eta} &
    \D \rtwocell<6>^J_K{\;\;\varepsilon} &
    \E
  }
\]
\end{aufgabe}

\begin{aufgabe}{Überraschende Kommutativität}
Die \emph{erste Homotopiegruppe}~$\pi_1(X,x_0)$ eines topologischen Raums~$X$
mit Basispunkt~$x_0$ ist die Menge aller stetigen Abbildungen~$\gamma : [0,1]
\to X$ mit~$\gamma(0) = \gamma(1) = x_0$, wobei zwei solche Abbildungen genau
dann miteinander identifiziert werden, wenn sie vermöge einer
basispunktfixierenden Abbildung~$[0,1] \times [0,1] \to X$ homotop sind.

Die \emph{zweite Homotopiegruppe}~$\pi_2(X,x_0)$ ist
die Menge aller stetigen Abbildungen~$H : [0,1]^2 \to X$, die den Rand des
Einheitsquadrats auf den Basispunkt~$x_0$ abbilden, wobei zwei solche
Abbildungen~$H, \widetilde H$ genau dann miteinander identifiziert werden, wenn sie vermöge
einer stetigen Abbildung~$K : [0,1] \times [0,1]^2 \to X$
mit~$K(0,\placeholder) = H$, $K(1,\placeholder) = \widetilde H$,
$K(\placeholder,\partial [0,1]^2) = \text{konst. $x_0$}$ homotop
sind.

\begin{enumerate}
\item Finde \emph{zwei} kanonische Gruppenstrukturen auf~$\pi_2(X,x_0)$.

\emph{Tipp:} Das hat etwas damit zu tun, wie man das Quadrat horizontal und
vertikal in zwei Rechtecke gleicher Größe aufteilen kann.

\item Beweise, dass die beiden Gruppenstrukturen dasselbe Vertauschungsgesetz
erfüllen wie in Aufgabe~3c).
\item Zeige, dass die beiden Gruppenstrukturen übereinstimmen und kommutativ
sind.

\emph{Tipp:} Das gilt allgemein -- zwei binäre Operationen mit neutralem
Element, die wie in Aufgabe~3c) miteinander verträglich sind, sind tatsächlich
gleich und kommutativ.

\item Sei~$A$ ein Objekt einer~2-Kategorie~$\C$. Sei~$\id_A : A \to A$ der
(1-)Identitätsmorphismus. Sei~$Z(A) := \End(\id_A)$ die Menge
aller~2-Morphismen von~$\id_A$ nach~$\id_A$.

Finde zwei Monoidstrukturen auf~$Z(A)$, zeige, dass sie miteinander
verträglich sind, und folgere daher, dass sie gleich und kommutativ sind.

\item Der \emph{Fundamental-2-Gruppoid}~$\Pi_2(X)$ ist folgende 2-Kategorie:
Die Objekte sind die Punkte von~$X$, die Morphismen sind Wege zwischen den
Punkten und die~$2$-Morphismen zwischen Wegen~$\gamma, \widetilde\gamma$ sind
Homotopieklassen von stetigen Abbildungen~$H : [0,1] \times [0,1] \to X$
mit~$H(0,\placeholder) = \gamma$, $H(1,\placeholder) = \widetilde\gamma$.

Zeige, dass $Z(x_0) = \pi_2(X,x_0)$. Folgere mit dieser Erkenntnis
Teilaufgabe~c) aus~d).
\end{enumerate}
\end{aufgabe}

\begin{aufgabe}{Eine Kategorie mit endlichen Mengen als Objekten}
Zeige, dass folgende Setzungen eine Kategorie~$\Sigma$ definieren. Bestätige
also das Assoziativgesetz und finde die Identitätsmorpismen.
\begin{align*}
  \text{Objekte: } & \text{alle endlichen Mengen} \\
  \text{Morphismen: } &
    \Hom(X,Y) := \bigl\{
      \text{Abbildungen~$f : X \to Y$ zusammen mit} \\
    & \qquad\qquad\qquad\qquad \text{
      Totalordnungen auf den Fasern~$f^{-1}[\{y\}]$, $y \in Y$} \bigr\}
\end{align*}
Die Komposition soll dabei wie folgt definiert sein: Die Abbildungsteile
verkettet man wie gewöhnlich, und auf den Fasern~$(g \circ f)^{-1}[\{z\}]$
definiert man folgende Totalordnung: $i \preceq j$ genau dann, wenn entweder
$f(i) \neq f(j)$ und~$f(i) \preceq f(j)$ in~$g^{-1}[\{f(i)\}]$, oder $f(i) =
f(j)$ und~$i \preceq j$ in~$f^{-1}[\{f(i)\}]$.
\end{aufgabe}

\end{document}
