\documentclass{uebblatt}
\usepackage{calc}

\begin{document}

\maketitle{8}{}

\begin{aufgabe}{Prismentriangulierung}
In der Vorlesung wurde~$\Delta_p \times I$ durch
Simplizes~$\delta_{p\ell}[\Delta_{p+1}]$ trianguliert (Seite~50f. in
Gelfand--Manin). Sei~$\delta_p = \sum_{\ell=0}^{p+1}
(-1)^\ell \delta_{p\ell}$, seien $\varepsilon_0$ und~$\varepsilon_1 : \Delta_p \to
\Delta_p \times I$ die singulären~$p$-Simplizes
mit~$\varepsilon_i(\lambda) = (\lambda,i)$ und sei
\[ \delta^{(i)}_{p-1} = \sum_{\ell = 0}^p (-1)^\ell
  (\Delta_{\partial^i_p} \times \id_I) \circ \delta_{p-1,\ell} :
  \Delta_p \to \Delta_p \times I \]
eine Darstellung der~$i$-ten Seite von~$\Delta_p \times I$ als singuläre Kette.
Die Abbildung $\Delta_{\partial^i_p}$ hat dabei den Typ $\Delta_{p-1} \to
\Delta_p$ und wird von der Korandabbildung~$\partial^i_p : [p-1] \to [p]$ induziert.

\begin{enumerate}
\item Interpretiere folgende Identität anschaulich und beweise sie:
\[ d \delta_p = -\sum_{i=0}^p (-1)^i \delta_{p-1}^{(i)} + \varepsilon_1 -
\varepsilon_0 \]

\item Inwieweit spielt sich beim Beweis, dass homotope stetige Abbildungen
kettenhomotope Komplexmorphismen induzieren, eigentlich alles auf dem
Standardprisma ab? Inwieweit genügt es, die Behauptung für diesen Fall zu
beweisen?
\end{enumerate}
\end{aufgabe}

\begin{aufgabe}{Viele Homotopiebegriffe}
In der Vorlesung wurde bewiesen, dass zueinander homotope stetige Abbildungen
zueinander homotope Morphismen zwischen den zugehörigen singulären
Kettenkomplexen induzieren. In dieser Aufgabe möchten wir verstehen, dass
dieses Resultat tatsächlich über zwei kleinere und unabhängig voneinander
nützliche Beobachtungen faktorisiert.
\begin{enumerate}
\item Seien~$f, g : X \to Y$ zueinander homotope stetige Abbildungen zwischen
topologischen Räumen. Diese induzieren bekanntlich simpliziale Abbildungen~$Sf,
Sg : SX \to SY$ zwischen den zugehörigen singulären simplizialen Mengen. Zeige,
dass diese im Sinn von Aufgabe~4 von Blatt~4 zueinander homotop (sogar einfach
homotop) sind.

\emph{Zur Erinnerung:} $(SX)_n = \Hom_\Top(\Delta_n,X)$.
\emph{Tipp:} Dieses Ergebnis ist recht formaler Natur. Du wirst keine
Triangulierungen von Prismen benötigen.

\item Seien~$\varphi, \psi : X \to Y$ zueinander homotope simpliziale
Abbildungen zwischen simplizialen Mengen. Zeige, dass für beliebige
Koeffizientengruppen~$A$ die induzierten
Kettenkomplexmorphismen~$C_\bullet(X,A) \to C_\bullet(Y,A)$ zueinander
kettenhomotop sind.

\emph{Tipp:} Vielleicht ist es einfacher, die Kettenhomotopie der induzierten
Morphismen zwischen den normalisierten Moore-Komplexen (siehe Aufgabe~2 von
Blatt~7) nachzuweisen. Wieso genügt das?
\end{enumerate}
\end{aufgabe}

\newpage

\begin{aufgabe}{Der Hochschild-Komplex}
Sei~$k$ ein kommutativer Ring (mit Eins),~$A$ eine assoziative (aber nicht
unbedingt kommutative)~$k$-Algebra und~$M$ ein~$A$-$A$-Bimodul (was ist das?).
Wir statten die~$k$-Moduln~$C_n(A,M) := M \otimes A^{\otimes n}$ (alle
Tensorprodukte über~$k$) mit den Randabbildungen
\begin{align*}
  d^0(m \otimes a_1 \otimes \cdots \otimes a_n) &:=
    ma_1 \otimes a_2 \otimes \cdots \otimes a_n \\
  d^i(m \otimes a_1 \otimes \cdots \otimes a_n) &:=
    m \otimes a_1 \otimes \cdots \otimes a_{i-1} \otimes a_i a_{i+1} \otimes
    a_{i+2} \otimes \cdots \otimes a_n \\
  d^n(m \otimes a_1 \otimes \cdots \otimes a_n) &:=
    a_n m \otimes a_1 \cdots \otimes a_{n-1}
\end{align*}
und dem Differential~$d = \sum_{i=0}^n (-1)^i d^i : C_n(A,M) \to C_{n-1}(A,M)$
aus. Die Homologie von~$C_\bullet(A,M)$ ist die \emph{Hochschild-Homologie} der
Algebra~$A$ mit Koeffizienten in~$M$.
\begin{enumerate}
\item Zeige, dass~$d$ wirklich ein Differential ist.
\item Was ist~$H_0(C_\bullet(A,A))$?
\end{enumerate}
\end{aufgabe}

\begin{aufgabe}{Der Koszul-Komplex}
Sei~$A$ ein kommutativer Ring und~$M$ ein~$A$-Modul.
Seien~$\varphi_1,\ldots,\varphi_p : M \to M$ paarweise kommutierende
Endomorphismen. Auf den~$A$-Moduln~$M^n := M \otimes_\ZZ \Lambda^n \ZZ^p$
definieren wir das Differential
\[ d(x \otimes (e_{j_1} \wedge \cdots \wedge e_{j_n})) :=
  \sum_{j=1}^p \varphi_j(x) \otimes (e_j \wedge e_{j_1} \wedge \cdots \wedge
  e_{j_n}). \]
Dabei ist~$(e_1,\ldots,e_p)$ die kanonische Basis in~$\ZZ^p$ und~$\Lambda^n
\ZZ^p$ die~$p$-te \emph{äußere Potenz} von~$\ZZ^p$.
\begin{enumerate}
\item Zeige, dass diese Setzung wirklich zu einem Differential führt.
\end{enumerate}
Bedeutsam ist der Koszul-Komplex unter anderem aus folgendem Grund:
Sei speziell~$M = A$ und seien die~$\varphi_i$ durch Multiplikation mit
gewissen Elementen~$x_i$ gegeben. Sei die Folge~$(x_1,\ldots,x_n)$
\emph{regulär}, das heißt, dass für~$j = 1,\ldots,p$ das Element~$x_i$ im
Ring~$A/(x_1,\ldots,x_{i-1})$ regulär ist. Dann gilt~$H^p(M^\bullet) \cong
A/(x_1,\ldots,x_n)$ und die sonstige Kohomologie verschwindet. Der
Komplex~$M^\bullet$ definiert daher eine \emph{Auflösung}
von~$A/(x_1,\ldots,x_n)$.
\begin{enumerate}
\addtocounter{enumi}{1}
\item Gib für~$p = 1$ und~$p= 2$ den Koszul-Komplex in diesem Spezialfall
explizit an.
\item Wenn du möchtest, kannst du versuchen, die Aussage über die Kohomologie
zu beweisen. Momentan ist das aber etwas mühevoll, später werden wir
vereinfachende Techniken kennenlernen.
\end{enumerate}
\end{aufgabe}

\begin{aufgabe}{Differentialformen}
\emph{Folgt in Kürze.}
\end{aufgabe}

\end{document}
