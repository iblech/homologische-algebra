\documentclass{uebblatt}
\usepackage{calc}

\begin{document}

\maketitle{8}{}

\begin{aufgabe}{Prisma-Triangulierung}
\ldots Bestätige Formel (I.21) in Gelfand--Manin \ldots

\ldots Inwieweit spielt sich alles nur auf dem Standardprisma ab? \ldots
\end{aufgabe}

\begin{aufgabe}{Viele Homotopiebegriffe}
In der Vorlesung wurde bewiesen, dass zueinander homotope stetige Abbildungen
zueinander homotope Morphismen zwischen den zugehörigen singulären
Kettenkomplexen induzieren. In dieser Aufgabe möchten wir verstehen, dass
dieses Resultat tatsächlich über zwei kleinere und unabhängig voneinander
nützliche Beobachtungen faktorisiert.
\begin{enumerate}
\item Seien~$f, g : X \to Y$ zueinander homotope stetige Abbildungen zwischen
topologischen Räumen. Diese induzieren bekanntlich simpliziale Abbildungen~$Sf,
Sg : SX \to SY$ zwischen den zugehörigen singulären simplizialen Mengen. Zeige,
dass diese im Sinn von Aufgabe~4 von Blatt~4 zueinander homotop (sogar einfach
homotop) sind.

\emph{Zur Erinnerung:} $(SX)_n = \Hom_\Top(\Delta_n,X)$.
\emph{Tipp:} Dieses Ergebnis ist recht formaler Natur. Du wirst keine
Triangulierungen von Prismen benötigen.

\item Seien~$\varphi, \psi : X \to Y$ zueinander homotope simpliziale
Abbildungen zwischen simplizialen Mengen. Zeige, dass für beliebige
Koeffizientengruppen~$A$ die induzierten
Kettenkomplexmorphismen~$C_\bullet(X,A) \to C_\bullet(Y,A)$ zueinander
kettenhomotop sind.

\emph{Tipp:} Das hat etwas mit Triangulierungen von Prismen zu tun.
\end{enumerate}
\end{aufgabe}

\begin{aufgabe}{\ldots}
\ldots Hochschild- und Koszul-Kohomologie \ldots
\end{aufgabe}

\end{document}
