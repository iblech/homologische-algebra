\documentclass{uebblatt}
\haiitrue

\begin{document}

\maketitle{27}{}

\begin{aufgabe}{Kohomologie von~$\RR^1$ mit kompaktem Träger}
\begin{enumerate}
\item Zeige, dass~$0 \to \ul{\RR} \to \C^1 \to
\C \to 0$ eine exakte Sequenz von Garben auf~$\RR^1$ ist. Dabei schickt der Morphismus~$\C^1
\to \C$ eine stetig differenzierbare Funktion auf ihre Ableitung.
\item Wieso sind~$\C$ und~$\C^1$ weiche Garben?
\item Berechne~$H^\bullet_c(\RR^1, \RR)$.
\end{enumerate}
\end{aufgabe}

\begin{aufgabe}{Dimensionstheorie über Kohomologie mit kompaktem Träger}
Die \emph{Dimension} eines topologischen Raums~$X$ ist die kleinste Zahl~$n
\geq 0$, sodass~$H_c^{> n}(X,\E)$ für alle Garben~$\E$ abelscher Gruppen
auf~$X$ verschwindet. Sei im Folgenden~$X$ ein lokal kompakter Hausdorffraum,
der abzählbar im Unendlichen ist.
\begin{enumerate}
\item Sei~$Y \subseteq X$ eine offene oder abgeschlossene Teilmenge.
Zeige~$\dim_c Y \leq \dim_c X$.

{\tiny\emph{Tipp:} Ist~$i : Y \hookrightarrow X$ die Inklusion, so gilt
$H_c^\bullet(Y,\E) \cong H_c^\bullet(X,i_! \E)$.\par}
\item Sei~$0 \to \E \to \L^0 \to \cdots \to \L^{n-1} \to \L^n \to 0$ eine
exakte Sequenz von Garben abelscher Gruppen auf~$X$.
Seien~$\L^0,\ldots,\L^{n-1}$ weich. Sei~$\dim_c X \leq n$. Zeige, dass dann
auch~$\L^n$ weich ist.

{\tiny\emph{Tipp:} Eine Garbe~$\F$ ist genau dann weich, wenn~$H^1_c(U,\F) = 0$
für alle offenen Teilmengen~$U \subseteq X$. Zerlege die Sequenz in viele
kurze, um~$H^1_c(U,\L^n) \cong H^{n+1}_c(U,\E)$ nachzuweisen.\par}
\item Sei~$X$ durch offene Mengen~$U$ mit~$\dim_c U \leq
n$ überdeckt. Zeige~$\dim_c X \leq n$.

{\tiny\emph{Tipp:} Eine Garbe ist genau dann weich, wenn sie lokal weich
ist.\par}
\item Sei~$f : X \to Y$ eine stetige Abbildung. Sei~$\dim_c X \leq n$.
Zeige~$R^{> n} f_!(\E) = 0$ für alle Garben~$\E$ abelscher Gruppen auf~$X$.
\item Zeige, dass in der Situation aus~d)~$\RR f_!$ als Funktor~$D^b \to D^b$ und
auch als Funktor~$D^- \to D^-$ wohldefiniert ist.

{\tiny\emph{Tipp:} Für den ersten Teil nutze die Spektralsequenz~$E_2^{pq} =
R^p f_! (H^q(\E^\bullet)) \Rightarrow \RR^n f_!(\E^\bullet)$.\par}
\end{enumerate}
\end{aufgabe}

\begin{aufgabe}{Halme des direkten Bilds mit kompaktem Träger}
Vollziehe den Beweis von Proposition~III.8.10 über die Halme des direkten Bilds
mit kompaktem Träger genau nach. Verwende das Buch \emph{Cohomology of Sheaves}
von Birger Iversen, wenn du nicht weiterkommst.
\end{aufgabe}

\begin{aufgabe}{Abgeleitetes Zurückziehen und Vordrücken}
Sei~$f : X \to Y$ ein Morphismus lokal geringter Räume. Zeige, dass~$\LL f^*$
linksadjungiert zu~$\RR f_*$ ist. Wie ist das präzise zu formulieren?
\end{aufgabe}

\begin{center}\emph{-- Für eine ungarbige Aufgabe bitte wenden. --}\end{center}

\newpage

\begin{aufgabe}{Rechnen modulo Torsion}
Sei~$\Ab_\text{fp}$ die abelsche Kategorie der endlich präsentierten abelschen
Gruppen und $\T$ ihre volle Unterkategorie der Torsionsgruppen.
\begin{enumerate}
\item Mache dir klar, dass~$\T$ eine Serresche Unterkategorie
von~$\Ab_\text{fp}$ ist.
\item Konstruiere einen Funktor $\overline{F} : \Ab_\text{fp}/\T \to
\Vect(\QQ)_\text{findim}$ mit $A \mapsto A \otimes_\ZZ \QQ$.

{\tiny\emph{Tipp:} Verwende die universelle Eigenschaft von~$\Ab_\text{fp}/\T$
(siehe Blatt~16, Aufgabe~4) und die Flachheit von~$\QQ$ über~$\ZZ$.\par}

\item Zeige, dass~$\overline{F}$ treu ist.

{\tiny\emph{Tipp:} Zeige, dass aus $A \otimes_\ZZ \QQ = 0$ folgt,
dass~$A$ eine Torsionsgruppe ist. Verwende dann
\href{http://stacks.math.columbia.edu/tag/06XK}{Tag~06XK aus dem Stacks
Project}.\par}
\item Zeige, dass in~$\Ab_\text{fp}/\T$ der Morphismus~$\ZZ \xrightarrow{\cdot
n} \ZZ$ für~$n \geq 1$ invertierbar ist. Folgere, dass~$\overline{F}$
voll und daher eine Kategorienäquivalenz ist.
\item Sei eine konvergente Spektralsequenz in $\Ab_\text{fp}$ gegeben. Was ist
zu tun, wenn man vorgeben möchte, dass alle kurzen exakten Sequenzen
in~$\Ab_\text{fp}$ zerfallen? Wie schwächt man seine Resultate dadurch ab?
\end{enumerate}

{\tiny\emph{Hinweis:} Allgemeiner ist für jeden Integritätsbereich~$R$ der
Quotient der Kategorie der~$R$-Moduln modulo der Unterkategorie der
Torsionsmoduln äquivalent zur Kategorie der Vektorräume über dem
Quotientenkörper von~$R$, siehe
\href{http://stacks.math.columbia.edu/tag/026Z}{Stacks Project, Abschnitt \emph{The
category of modules modulo torsion modules}}.\par}
\end{aufgabe}

\end{document}
