\documentclass{uebblatt}
\haiitrue

\begin{document}

\maketitle{27}{}

\begin{aufgabe}{Rechnen modulo Torsion}
Sei~$\Ab_\text{fp}$ die abelsche Kategorie der endlich präsentierten abelschen
Gruppen und $\T$ ihre volle Unterkategorie der Torsionsgruppen.
\begin{enumerate}
\item Mache dir klar, dass~$\T$ eine Serresche Unterkategorie
von~$\Ab_\text{fp}$ ist.
\item Konstruiere einen Funktor $\overline{F} : \Ab_\text{fp}/\T \to
\Vect(\QQ)_\text{findim}$ mit $A \mapsto A \otimes_\ZZ \QQ$.

{\tiny\emph{Tipp:} Verwende die universelle Eigenschaft von~$\Ab_\text{fp}/\T$
(siehe Blatt~16, Aufgabe~4) und die Flachheit von~$\QQ$ über~$\ZZ$.\par}

\item Zeige, dass~$\overline{F}$ treu ist.

{\tiny\emph{Tipp:} Zeige, dass aus $A \otimes_\ZZ \QQ = 0$ folgt,
dass~$A$ eine Torsionsgruppe ist. Verwende dann
\href{http://stacks.math.columbia.edu/tag/06XK}{Tag~06XK aus dem Stacks
Project}.\par}
\item Zeige, dass in~$\Ab_\text{fp}/\T$ der Morphismus~$\ZZ \xrightarrow{\cdot
n} \ZZ$ für~$n \geq 1$ invertierbar ist. Folgere, dass~$\overline{F}$
voll und daher eine Kategorienäquivalenz ist.
\item Sei eine konvergente Spektralsequenz in $\Ab_\text{fp}$ gegeben. Was ist
zu tun, wenn man vorgeben möchte, dass alle kurzen exakten Sequenzen
in~$\Ab_\text{fp}$ zerfallen? Wie schwächt man seine Resultate dadurch ab?
\end{enumerate}
\end{aufgabe}

\begin{itemize}
\item Dimensionsbehauptungen aus dem Buch (Seite 237f.)
\item Details zum Beweis von Seite 226
\item Eine abgeleitete Adjunktion
\end{itemize}

\end{document}
