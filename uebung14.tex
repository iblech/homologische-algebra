\documentclass{uebblatt}
\haiitrue

\begin{document}

\maketitle{14}{}

\begin{aufgabe}{Definition des Kerns}
Sei~$\C$ eine~$\Ab$-angereicherte Kategorie. Sei~$\varphi : X \to Y$ ein
Morphismus in~$\C$. Zeige, dass für ein Objekt~$K$ zusammen mit einem Morphismus~$k
: K \to X$ folgende Aussagen äquivalent sind:
\begin{enumerate}
\item[1.] Das Paar stellt den Funktor
$\C^\op \to \Ab$, $T \mapsto
  \Kernel(\Hom_\C(T,X) \to \Hom_\C(T,Y))$ dar.
\item[2.] Das Paar hat folgende universelle Eigenschaft:
Zu jedem
Morphismus~$k' : K' \to X$ mit~$\varphi \circ k' = 0$ gibt es genau einen
Morphismus~$h : K' \to K$ mit~$k' = k \circ h$.
\marginpar{\hspace{-1.5cm}$\xymatrixcolsep{2pc}\xymatrixrowsep{2pc}\xymatrix{
  & K' \ar[d]^{k'} \ar[rd]^0 \ar@{-->}[ld]_h \\
  K \ar[r]_k & X \ar[r]_{\varphi} & Y
}$}
\item[3.] Für alle Objekte~$T$ ist folgende Sequenz abelscher Gruppen
exakt.
\[ 0 \longrightarrow \Hom_\C(T,K) \longrightarrow \Hom_\C(T,X)
  \longrightarrow \Hom_\C(T,Y) \]
\end{enumerate}

\emph{Zur Erinnerung:} Ein Paar bestehend aus einem Objekt~$K$ und einem
Element~$k \in F(K)$ stellt genau dann einen Funktor~$F : \C^\op \to \Set$ dar,
wenn es folgende universelle Eigenschaft hat: Für jedes Objekt~$K'$ und jedes
Element~$k' \in F(K')$ existiert genau ein Morphismus~$h : K' \to K$ mit~$k' =
F(h)(k)$.
\end{aufgabe}

\begin{aufgabe}{Kerne und Monomorphismen}
Sei~$\varphi : X \to Y$ ein Morphismus in einer Kategorie~$\C$, die die Axiome~A1
und~A2 erfüllt.
\begin{enumerate}
\item Zeige, dass~$\varphi$ genau dann ein Monomorphismus (d.\,h.
linkskürzbar) ist, wenn das Nullobjekt zusammen mit dem eindeutigen
Morphismus nach~$X$ ein Kern von~$\varphi$ ist.
\item Formuliere und beweise mit wenig Aufwand die duale Aussage.
\item Sei~$\C$ sogar abelsch und~$\varphi$ sowohl ein Mono- als auch ein
Epimorphismus. Zeige, dass~$\varphi$ dann sogar ein Isomorphismus ist.
(Man sagt auch, abelsche Kategorien seien \emph{balanciert}. Welche wichtigen
Kategorien sind nicht balanciert?)
\end{enumerate}
\end{aufgabe}

%\begin{aufgabe}{Differenzkerne}
%Seien~$f,g : X \to Y$ Morphismen in einer~$\Ab$-angereicherten Kategorie.
%Zeige, dass Differenzkerne (Equalizer) von~$X \rightrightarrows Y$ dasselbe
%sind wie Kerne von~$f-g$.
%\end{aufgabe}

\begin{aufgabe}{Rückzug von Mono- und Epimorphismen}
Sei in einer beliebigen Kategorie~$\C$ ein Faserproduktdiagramm gegeben.
\begin{enumerate}
\item Zeige: Ist~$f$ ein Monomorphismus, so auch~$f'$.
\item Sei~$\C$ sogar abelsch. Zeige: Ist~$f$ ein Epimorphismus, so auch~$f'$.

\emph{Tipp:} Vollziehe die Behauptung erst im Fall~$\C = \lMod{R}$ nach. Hole
dir dann bessere Tipps ab.
\end{enumerate}
\[ \xymatrixcolsep{3pc}\xymatrixrowsep{3pc}\xymatrix{
  Y' \ar[r]^{g'} \ar[d]_{f'} & Y \ar[d]^f \\
  X' \ar[r]_g & X
} \]
\end{aufgabe}

\newpage

\begin{aufgabe}{Homotopietheorie von Nerven}
\begin{enumerate}
\item Sei~$\eta : F \to G$ eine natürliche Transformation zwischen
Funktoren~$F,G : \C \to \D$. Zeige, dass die induzierten simplizialen
Abbildungen~$NF, NG : N\C \to N\D$ zwischen den Nerven zueinander homotop
sind.
\item Sei~$F \dashv G$ ein adjungiertes Funktorpaar. Zeige, dass~$NF$ eine
Homotopieäquivalenz ist (mit~$NG$ als schwachem Inversen).
\item Sei~$\C$ eine Kategorie, die ein initiales oder terminales Objekt
besitzt. Zeige, dass~$N\C$ zusammenziehbar ist, d.\,h. homotopieäquivalent zu
einem Punkt.

\emph{Tipp:} Elegant kann man das mit Teilaufgabe~b) lösen.
\end{enumerate}
\end{aufgabe}

\end{document}
