\documentclass{uebblatt}
\haiitrue
\adventurestrue

\begin{document}

\maketitle{28}{}

\vspace{-0.5cm}
\begin{center}
  \setlength{\fboxsep}{0pt}
  \fbox{\includegraphics[scale=0.34]{sheafi}}
\end{center}
\vspace{0.3cm}

\begin{aufgabe}{Verbesserung des Stacks Projects}
Sicherlich kennst du das Stacks Project,
\url{http://stacks.math.columbia.edu/}, das freie Nachschlagewerk zu Themen wie
Garben, Homologische Algebra, Schematheorie und Stacks. Es fasst mehr als 4500
Seiten und ist das Werk vieler Mitwirkenden, mit Aise Johan de
Jong als Hauptautor. Nutze es zum Lernen und korrigiere Fehler! Korrekturen
einzubringen dauert keine fünf Minuten, auf
\url{https://github.com/stacks/stacks-project} ist der Quellcode direkt
editierbar.
\end{aufgabe}

\begin{aufgabe}{Verbesserung des nLab}
Eine Quelle tiefer kategorieller Einsichten und vieler Beispiele ist das nLab,
\url{http://ncatlab.org/nlab/show/HomePage}. Auch zu elementaren Konzepten sind
dort die abstrakten Hintergründe verzeichnet. Nutze und verbessere es!

{\tiny\emph{Hinweis:} Bei Verbesserungen, die über Tippfehlerkorrekturen
hinausgehen, gehört es zum guten Ton, seine Änderungen auf dem \emph{nForum}
bekanntzugeben.\par}
\end{aufgabe}

\begin{aufgabe}{Engagement im Matheschülerzirkel}
Mach beim Augsburger Matheschülerzirkel mit und übernimm die Verantwortung für
einen Präsenz- oder Korrespondenzzirkel oder die Meta-Organisation!
Außerdem werden noch Betreuer fürs \emph{Mathecamp} gesucht (22. bis 27. August
2015). Die Arbeit mit den Kindern bringt viel Spaß und Freude, es lohnt sich sehr.
\end{aufgabe}

\begin{center}\emph{-- Zu ungarbig? Bitte wenden! --}
\end{center}

\newpage

\begin{aufgabe}{Freie Gruppen}
\begin{enumerate}
\item Sei~$A$ eine abelsche Gruppe. Zeige: Ist~$A$ frei, so zerfällt jede kurze
exakte Sequenz der Form~$0 \to \ZZ \to B \to A \to 0$.
\item Gilt die Umkehrung? {\tiny\emph{Hinweis:} Fiese Frage.}
\end{enumerate}
\end{aufgabe}

\begin{aufgabe}{Kohomologie von~$\RR^n$ mit kompaktem Träger}
\begin{enumerate}
\item Verwende die Auflösung~$0 \to \ul{\RR} \to \Omega^\bullet$ der konstanten
Garbe~$\ul{\RR}$ durch die Garben der Differentialformen,
um~$H_c^\bullet(\RR^n,\ul{\RR})$ zu berechnen.
\item Zeige~$\dim_c \RR^n = n$. {\tiny Dazu folgt noch ein ausführlicher Tipp.}
\end{enumerate}
\end{aufgabe}

\begin{aufgabe}{Bewegung längs Immersionen}
\begin{enumerate}
\item Sei~$i : Z \hookrightarrow X$ die Inklusion eines abgeschlossenen
Teilraums. Zeige~$i_* \cong i_!$.
\item Sei~$j : U \hookrightarrow X$ die Inklusion eines offenen Teilraums.
Zeige~$j^* \cong j^!$.
\end{enumerate}
\end{aufgabe}

\begin{aufgabe}{Alexander-Dualität}
Sei~$i : Z \hookrightarrow X$ die Inklusion einer abgeschlossenen Teilmenge
einer~$n$-dimensionalen Mannigfaltigkeit~$X$. Sei~$\E$ eine Garbe
von~$k$-Vektorräumen auf~$Z$. Zeige: \[ H_c^r(Z,\E)^\vee \cong
\Ext_k^{n-r}(i_*\E, \omega_X[n-r]). \]

{\tiny\emph{Hinweis:} Auf der linken Seite tritt die \emph{lokale
Kohomologie}~$H_c^r(Z,\E)$ auf. Die ist die~$r$-te Rechtsableitung des
Funktors~$\Gamma_Z : \AbSh(X) \to \Ab$, der eine Garbe~$\E$ auf die Menge
derjenigen globalen Schnitte, deren Träger in~$Z$ liegt, schickt. \emph{Tipp:}
Verwende~$\Gamma_Z(\E) \cong \Hom_{\AbSh(Z)}(\ZZ,i^!\E)$ und beginne mit dem
Isomorphismus~$H_c^r(Z,\E)^\vee \cong \Hom_{D^+(\Vect(k))}(\RR \Gamma_c(\E),
k[-r])$.\par}
\end{aufgabe}

\begin{aufgabe}{$t$-Strukturen auf triangulierten Kategorien}
Eine \emph{$t$-Struktur} auf einer triangulierten Kategorie~$\C$ besteht aus
zwei vollen Unterkategorien~$\C_{\geq 0}, \C_{\leq 0} \hookrightarrow \C$ mit
folgenden Eigenschaften:
\begin{itemize}
\item Für~$X \in \C_{\leq 0}$ und~$Y \in \C_{\geq 0}$ ist~$\Hom_\C(X,Y[-1]) =
0$.
\item Die Unterkategorie~$\C_{\geq 0}$ ist
unter~$\smallplaceholder[-1]$ und~$\C_{\leq 0}$ ist
unter~$\smallplaceholder[1]$ abgeschlossen.
\item Für jedes Objekt~$X \in \C$ existiert ein ausgezeichnetes Dreieck~$A \to
X \to B \to$ mit~$A \in \C_{\leq 0}$ und~$B \in \C_{\geq 0}[-1]$.
\end{itemize}
\begin{enumerate}
\item Sei~$\A$ eine abelsche Kategorie und~$\C \defeq \D(\A)$ die zugehörige
abgeleitete Kategorie. Sei~$\C_{\geq 0}$ die volle Unterkategorie bestehend aus
denjenigen Komplexen, deren Kohomologie in Graden~$\geq 0$ konzentriert ist.
Sei analog~$\C_{\leq 0}$ definiert. Zeige, dass diese Unterkategorien
eine~$t$-Struktur auf~$\C$ bilden. {\tiny\emph{Tipp:} Gute Abschneidung.}
\item Zeige weiter, dass die Kategorie~$\C_{\geq 0} \cap \C_{\leq 0}$ äquivalent zu~$\A$ ist.
\item Sei eine~$t$-Struktur auf einer allgemeinen triangulierten Kategorie~$\C$
gegeben. Zeige, dass ihr \emph{Herz} -- das ist der der Schnitt~$\C_{\geq 0}
\cap \C_{\leq 0}$ -- eine abelsche Kategorie ist.
\end{enumerate}
{\tiny\emph{Hinweis:} Die abgeleitete Kategorie des Herzens einer $t$-Struktur
auf einer triangulierten Kategorie~$\C$ stimmt nicht unbedingt mit~$\C$
überein. $t$-Strukturen sind wichtig in der Stringtheorie.\par}
\end{aufgabe}

\end{document}

https://math.berkeley.edu/~amathew/verd.pdf
