\documentclass{uebblatt}
\haiitrue

\begin{document}

\maketitle{28}{}

\begin{aufgabe}{Verbesserung des Stacks Projects}
Sicherlich kennst du das Stacks Project,
\url{http://stacks.math.columbia.edu/}, das freie Nachschlagewerk zu Themen wie
Garben, Homologische Algebra, Schematheorie und Stacks. Es fasst mehr als 4500
Seiten und ist das Werk vieler Mitwirkenden unter der Leitung von Aise Johan de
Jong. Nutze es zum Lernen und korrigiere Fehler! Korrekturen einzubringen
dauert keine fünf Minuten, auf \url{https://github.com/stacks/stacks-project}
ist der Quellcode direkt editierbar.
\end{aufgabe}

\begin{aufgabe}{Verbesserung des nLab}
Eine Quelle tiefer kategorieller Einsichten und vieler Beispiele ist das nLab,
\url{http://ncatlab.org/nlab/show/HomePage}. Auch zu elementaren Konzepten sind
dort die abstrakten Hintergründe verzeichnet. Nutze und verbessere es!

{\tiny\emph{Hinweis:} Bei Verbesserungen, die über Tippfehlerkorrekturen
hinausgehen, gehört es zum guten Ton, seine Änderungen auf dem \emph{nForum}
bekanntzugeben.\par}
\end{aufgabe}

\end{document}
